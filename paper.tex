\documentclass[11pt]{amsart}
\usepackage{amscd, amsmath, amssymb, amsthm, amsfonts, amstext, verbatim, enumitem, graphicx, mathtools, xfrac, tikz-cd, microtype, nameref, thmtools}
\usepackage[breaklinks=true]{hyperref}
\usepackage[capitalize]{cleveref}

\definecolor{darkblue}{rgb}{0.0, 0.0, 0.6}

\hypersetup{
  pdfauthor={Sergey Sinchuk, Andrei Smolensky},
  pdftitle={Decompositions of congruence subgroups of Chevalley groups},  
  colorlinks=true,
  urlcolor=darkblue,
  linkcolor=darkblue,
  citecolor=darkblue}

\usepackage[hyperref=true, backend=bibtex, citestyle=numeric-comp, sortlocale=en_US, url=false, doi=false, eprint=true, maxbibnames=4]{biblatex}            
\addbibresource{paper.bib}
\renewbibmacro*{volume+number+eid}{\ifentrytype{article}{\- \iffieldundef{volume}{}{Vol.~\printfield{volume},}\iffieldundef{number}{}{ No.~\printfield{number},}}}
\renewbibmacro{in:}{\ifentrytype{article}{}{\printtext{\bibstring{in}\intitlepunct}}}
\newbibmacro{string+doi}[1]{\iffieldundef{doi}{\iffieldundef{url}{#1}{\href{\thefield{url}}{#1}}}{\href{https://doi.org/\thefield{doi}}{#1}}}
\DeclareFieldFormat[article, inproceedings, inbook, thesis]{title}{\usebibmacro{string+doi}{\mkbibquote{#1}}}
\renewcommand*{\bibfont}{\footnotesize}

\usepackage[margin=2cm]{geometry}

\newlist{thmlist}{enumerate}{1} \setlist[thmlist]{label={\rm(\arabic{thmlisti})}, ref=\thethm.(\roman{thmlisti}),noitemsep} \Crefname{thmlisti}{Theorem}{Theorems}
\newlist{lemlist}{enumerate}{1} \setlist[lemlist]{label={\rm(\arabic{lemlisti})}, ref=\thelemma.(\roman{lemlisti}),noitemsep} \Crefname{lemlisti}{Lemma}{Lemmas}

\theoremstyle{plain}

\newtheorem{thm}{Theorem}
\Crefname{thm}{Theorem}{Theorems}
\numberwithin{equation}{section}

\newtheorem{lemma}{Lemma}
\numberwithin{lemma}{section}
\Crefname{lemma}{Lemma}{Lemmas}

\newtheorem{cor}[lemma]{Corollary}
\Crefname{cor}{Corollary}{Corollaries}

\newtheorem{prop}[lemma]{Proposition}
\Crefname{prop}{Proposition}{Propositions}

\newtheorem*{thm*}{Theorem}
\newtheorem*{lemma*}{Lemma}

\theoremstyle{definition}

\newtheorem{dfn}[lemma]{Definition}
\Crefname{dfn}{Definition}{Definitions}
\newtheorem{example}[lemma]{Example}
\Crefname{example}{Example}{Examples}

\theoremstyle{remark}

\newtheorem{rem}[lemma]{Remark}
\Crefname{rem}{Remark}{Remarks}

\DeclareMathOperator{\K}{K}
\DeclareMathOperator{\SK}{SK}
\DeclareMathOperator{\G}{G}
\DeclareMathOperator{\GL}{GL}
\DeclareMathOperator{\SL}{SL}
\DeclareMathOperator{\Sp}{Sp}
\DeclareMathOperator{\SO}{SO}
\DeclareMathOperator{\St}{St}
\DeclareMathOperator{\E}{E}
\DeclareMathOperator{\EP}{EP}
\DeclareMathOperator{\Par}{P}
\DeclareMathOperator{\Hom}{Hom}
\DeclareMathOperator{\B}{B}
\DeclareMathOperator{\Hh}{H}
\DeclareMathOperator{\U}{U}
\DeclareMathOperator{\Z}{Z}
\DeclareMathOperator{\M}{M}
\DeclareMathOperator{\SR}{SR}
\DeclareMathOperator{\sr}{sr}
\DeclareMathOperator{\asr}{asr}
\DeclareMathOperator{\shape}{shape}
\DeclareMathOperator{\Rad}{Rad}
\DeclareMathOperator{\Max}{Max}
\DeclareMathOperator{\Spec}{Spec}
\DeclareMathOperator{\Spin}{Spin}
\DeclareMathOperator{\Epin}{Epin}
\DeclareMathOperator{\Stab}{Stab}
\DeclareMathOperator{\ASR}{ASR}
\DeclareMathOperator{\Ums}{Ums}
\DeclareMathOperator{\Umd}{Umd}
\DeclareMathOperator{\rk}{rk}
\newcommand{\rA}{\mathsf{A}}
\newcommand{\rB}{\mathsf{B}}
\newcommand{\rC}{\mathsf{C}} 
\newcommand{\rD}{\mathsf{D}} 
\newcommand{\rE}{\mathsf{E}}
\newcommand{\rF}{\mathsf{F}}
\newcommand{\rG}{\mathsf{G}}

\makeatletter
\newcommand{\indexbox}[1]{\text{\fboxsep=.1em\fbox{\m@th$\displaystyle#1$}}}
\makeatother
   
\title[Decompositions of congruence subgroups of Chevalley groups]{Decompositions of congruence subgroups \\ of Chevalley groups}
\keywords {parabolic decomposition, bounded generation, $\SL_2$-factorization, subsystem factorization, product decomposition. {\em Mathematical Subject Classification (2010):} 20G40, 20G35, 19B14}
\author{Sergey Sinchuk}
\address{Chebyshev laboratory, St. Petersburg State University, St. Petersburg, Russia}
\email{sinchukss {\it at} yandex.ru;}

\author{Andrei Smolensky}
\address{Department of Mathematics and Mechanics, St. Petersburg State University, St. Petersburg, Russia}
\email{andrei.smolensky {\it at} gmail.com}
\thanks{The authors of the present paper acknowledge the financial support of the Russian Science Foundation grant 14-11-00297}

\date {\today}

\begin{document}

\begin{abstract} 
We formulate and prove relative versions of several decompositions known in the theory of Chevalley groups over commutative rings.
These decompositions are used to obtain factorizations in terms of subsystem subgroups of type $\rA_\ell$ and 
 upper estimates of the width of principal congruence subgroups in terms of Tits--Vaserstein generators.
Some of our results are new even in the absolute case and were previously studied only for groups over finite fields.
\end{abstract}

\maketitle

\section{Introduction}\label{sec:intro}
The aim of this paper is to generalize several decompositions of Chevalley groups over rings 
 to their principal congruence subgroups and relative elementary subgroups.
We focus on the so-called ``parabolic decompositions'', i.e. decompositions involving parabolic and unipotent factors. 
One of our main results is the following theorem.
\begin{thm}[Dennis--Vaserstein decomposition]\label{thm:DennisVaserstein}
Let $\Phi$ be an irreducible root system of rank $\ell\geqslant 2$ and $\{ r, s \}$ be a pair of distinct endnodes of the Dynkin diagram of $\Phi$.
Denote by $d$ the distance between $r$ and $s$ on the Dynkin diagram.
Assume that $\Phi$, $I$ and $\{r, s\}$ satisfy either of the following assumptions:
\begin{thmlist}
 \item $\sr(I) \leqslant d$ for $\Phi$ classical or $\Phi=\rG_2, \rF_4$;
 \item $\sr(I) \leqslant d$ for $\Phi=\rE_6,\rE_7$ with $\{r, s\} = \{2, \ell\}$;
 \item $\asr(I) \leqslant d$ for $\Phi=\rE_6,\rE_7$ with $\{r, s\} = \{1, \ell \}$, 
\end{thmlist}
Then $\E(\Phi, R, I) = \EP_r(R, I) \cdot \U(\Sigma^-_r \cap \Sigma^-_s, I) \cdot \EP_s(R, I).$
\end{thm}
Even in the special case $I=R$ the above theorem contains several new cases as compared to~\cite[Theorems~2.5 and~4.1]{St78}.
The condition $\asr(I) \leq d$ is a relative version of the \emph{absolute stable range} condition of D.~Estes and J.~Ohm, which depends only on the ideal $I$ (see section~\ref{sec:rel-asr}).
Another results of the paper are relative analogues of Gauss and Bass--Kolster decompositions (see~\cref{thm:srRI1,thm:BassKolster}).

We apply our decompositions to several problems. First, we obtain upper estimates of the number of factors in the presentation of a Chevalley group as a product of its subsystem subgroups of type $\rA_\ell$.
For example, in~\cref{thm:SL2width} such estimates are given for relative classical groups and products of subgroups $\SL(2, R, I)$.
Our another result is the decomposition in terms of subgroups of type $\rA_\ell$ of submaximal rank. 
Using~\cref{thm:DennisVaserstein} we prove the following theorem inspired by the main result of~\cite{Nik07}.
\begin{thm}\label{thm:spin-sln-prod}
Assume that $\sr(I) \leqslant 2$. Then the group $\Epin(2\ell, R, I)=\E(\rD_\ell, R, I)$ is a product of at most $9$ regularly embedded subgroups of type $\rA_{\ell-1}$.
\end{thm}

Finally, we obtain upper bounds on the width of $\E(\Phi, R, I)$ with respect to Tits--Vaserstein generators for several important classes of small-dimensional rings (see~\cref{thm:width}). 

\section{Preliminaries}\label{sec:preliminaries}
Let $\Phi$ be a reduced irreducible root system of rank $\ell$ with a fixed basis of fundamental roots $\Pi=\{\alpha_1, \ldots, \alpha_l\}$.
We use the conventional numbering of basis vectors of $\Pi$ which follows Bourbaki (see~\cite[Table~1]{PSV98}).

For a root $\alpha\in\Phi$ we denote by $m_i(\alpha)$ the $i$-th coefficient in the expansion of $\alpha$ in $\Pi$,
 i.\,e. $\alpha = \sum_{i=1}^n m_i(\alpha) \alpha_i$.

A proper closed subset of roots $S\subseteq \Phi$ is called {\it parabolic} (resp. {\it reductive}, resp. {\it special}) if $\Phi=S \cup -S$ (resp. $S = -S$, resp. $S \cap -S=\varnothing$).
Any parabolic subset $S \subseteq \Phi$ can be decomposed into the disjoint union of its \emph{reductive} and \emph{special} parts, i.\,e. 
$S = \Sigma_S \sqcup \Delta_S$, where $\Sigma_S \cap (-\Sigma_S) = \varnothing$, $\Delta_S = -\Delta_S$.

We denote by $(\alpha, \beta)$ the scalar product of roots and by $\langle \alpha, \beta\rangle$ the integer $2(\alpha, \beta)/(\beta, \beta)$.
Let $\varpi_1, \ldots, \varpi_\ell$ be the \emph{fundamental weights }of $\Phi$.

Denote by $W(\Phi)$ the subgroup of isometries of $\Phi$ generated by all simple reflections $\sigma_\alpha$, $\alpha\in\Phi$.
For a reductive subset $\Delta\subseteq \Phi$ denote by $W(\Delta)$ the subgroup of $W(\Phi)$ generated by $\sigma_\alpha$ for $\alpha\in\Delta$.

Let $J\subseteq \Pi$ be a subset of fundamental roots. Define
\begin{align*}
& \shape(J, \beta)=\sum\nolimits_{i\in J} m_i(\beta) \alpha_i, \\
& \Delta_J = \{\alpha \in \Phi \mid \shape(J, \alpha)=0\}, \\
& \Sigma^\pm_J = \{\alpha \in \Phi \mid \shape(J, \alpha) \in \mathbb{Z}_{\geqslant0} \Phi^\pm \}, \\
& S_J^\pm = \Delta_J \sqcup \Sigma_J^\pm.
\end{align*}
Clearly, $\Delta_J$ is a reductive subset, while $S^\pm_J$ and $\Sigma^\pm_J$ are parabolic and special subsets, respectively.
For two disjoint subsets $I, J\subseteq \Pi$ one has 
\[ \Sigma^\pm_{I \cup J} = \Sigma^\pm_I\cup\Sigma^\pm_J, \quad \Delta_{I\cup J} = \Delta_I \cap \Delta_J. \]
We omit curly braces in the above notations when $J$ is a one- or two-element set, e.\,g. $\Delta_k=\Delta_{\{k\}}$ and $\Sigma_{i, j}^\pm=\Sigma_{\{i, j\}}^\pm$, etc.
We call $\shape(\{\alpha_s\}, \beta)$ simply the $s$-shape of $\beta$.

\begin{lemma}[{\cite[Lemma~1]{ABS}}]\label{lemma:abs}
Let $\alpha, \beta \in \Sigma^\pm_J$ be a pair of roots having the same length such that $\shape(J, \alpha)=\shape(J, \beta)\neq 0$.
Then $\alpha$ and $\beta$ are conjugate under the action of $W(\Delta_J)$.
\end{lemma}

Let $\G(\Phi, R)$ be the simply-connected Chevalley group of type $\Phi$ over an arbitrary commutative ring $R$ and let $\E(\Phi, R)$ be its elementary subgroup,
i.\,e. the subgroup generated by the elementary root unipotents $x_\alpha(\xi)$, $\alpha\in\Phi$, $\xi\in R$, see~\cite{VP, St78, S}.
For an ideal $I \trianglelefteq R$ we denote by $\G(\Phi, R, I)$ the \emph{principal congruence subgroup} of level $I$.

The \emph{relative elementary Chevalley subgroup} $\E(\Phi, R, I) \leqslant \G(\Phi, R, I)$ is defined as the normal closure in $\E(\Phi, R)$ of the subgroup $\E(\Phi, I)$ generated by the set $\mathcal{X} = \{x_\alpha(\xi) \mid \alpha\in \Phi, \ \xi \in I \}$.

Set $z_\alpha(s, \xi) = x_{\alpha}(s)^{x_{-\alpha}(\xi)}$, $s\in I$, $\xi\in R$.
It is a classical result of J.~Tits and L.~Vaserstein (see~\cite{Tits76}, \cite[Theorem~2]{Va86}) that the relative elementary subgroup $\E(\Phi, R, I)$ is generated by the set of elements $\mathcal{Z}(\Phi)$.
Here for a subset of roots $S\subseteq\Phi$ we denote by $\mathcal{Z}(S)$ the set
\[ \mathcal{Z}(S) =  \mathcal{X} \cup \{ z_\alpha(s, \xi) \mid s\in I, \ \xi \in R, \ \alpha \in S\}.\]
\begin{lemma}[{\cite[Corollary~3.3]{S}}]\label{lemma:Stepanov-ideal}
Let $\Phi$ be a root system of rank $\geqslant2$, let $R$ be a commutative ring and $I\trianglelefteq R$ be its ideal.
If $\Phi\neq\rC_\ell$ then $\E\left(\Phi, R, I^2\right)\leqslant\E(\Phi, I)$, otherwise $\E\left(\Phi, R, II^{\indexbox{2}}\right)\leqslant\E(\Phi, I)$.
\end{lemma}
Here $I^{\indexbox{2}}$ denotes the ideal generated by the squares $a^2$, where $a\in I$.

For $\varepsilon\in R^*$ set $w_\alpha(\varepsilon) = x_\alpha(\varepsilon) x_{-\alpha}(-\varepsilon^{-1}) x_{\alpha}(\varepsilon).$
If $\rk(\Phi)\geqslant 2$ the following relation holds:
\begin{equation}\label{rel:R3}
w_\alpha(\varepsilon) x_{\beta}(\xi) w_\alpha(\varepsilon)^{-1} =
x_{\sigma_{\alpha}\beta} \left(\eta_{\alpha, \beta}\cdot\varepsilon^{-\langle\beta, \alpha \rangle}\xi\right), \quad \varepsilon\in R^*, \ \xi\in R.
\end{equation}
where $\eta_{\alpha, \beta}=\pm 1$. The coefficients $\eta_{\alpha, \beta}$ can be expressed in terms of the structure constants of the corresponding Lie algebra (see~\cite[\S13]{VP}).
For a reductive subset $\Delta \subseteq \Phi$ denote by $\widetilde{W}(\Delta)$ the \emph{extended Weyl group}, i.\,e. the subgroup of $\E(\Phi, R)$ generated by all $w_{\alpha}(1)$, $\alpha \in \Delta$.
\begin{lemma} \label{lemma:weylfacts} Let $\Phi$ be an irreducible root system and let $I$ be an ideal of $R$. 
\begin{lemlist}
\item \label{item-trans1} For every two roots $\alpha, \beta \in \Phi$ of the same length there exists $w \in \widetilde{W}(\Phi)$ such that $X_{\alpha}(I)^w = X_\beta(I)$.
\item \label{item-trans2} Let $\alpha_s\in \Pi$ be a fundamental root and $\alpha, \beta \in \Sigma^\pm_s$ be two roots having the same length and $s$-shape.
 Then there exists $w\in \widetilde{W}(\Delta_s)$ such that $X_\alpha(I)^w = X_\beta(I)$. \end{lemlist}
\end{lemma}
\begin{proof}
The first assertion follows from \eqref{rel:R3} and the well-known fact that $W(\Phi)$ acts transitively on the set of roots of the same length.
The second assertion follows from \eqref{rel:R3} and \cref{lemma:abs}.
\end{proof}

Define the semisimple root elements $h_\alpha(\varepsilon)$, $\alpha\in\Phi$, $\varepsilon\in R^*$ as $h_\alpha(\varepsilon)=w_\alpha(\varepsilon)w_\alpha(-1)$.
The elements $h_\alpha(\varepsilon)$ satisfy the following relation:
\begin{alignat}{2} 
& \phantom{[} h_\alpha(\varepsilon)x_\beta(\xi)h_\alpha(\varepsilon)^{-1} = x_\beta\left(\varepsilon^{\langle\beta, \alpha\rangle}\xi\right), && \alpha, \beta \in \Phi, \ \varepsilon \in R^*, \xi \in R. \label{rel:h-w}
\end{alignat}

For a special subset of roots $\Sigma\subseteq \Phi$ we denote by $\U(\Sigma, I)$ the subgroup spanned by all $x_{\alpha}(I)$ for $\alpha\in \Sigma$. 
For $J\subset\Pi$ the subgroup $\U(\Sigma_J^+, I)$ is normalized by $\E(\Delta_J, R)$, hence the Minkowski product set $\EP_J(R, I) = \E(\Delta_J, R, I) \cdot \U(\Sigma_J^+, I)$ is a subgroup, which we call a \emph{standard elementary parabolic subgroup}.
The following two equalities will be referred to in the sequel as \emph{Levi decomposition}:
\begin{equation} \label{rel:Levi-decomp} \EP_J(R, I) = \U(\Sigma_J^+, I) \cdot \E(\Delta_J, R, I) = \E(\Delta_J, R, I) \cdot \U(\Sigma_J^+, I). \end{equation}
When $J = \{ \alpha_s \}$ for some $1 \leq s\leq \ell$, we use the shorthand $\EP_s(R, I)$ for $\EP_{\{s\}}(R, I)$.
In the case $I=R$, we also write $\EP_J(R)$ instead of $\EP_J(R, R)$.

Denote by $\Hh(\Phi, R)$ the subgroup generated by all $h_\alpha(\varepsilon)$, $\alpha\in\Phi$, $\varepsilon\in R^*$, and set
\[ \Hh(\Phi, R, I) = \Hh(\Phi, R)\cap\G(\Phi, R, I)=\langle h_\alpha(\varepsilon), \ \alpha\in\Phi, \ \varepsilon\in R^*\cap(1+I)\rangle. \]
Consider an element $\varepsilon\in R^*\cap(1+I)$, and rewrite it as $\varepsilon=1+s$, $s\in I$. The element $h_\alpha(\varepsilon)$ can be expressed as
\begin{multline} \label{eq:rel-tor-elementary}
h_\alpha(1+s) = x_\alpha\left(-1\middle)\, x_{-\alpha}\middle(-s\middle)\, x_\alpha\middle((1+s)^{-1}\middle)\, x_{-\alpha}\middle(s(1+s)\right) = \\
= x_\alpha\left((1+s)^{-1}-1\middle)\, z_{-\alpha}\middle(-s, (1+s)^{-1}\middle)\, x_{-\alpha}\middle(s(1+s)\right).
\end{multline}
Notice that $(1+s)^{-1}\in 1+I$ hence all the above factors lie in $\E(\Phi, R, I)$. 
 This shows, in particular, that $\Hh(\Phi, R, I) \leqslant \E(\Phi, R, I)$.

The following lemma is, in essence, a relative version of the classical Chevalley--Matsumoto decomposition.
It can be easily deduced from the absolute statement (see e.\,g.~\cite[Theorem~1.3]{St78}).
\begin{lemma}\label{lemma:Chevalley-Matsumoto}
Let $\pi$ be a basic fundamental representation of $\G_{sc}(\Phi, R)$ with the highest weight $\varpi_s$.
Assume that $g\in \G_{sc}(\Phi, R, I)$ is such that $(g\cdot v^+)_{\varpi_{s}}=1$, then $g \in \U(\Sigma_s^-, I) \cdot \G_{sc}(\Delta_s, R, I) \cdot \U(\Sigma_s^+, I)$.
\end{lemma}

\section{Stability conditions}\label{sec:stability-conditions}
In this section we define the stability conditions used in the statements of our decomposition theorems in section~\ref{sec:factorizations}.
First, we recall the definition of the stable rank of an ideal introduced by L.~Vaserstein in~\cite{Va69, Va71}.
As we will be mainly concerned with applications to Chevalley groups, our exposition is limited to commutative rings.
Next, in section~\ref{sec:rel-asr} we define relative version of the absolute stable range condition introduced in~\cite{EO, MKV}.
Finally, we formulate and prove several technical lemmas describing the action of certain unipotent subgroups on unimodular columns under the assumption of suitable stability conditions.

\subsection{Relative stable rank}
Recall that a row $a\in{}^n\!R$ is called \emph{$I$-unimodular} if it is congruent to $(1, 0, \ldots, 0)$ modulo $I$ and its components $a_1, a_2, \ldots, a_n$ generate $R$ as an ideal.
A column $b \in R^n$ is called $I$-unimodular if its transpose $b^t$ is an $I$-unimodular row.
We denote the set of all $I$-unimodular rows (resp. columns) by $\Umd(n, I)$ (resp. $\Ums(n, I)$).
When $I=R$, we refer to $R$-unimodular rows and columns as simply \emph{unimodular}.
It is not hard to show that for an $I$-unimodular row $a$ there exists an $I$-unimodular column $b$ such that $ab=1$ (see~\cite[\S2]{Va69}).

An $I$-unimodular row $a=(a_1, \ldots, a_{n+1})$ is called \emph{stable} if one can choose $b_1, \ldots, b_n\in I$ such that the row $(a_1+a_{n+1}b_1, \ldots, a_n+a_{n+1}b_n)$ is also $I$-unimodular. 
We say that $I$ satisfies the stable range condition $\SR_n(I)$ if any $I$-unimodular row of length $n+1$ is stable.
For $m \geqslant n$ the condition $\SR_n(I)$ implies $\SR_m(I)$, see~\cite[Theorem~1]{Va71}.
It is also clear that the condition $\SR_n(I)$ does not depend on the choice of the ambient ring $R$.
By definition, the \emph{stable rank} of $I$ (denoted $\sr(I)$) is the smallest natural number $n$ such that $\SR_n(I)$ holds (we set $\sr(I)=\infty$ if no such $n$ exists).

\subsection{Relativization of the absolute stable rank}\label{sec:rel-asr}
For a row $a=(a_1, \ldots, a_n)\in{}^n\!R$ denote by $\mathfrak{J}(a)$ the intersection of all maximal ideals of $R$ containing $a_1, \ldots, a_n$.
It is easy to see that a row $a\in R^n$ is unimodular if and only if $\mathfrak{J}(a)=R$. 
Clearly, for any $g\in\GL(n,R)$ one has $\mathfrak{J}(a\cdot g)=\mathfrak{J}(a)$.
\begin{dfn}\label{dfn:j-stable}
We say that a row $a=(a_1, \ldots, a_{n+1})\in{}^{n+1}\!I$ can be \emph{$I$-shortened}, if there exist $c_1, \ldots, c_n\in I$ such that
$\mathfrak{J}(a_1, \ldots, a_{n+1})=\mathfrak{J}(a_1+c_1a_{n+1}, \ldots, a_n+c_na_{n+1}).$
\end{dfn}
\begin{dfn}\label{dfn:asr}
We say that an ideal $I$ satisfies the condition $\ASR_n(I)$ if it satisfies $\SR_n(I)$ and, moreover, any row $a\in{}^{n+1}\!I$ can be $I$-shortened.
\end{dfn}

It is easy to see that $\ASR_m(I)$ implies $\ASR_n(I)$ for any $n\geqslant m$. 
By definition, the \emph{absolute relative stable rank} $\asr(I)$ is the smallest natural $n$ such that $\ASR_n(I)$ holds (again we set $\asr(I)=\infty$ if no such $n$ exists).

A priori our definition of $\ASR_n(I)$ depends on $R$.
Below we will see that in fact there is no such dependence.
The following lemma is a relative version of~\cite[Lemma~8.2]{MKV}. 
\begin{lemma}\label{lemma:relative-asr-unimod}
For a commutative ring $R$ and an ideal $I \trianglelefteq R$ the following statements are equivalent:
\begin{lemlist} 
\item\label{asr-j-stable} Any row $a\in{}^{n+1}\!I$ can be $I$-shortened;
\item\label{asr-bak-like} For any $I$-unimodular row $(b, a_1, \ldots, a_n, d)\in\Umd(n+2, I)$ there exist $c_1, \ldots, c_n\in I$ 
 such that $(b+b', a_1+c_1d, \ldots, a_n+c_nd)$ is $I$-unimodular for any $b'\in J$, where $J=I \cdot a_1 + \ldots + I \cdot a_n + I \cdot d\leqslant I$.
\end{lemlist} 
\end{lemma}
\begin{proof}
Assume first that any row $a\in{}^{n+1}\!I$ can be $I$-shortened. 
In particular, for a given $I$-unimodular row $(b, a_1, \ldots, a_n, d)\in\Umd(n+2, I)$ there exist $c_1, \ldots, c_n$ such that
\[\mathfrak{J}(a_1, \ldots, a_{n+1})=\mathfrak{J}(a_1+c_1a_{n+1}, \ldots, a_n+c_na_{n+1}).\]
Therefore $(b,a_1+c_1d,\ldots,a_n+c_nd)$ is also unimodular. Of course, for any $b'\in J$ we could replace $b$ with $b+b'$ from the very start.

To show the converse take an arbitrary row $(a_1, \ldots, a_n, d)\in{}^{n+1}\!I$ and consider the $I$-unimodular row $(1, a_1, \ldots, a_n, d)\in\Umd(n+2, I).$
By the hypothesis, there exist $c_1, \ldots, c_n\in I$ such that
\[ v=(1+b', a'_1, \ldots, a'_n)=(1+b', a_1+c_1d, \ldots, a_n+c_nd) \]
is unimodular for any $b'\in J$.
Assume that there exists a maximal ideal $\mathfrak{m}\trianglelefteq R$ such that all $a'_1, \ldots, a'_n$ are contained in $\mathfrak{m}$, but at least one of the elements $d$, $a_i$ is not.
Then clearly $d\notin\mathfrak{m}$ and $I\not\subseteq \mathfrak{m}$ (otherwise $a_i=a'_i-c_id\in\mathfrak{m}$, contrary to the assumption).
Now we can find $t\in I$ such that its image $\bar{t}$ in the residue field $R/\mathfrak{m}$ equals $-\bar{1}/\bar{d}$.
This means that $1 + b' \in \mathfrak{m}$ for $b'=td\in J$, which contradicts the unimodularity of $v$.
This shows that no such $\mathfrak{m}$ may exist and, therefore, $\mathfrak{J}(a'_1, \ldots, a'_n)=\mathfrak{J}(a_1, \ldots, a_n, d)$.
\end{proof}

Obviously, the second statement of \cref{lemma:relative-asr-unimod} does not depend on $R$, hence, as suggested by the notation, $\asr(I)$ is independent of $R$.

Let $R$ be a commutative ring. We denote by $\Max(R)$ its \emph{maximum spectrum}, i.e. the set of maximal ideals of $R$, equipped with the Zariski topology. For a topological space $X$ denote by $\dim(X)$ its usual topological dimension.
From the definition of $\asr(I)$ and \cite[Theorem~2.3]{EO} (or~\cite[Theorem~3.7]{MKV}) it follows that
\begin{equation} \label{sr-estimates} \sr(I)\leqslant\asr(I)\leqslant\asr(R)\leqslant \dim(\Max(R))+1\leqslant\dim(\Spec(R))+1. \end{equation}

\subsection{Action of unipotent radicals}\label{sec:ur-action}

In this section we work with standard vector representations of classical groups (i.\,e. representations with highest weight $\varpi_1$).
It will be convenient for us to number the weights of these representations as in~\cite[\S~1B]{St78}:
For example, we write $1$ instead of $\varpi_1$, $2$ instead of $\varpi_1-\alpha_1$ etc.

Let $\lambda_1, \lambda_2 \in \Lambda(\pi)$ be a pair of weights of a representation $\pi$ such that $\lambda_1-\lambda_2\in \Phi$.
It will be convenient for us to write $x_{\lambda_1, \lambda_2}(\xi)$ instead of $x_{\lambda_1-\lambda_2}(\xi)$.
For example, for $\Phi=\rA_\ell$ we have $x_{1, 2}(\xi)=x_{1-2}(\xi)=x_{\varpi_1 - \varpi_1 + \alpha_1}(\xi) = x_{\alpha_1}(\xi)$.

\begin{lemma}\label{lemma:PSV-symplectic-trick}
 Let $v=(v_1, \ldots, v_n)^t$ be a column. Denote by $v'$ the vector composed of squares of the components of $v$, i.\,e. $v'=(v_1^2, \ldots, v_n^2)^t$.
 Then for any matrix $b \in M(n, I)$ one can find a symmetric matrix $a \in M(n, I)$, $a=a^t$ such that $b \cdot v' = a \cdot v$. \end{lemma}
\begin{proof}
Straightforward computation shows that the assertion of lemma holds for the matrix $a$ defined by
\begin{equation*}
a_{ij} = b_{ij} v_{j} + b_{ji} v_{i},\ j\neq i,\quad a_{ii} = b_{ii} v_{i} - \sum\limits_{\mathclap{j=1,\ j\neq i}}^\ell b_{ji} v_{j}. \qedhere
\end{equation*}
\end{proof}

Let $v\in V=R^{2\ell}$ be a vector of the natural representation of $\G(\rD_\ell, R)$.
Denote by $v_+$ and $v_-$ the upper and the lower halves of $v$, i.\,e. $v_+=(v_1, \ldots, v_\ell)^t$, $v_-=(v_{-\ell}, \ldots, v_{-1})^t$.
\begin{lemma}\label{lemma:asrUnip}
For any $I$-unimodular column $v=(v_+, v_-)^t\in\Ums(2\ell, I)$ there exists $g\in\U(\Sigma^+_\ell, I) \leq \E(\Phi, R, I)$ 
such that $(g \cdot v)_+ \in \Ums(\ell, I)$ under the following assumptions on $\Phi$ and $I$.
\begin{lemlist}
\item \label{item:asrUnipC} $\Phi=\rC_\ell$ and $\sr(I) \leqslant \ell$;
\item \label{item:asrUnipD} $\Phi=\rD_\ell$ and $\asr(I)\leqslant \ell -1$.
\end{lemlist}
\end{lemma}
\begin{proof} 
\textsc{Case $\Phi=\rC_\ell$.}
Denote by $p$ the matrix of size $\ell$ such that its only nonzero entries equal $1$ and are on the skew-diagonal, i.\,e. $p_{ij}=\delta_{i, \ell-j+1}$. 
For $b \in M(\ell, I)$ set $g(b)=\left(\begin{smallmatrix} e_\ell & p \cdot b \\ 0 & e_{\ell} \end{smallmatrix}\right)$.
Clearly, if $b$ is symmetric then $g(b)$ lies in $\U(\Sigma_\ell^+, I)\leq \E(\rC_\ell, R, I)$.

Notice that the column $v'=(v_1,\ldots,v_\ell, v_{-\ell}^2,\ldots,v_{-1}^2)^t$ is $I$-unimodular.
By the definition of the relative stable rank we can find a matrix $b\in \M(\ell,I)$ such that the upper half $v''_+$ of the vector $v''= g(b) \cdot v'$ is $I$-unimodular.
It is clear that $v''_+ = v_+ + pb v'_-$. 
Finally, applying \cref{lemma:PSV-symplectic-trick}, we find a symmetric matrix $a$ such that
\[ (g(a)\cdot v)_+=v_+ + pav_- = v_+ + pbv'_- = v''_+ \in \Ums(\ell, I). \]

\textsc{Case $\Phi=\rD_\ell$.} Denote by $J$ the ideal of $R$ spanned by the components of $v_{-}$. Clearly, $J \subseteq I$.
Since $\sr(I/J) \leqslant \ell-1$, the elementary group $\E(\rA_{\ell-1}, R/J, I/J)$ acts transitively on $\Ums(\ell, I/J)$ (see~\cite[Theorem~2.3c]{Va69}). 
This implies the existence of an element $h\in \E(\Delta_\ell, R, I)$ such that the vector $v' = h \cdot v$ satisfies $v'_i \equiv \delta_{i1} \pmod J$ for $i=1, \ldots, \ell$.

Clearly, $(v'_1, v'_{-\ell}, \ldots, v'_{-1})$ is $I$-unimodular.
Applying \cref{asr-bak-like}, we find $c_2, \ldots, c_\ell\in I$ such that for $v''= \prod_{i=2}^{\ell}x_{-i, -1}(c_i)\cdot v'$ one has
$(v''_1, v''_{-\ell},\ldots, v''_{-2})\in\Ums(\ell+1, I)$.
Now, applying the condition $\sr(I) \leqslant \ell-1$ once again, we find
$d_1, d_3, \ldots, d_{\ell}\in I$ such that the entries $(v'''_1, v'''_{-\ell}, \ldots, v'''_{-3})$
of $v'''=x_{-2, 1}(d_1) \cdot \prod_{i=3}^{\ell} x_{-2, -i}(d_i) \cdot v''$ form an $I$-unimodular column.

We can find $f_1, f_3,\ldots, f_\ell \in R$ such that $f_1v'''_1+\sum_{i=3}^\ell f_i v'''_{-i} = 1$.
Set $\xi = v'''_1-v'''_2-1 \in I$, $v^{(4)}=x_{1,2}(\xi f_1) \cdot \prod_{i=3}^\ell x_{-i,2}(\xi f_i) \cdot v'''$.
Clearly $v^{(4)}_2 = v^{(4)}_1-1$, therefore $v^{(4)}_+$ is $I$-unimodular.
Summarizing the above, we have found $g\in \EP_\ell(R, I)$ such that $v^{(4)}=g \cdot v$
and the assertion of the lemma immediately follows from the Levi decomposition. \end{proof}

\begin{lemma} \label{lemma:uraction} 
Let $\Phi=\rA_\ell, \rC_\ell, \rD_\ell$. Denote by $\pi$ the natural representation of $\G(\Phi, R)$ on $V=R^n$, $n=\ell+1, 2\ell, 2\ell$ respectively.
Assume that one of the following assumptions holds:
\begin{lemlist}
 \item \label{item:uractionA} $\Phi=\rA_\ell$, $\Gamma=\{ k+1, \ldots, \ell+1\} \subset \Lambda(\pi)$ and $\sr(I)\leq k\leq \ell$;
 \item \label{item:uractionC} $\Phi=\rC_\ell$, $\Gamma=\{-\ell,\ldots, -2, -1\} \subset \Lambda(\pi)$ and $\sr(I)\leq \ell$;
 \item \label{item:uractionD} $\Phi=\rD_\ell$, $\Gamma=\{-\ell,\ldots, -2, -1\} \subset \Lambda(\pi)$ and $\asr(I)\leq \ell-1$. 
\end{lemlist}
Then for any $g\in \G(\Phi, R, I)$ there exist $x\in \U(\Phi^+, I)$, $y\in \U(\Phi^-, I)$ such that $(yxg \cdot v^+)_\lambda = 0$ for all $\lambda\in \Gamma$.
\end{lemma}
\begin{proof} Denote by $v$ the image of the highest weight vector $v^+$ under $g$.

\textsc{Case $\Phi=\rA_\ell$.} From the definition of the relative stable rank it follows that we can find 
$x= \left(\begin{smallmatrix} e_k & a \\ 0 & e_{n-k} \end{smallmatrix}\right) \in \U(\Sigma_k^+, I)$ such that 
the upper $k$ components of $v'= x \cdot v$ form an $I$-unimodular column. 
Now, to obtain zeroes at the desired positions it remains to subtract from $v'_{k+1},\ldots, v'_{\ell+1}$ suitable multiples of $v'_1, \ldots, v'_k$.
This operation corresponds to the left multiplication by some element $y\in\U(\Sigma_k^-, I)$.

\textsc{Case $\Phi=\rC_\ell$.} 
Applying \cref{item:asrUnipC}, we find $x \in \U(\Sigma_\ell^+, I)$ such that the upper half $v'_+$ of $v' = x \cdot v$ is unimodular.
Set $g(a) = \left(\begin{smallmatrix} e_\ell & 0 \\ p \cdot a & e_{\ell} \end{smallmatrix}\right)$.
Clearly, if $a$ is symmetric, then $g(a) \in \U(\Sigma_\ell^-, I)$.
Since the column $v''_+ = ({v'_1}^2, \ldots, {v'_\ell}^2)^t$ is $I$-unimodular, there exists a matrix $b \in M(\ell, I)$ such that $v'_- + p b v''_+ = 0$.
Finally, using \cref{lemma:PSV-symplectic-trick}, we find a symmetric matrix $a$ such that $(g(a) \cdot v')_- = p a v'_+ + v'_- = p b v''_+ + v'_- = 0$.

\textsc{Case $\Phi=\rD_\ell$.} From the proof of \cref{item:asrUnipD} it follows that there exists $h_1 \in \EP_\ell(R, I)$ such that for $v'=h_1\cdot v$ one has $v'_2=v'_1-1\in I$.
Clearly, for $v'' = z_{-\alpha_{2}}(-v'_2, 1)\cdot v'$ one has $v''_1=1$, hence by \cref{lemma:Chevalley-Matsumoto} there exists $h_2 \in \U(\Phi^-, I)$ such that the element
$g'=h_2 \cdot z_{-\alpha_{2}}(-v'_2, 1) \cdot h_1 \cdot g$ fixes $v^+$. 
Using the Levi decomposition we can write $g'=h \cdot y \cdot x \cdot g$ for some $y\in\U(\Sigma^-_\ell, I)$, $x \in \U(\Sigma^+_\ell, I)$, $h\in\E(\Delta_\ell, R, I)$.
It is clear that $x$, $y$ are the desired elements.
\end{proof}

\section{Relative parabolic factorizations} \label{sec:factorizations}
In this section we formulate and prove relative versions of the decompositions from~\cite{St78}, which will be our main technical tools throughout the next section.

Let $G$ be a group and $A$ its subset. Denote by $L(A)$, $R(A)$ and $N(A)$ the left and the right stabilizers of $A$ and the normalizer of $A$:
\begin{align*} 
 L(A) & =  \left\{ g\in G\ \middle |\ g \cdot A \cup g^{-1} \cdot A \subseteq A \right\}\!,\\
 R(A) & =  \left\{ g\in G\ \middle |\ A \cdot g \cup A \cdot g^{-1} \subseteq A \right\}\!,\\
 N(A) & =  \left\{ g\in G\ \middle |\ g \cdot A \cdot g^{-1} \cup g^{-1} \cdot A \cdot g \subseteq A \right\}.
\end{align*}
It is easy to see that $L(A)$, $R(A)$ and $N(A)$ are subgroups of $G$. The following obvious lemma will be used several times in the sequel.
\begin{lemma} \label{lem:LN} The subgroup $L(A)$ is normalized by $N(A)$. Moreover, one has $R(A) \cap N(A) \subseteq L(A)$, $L(A) \cap N(A) \subseteq R(A)$. \end{lemma}
\begin{comment}
\begin{proof}
Let $h \in L(A)$, $g \in N(A)$. The first claim follows from the following chain of inclusions:
\begin{equation} h^g \cdot A = g^{-1} \cdot h \cdot g \cdot A \cdot g^{-1} \cdot g \subseteq g^{-1} \cdot h \cdot A \cdot g \subseteq {A}^g \subseteq A. \end{equation} 
Now if, in addition, $g \in R(A)$ we obtain
$g \cdot A = g \cdot A \cdot g^{-1} \cdot g \subseteq A \cdot g \subseteq A,$ which implies the second claim.
\end{proof}
\end{comment}

\subsection{Relative Gauss decomposition}\label{sec:gauss}
The proof of the Gauss decomposition presented below is similar to the proof in the absolute case (cf.~\cite[Theorem~5.1]{Sm12}).

\begin{prop} \label{thm:Gauss}
Let $\Phi$ be a reduced irreducible root system of rank $\ell > 1$ and let $\Delta_1$, $\Delta_\ell$ be
the reductive subsystems of $\Phi$ corresponding to the endnodes of the Dynkin diagram of $\Phi$.
Suppose that both relative elementary subgroups $\E(\Delta_i, R, I)$, $i=1, \ell$ admit Gauss decomposition:
\[ \E(\Delta_i, R, I) = \Hh(\Delta_i, R, I) \cdot \U(\Delta^+_i, I) \cdot \U(\Delta^-_i, I) \cdot \U(\Delta^+_i, I), \quad i=1, \ell. \]
Then $\E(\Phi, R, I)$ also admits Gauss decomposition:
\begin{equation} \label{eq:gauss1} \E(\Phi, R, I) = \Hh(\Phi, R, I) \cdot \U(\Phi^+, I) \cdot \U(\Phi^-, I) \cdot \U(\Phi^+, I). \end{equation}
\end{prop}
\begin{proof}
For a closed subset of roots $S \subseteq \Phi^+$ set $A(S, I) = \U(S, I) \cdot \U(-S, I) \cdot \U(S, I)$ (here by $-S$ we denote the corresponding subset of opposite roots).
Notice that from Levi decomposition it follows that $A(\Phi^+, I) = A(\Delta_i^+, I) \cdot A(\Sigma_i^+, I)$.

Denote by $A$ the product of subgroups in the right-hand side of~\eqref{eq:gauss1}.
First of all, notice that for $h \in \Hh(\Phi, R, I)$, $\beta \in \Phi$ and $\xi \in R$ one has $x_\beta(\xi)^h = x_\beta(\xi + s\xi)$ for some $s\in I$.
From this and the assumption of the proposition we obtain for $\alpha \in \Delta_1 \cup \Delta_\ell$, $\xi \in R$ that
\begin{multline} \nonumber
 A^{x_\alpha(\xi)} = (\Hh(\Phi, R, I) \cdot A(\Phi^+, I))^{x_\alpha(\xi)} \subseteq (\Hh(\Phi, R, I) \cdot \E(\Delta_i, R, I) \cdot A(\Sigma_i^+, I))^{x_\alpha(\xi)} \subseteq \\
  \subseteq \Hh(\Phi, R, I) X_{\alpha}(I) \cdot \E(\Delta_i, R, I) \cdot A(\Sigma_i^+, I) \subseteq \Hh(\Phi, R, I) \cdot \Hh(\Delta_i, R, I) A(\Delta_i^+, I) \cdot A(\Sigma_i^+, I) \subseteq A.
\end{multline}
On the other hand, for $\beta \in \Phi^+$ we have
\begin{equation} \nonumber
 X_{\beta}(I) \cdot A \subseteq \Hh(\Phi, R, I) \cdot X_{\beta}(I) A(\Phi^+, I) \subseteq A.
\end{equation}
Thus, $A$ is normalized by root subgroups $X_\alpha(R)$, $\alpha \in \Delta_1 \cup \Delta_\ell$ and therefore is normalized by $\E(\Phi, R)$.
On the other hand, from $\U(\Phi^+, I) \cdot A \subseteq A$ and~\cref{lem:LN} we conclude that $\mathcal{X} \cdot A \subseteq A$ and hence
 that $\E(\Phi, R, I) \cdot A \subseteq A$.
\end{proof}

\begin{thm}\label{thm:srRI1}
Let $\Phi$ be a root system, let $I$ be an ideal of an arbitrary commutative ring $R$, and assume that $\sr(I)=1$.
Then the relative elementary Chevalley group $\E(\Phi, R, I)$ admits Gauss decomposition:
\[ \E(\Phi, R, I) = \Hh(\Phi, R, I) \cdot \U(\Phi^+, I) \cdot \U(\Phi^-, I) \cdot \U(\Phi^+, I). \]
\end{thm}
\begin{proof}
In view of \cref{thm:Gauss} it suffices to show that Gauss decomposition holds in the special case $\Phi=\rA_1$.
Let $A=\begin{psmallmatrix}a & b \\ c & d\end{psmallmatrix}$ be an element of $\SL(2, R, I)$.
The first column of $A$ is $I$-unimodular, therefore there exists $z\in I$ such that $a+cz\in R^*$.
Multiplying $A$ on the left by $x_{12}(z)$, we get a matrix $A'=x_{12}(z)\cdot A=\begin{psmallmatrix}a' & b' \\ c & d\end{psmallmatrix}$ with invertible element $a'$ in the top-left corner.
After multiplying $A'$ on the left by $x_{21}(-c/a')$ and on the right by $x_{12}(-b'/a')$ we get a diagonal matrix. 
Thus we have obtained a relative Gauss decomposition for $A$:
\begin{equation*}
A=x_{12}(-z)\cdot x_{21}(c/a')\cdot
\begin{pmatrix} \varepsilon & 0 \\ 0 & 1/\varepsilon \end{pmatrix}
\cdot x_{12}(b'/a')=x_{12}(-z)\cdot
\begin{pmatrix} \varepsilon & 0 \\ 0 & 1/\varepsilon \end{pmatrix}
\cdot x_{21}(y) \cdot x_{12}(b'/a'), 
\end{equation*}
where $\varepsilon\in 1+I$ and $y\in I$. \end{proof}

\subsection{Relative Dennis--Vaserstein decompositions}\label{sec:dennis-vaserstein}
Let $\Phi$ be an irreducible root system of rank $\ell\geq 2$.
Let $\{r, s\}$ be the indices of a pair of distinct endnodes of the Dynkin diagram of $\Phi$.

Before we proceed with the proof of~\cref{thm:DennisVaserstein} let us remark that in the special case $\ell=2$, $\sr(I)\leq 1$
 we already known that $\E(\Phi, R, I)$ admits Gauss decomposition.
It is not hard to see that Dennis--Vaserstein decomposition follows from Gauss decomposition, 
 therefore, in the sequel we may assume, without loss of generality, that $\ell > 2$.

From Levi decomposition~\eqref{rel:Levi-decomp} we obtain the following equality of subsets of $\E(\Phi, R, I)$:
\begin{multline*}
\U(\Phi^+, I)\cdot \U(\Phi^-, I) \cdot \E(\Delta_r, R, I) \cdot \EP_s(R, I) = \\
= \U(\Sigma_r^+, I)\cdot \U(\Sigma^-_r, I) \cdot \E(\Delta_r, R, I) \cdot \EP_s(R, I) = \hspace{5em} \\
\hspace{5em} = \EP_r(R, I) \cdot \E(\Delta_s, R, I) \cdot \U(\Sigma_s^-, I)\cdot \U(\Sigma_s^+, I) = \\
= \EP_r(R, I) \cdot \U(\Sigma^-_r \cap \Sigma^-_s, I) \cdot \EP_s(R, I).
\end{multline*}
Denote the above subset by $A_{rs}$. 
Notice that if the equality $A_{rs} = \E(\Phi, R, I)$ implies the ``symmetric'' equality $A_{sr} = \E(\Phi, R, I)$, 
 therefore it suffices to consider only the possibilities for $\Phi$, $s$, $r$ listed in~\cref{table:dv-reps} below.

\begin{table}[htb]
\[\begin{array}{l @{\qquad} l @{\qquad} c @{\quad} c @{\quad} c @{\qquad} c @{\qquad} c @{\qquad} c}
\Phi                                  & s    &r      & |\Lambda(\pi)| & \text{type of $\pi$} & \text{type of $\Delta_r$} & |\Lambda(\pi')|& |\Gamma|  \\ \hline\vphantom{\Bigl(}
\rA_\ell, \ \ell\geqslant 2           & 1    &\ell   & \ell+1         & \text{natural}       & \rA_{\ell-1}              & \ell           & 1  \\     
\rB_\ell, \ \ell\geqslant 2           & 1    &\ell   & 2\ell+1        & \text{natural}       & \rA_{\ell-1}              & \ell           & 1  \\     
\rC_\ell, \ \ell\geqslant 2           & 1    &\ell   & 2\ell          & \text{natural}       & \rA_{\ell-1}              & \ell           & 1  \\
\rD_\ell, \ \ell\geqslant 4           & 1    &\ell   & 2\ell          & \text{natural}       & \rA_{\ell-1}              & \ell           & 2  \\ 
\rD_\ell, \ \ell\geqslant 4           & \ell &\ell-1 & 2^{\ell-1}     & \text{half-spinor}   & \rA_{\ell-1}              & \ell           & \ell-2  \\
\rE_\ell, \ \ell=6, 7                 & \ell &2      & 27, 56         & \text{minimal}       & \rA_{\ell-1}              & \ell           & 3       \\ 
\rE_\ell, \ \ell=6, 7                 & \ell &1      & 27, 56         & \text{minimal}       & \rD_{\ell-1}              & 2(\ell-1)      & \ell-1 \\
\rF_4,                                & 4    & 1     & 26             & \text{minimal}       & \rC_{3}                   & 6              & 3 \end{array}\]
 \caption{List of representations used in the proof of \cref{thm:DennisVaserstein}.} \label{table:dv-reps}
\end{table}
The key step in the proof of~\cref{thm:DennisVaserstein} is to verify that the normalizer $N(A_{rs})$ is suffiently large.
This is accomplished in a series of lemmas. The following lemma identifies the ``obvious'' elements of $N(A_{rs})$.
\begin{lemma}\label{lemma:dv-normal} 
For every $\alpha \in \Delta_{r, s} \cup \Phi^+ $ one has $X_\alpha(R) \subseteq N(A_{rs})$. \end{lemma}
\begin{proof}
Notice that for every $i$ the group $\EP_i(R, I)$ is normalized by $\EP_i(R)$, hence it is normalized by $X_\alpha(R)$, $\alpha \in S_i^+$.
Since $\E(\Delta_{r, s}, R)$ normalizes $\U(\Sigma_r^- \cap \Sigma_s^-, I)$, we obtain the required assertion for $\alpha \in \Delta_{r, s}$.

Now let $\alpha$ be a positive simple root. It is clear that $\alpha$ lies either in $\Delta_r$ or $\Delta_s$.
Assume, for example, the latter. Using Levi decomposition we obtain that $\U(\Sigma_r^- \cap \Sigma_s^-, I)^{X_\alpha(R)} \subseteq \U(\Sigma_s^-, I) \subseteq A_{rs}$, therefore $X_{\alpha}(R) \subseteq N(A_{rs})$.
Thus, $N(A_{rs})$ contains the subgroup $\U(\Phi^+, R)$ generated by $X_{\alpha}(R)$, $\alpha\in\Pi$, which completes the proof.
\end{proof}

\begin{lemma}\label{lemma:dv_unipotent} For any $1\leq i\leq n$ the following statements hold. 
\begin{thmlist} \item \label{item-dvu1} $\U(\Phi^+, I) = X_{\alpha_{i}}(I)\cdot \U(\Phi^+\setminus\{\alpha_{i}\}, I) = \U(\Phi^+\setminus\{\alpha_{i}\}, I)\cdot X_{\alpha_{i}}(I)$;
\item \label{item-dvu2} For any $\xi\in R$ one has $\U(\Phi^+\setminus\{\alpha_i\}, I)^{x_{-\alpha_{i}}(\xi)} \subseteq \U(\Phi^+, I)$;
\item \label{item-dvu3} $\U(\Phi^+, I)\cdot \U(\Phi^-, I) \subseteq \U(\Phi^+\setminus \{\alpha_i\}, I) \cdot \U(\Phi^-, I) \cdot X_{\alpha_{i}}(I) \cdot X_{-\alpha_{i}}(I)$.
\end{thmlist} \end{lemma}
\begin{proof} The first two assertions follow from Chevalley commutator formula, while the third one is a consequence of the first two. \end{proof}

The following lemma is the key point of the proof where stability assumptions are invoked.
\begin{lemma}\label{lemma:Stein_reduction}
Under the assumptions of \cref{thm:DennisVaserstein} one has $X_{-\alpha_r}(R) \subseteq N(A_{rs}).$
\end{lemma}
\begin{proof}
Let $\pi$ be the representation of $\G(\Phi, R)$ wit the highest weight $\varpi_s$ (see Table~\ref{table:dv-reps}).
Denote by $\Lambda(\pi)$ the set of weights of $\pi$.
We will be mainly concerned with the action of root unipotents $x_\alpha(\xi)$ on weight spaces $V_\lambda$, $\lambda \in \Lambda(\pi)$.
Perhaps the most intuitive way to visualize this action is to use the technique of {\it weight diagrams}.
We refer the reader to~\cite{PSV98}, where this technique is introduced properly and where the weight diagrams for the representations appearing in~\cref{table:dv-reps} have been depicted.

Notice that $\Delta_r$ is an irreducible classical root system of type $\rA_\ell$, $\rC_\ell$ or $\rD_\ell$.
Denote by $(\pi', V')$ the irreducible component of the restriction of $\pi$ to $\G(\Delta_r, R)$ containing the highest weight vector $v^+$ of $\pi$.
In all cases under consideration, $\pi'$ is isomorphic to the natural representation of $\G(\Delta_r, R)$.

Set $\Gamma = \{\lambda \in \Lambda(\pi') \mid \lambda - \alpha_r \in \Lambda(\pi) \}.$
We can visualize $\Gamma$ using {\it weight diagrams}  in the following manner.
After removing all bonds marked $r$, the weight diagram of $\pi$ splits into several connected components.
The subset $\Lambda(\pi') \subseteq \Lambda(\pi)$ corresponds to the component of the diagram containing the vertex corresponding to the highest weight.
Clearly, $\Gamma$ consists of the weights of $\Lambda(\pi')$ incident to the removed bonds.
From the consideration of weight diagrams it is easy to determine the number of elements in $\Lambda(\pi')$ and $\Gamma$ (see~Table~\ref{table:dv-reps}).

Denote by $B$ the subset of $\E(\Delta_r, R, I)$ consisting of elements $g$ such that $(g \cdot v^+)_\lambda = 0$ for all $\lambda\in\Gamma$.
Set $A\coloneqq\U(\Phi^+, I)\cdot \U(\Phi^-, I) \cdot B \cdot \EP_s(R, I).$
Let $g$ be an element of $\E(\Delta_r, R, I)$. Applying \cref{lemma:uraction} to the subsystem $\Delta_r$, we find
$x\in\U(\Delta_r^+, I)$ and $y\in \U(\Delta_r^-, I)$ such that $yx\cdot g \in B$.
This shows that $A = A_{rs}$, indeed:
\begin{equation*} \U(\Sigma^+_r, I) \cdot \U(\Sigma^-_r, I) \cdot g = \U(\Sigma^+_r, I) x^{-1} \cdot \U(\Sigma^-_r, I)^{x^{-1}} y^{-1} (yxg) \subseteq \U(\Phi^+, I) \cdot \U(\Phi^-, I) \cdot B. \end{equation*}

Notice that by the definition of $\Gamma$ and Matsumoto lemma~\cite[Lemma~2.3]{Ma69} for any $s\in I$, $ g\in B$ one has $x_{-\alpha_r}(s) \cdot g \cdot v^+ = g \cdot v^+$, therefore
\[ X_{-\alpha_{r}}(I)^{B} \subseteq \U(\Phi^-, I) \cap \Stab(v^+) \subseteq \U(\Delta_s^-, I) \subseteq \EP_s(R, I). \]
From the above inclusion we immediately obtain that
\begin{equation*} X_{\alpha_r}(I) \cdot X_{-\alpha_r}(I) \cdot B \cdot \EP_s(R, I) \subseteq X_{\alpha_r}(I) \cdot B \cdot \EP_s(R, I) \subseteq B \cdot \U(\Sigma_r^+, I) \cdot \EP_s(R, I) = B \cdot \EP_s(R, I), \end{equation*}
which together with the third statement of \cref{lemma:dv_unipotent} implies that
\begin{equation*} \label{rel:sred}
A = \U(\Phi^+\setminus\{\alpha_r\}, I) \cdot \U(\Phi^-, I) \cdot B \cdot \EP_s(R, I).
\end{equation*}
Finally, since $[B, X_{-\alpha_r}(R)] \subseteq \U(\Sigma_r^-, R) \cap \E(\Phi, R, I) = \U(\Sigma_r^-, I)$, we obtain the required assertion, indeed:
\begin{equation*} \label{rel:ninv} A^{X_{-\alpha_{r}}(R)} = \U(\Phi^+, I) \cdot \U(\Phi^-, I) \cdot B ^{X_{-\alpha_{r}}(R)} \cdot \EP_s(R, I) = A. \qedhere \end{equation*}
\end{proof}

We will need one more root-theoretic lemma before we proceed with the proof of~\cref{thm:DennisVaserstein}.
Let $\Phi$, $r$ and $s$ be as in~\cref{table:dv-reps}.
Denote by $O$ the subset of roots of $\Sigma_s^-$ lying in the orbit $W(\Delta_s) \cdot (\Sigma^-_s\cap \Delta_r)$.
\begin{lemma} \label{lemma:root-lemma}
Suppose that $\Phi \neq \rC_\ell$. The following alternative holds: either $O$ coincides with $\Sigma_s^-$ or
$\Phi=\rB_\ell, \rF_4$ and $\Sigma_s^- \setminus O$ consists of the sole short root $\gamma$, which can be presented as a sum 
 $\gamma = \alpha + \beta$ for some short root $\alpha \in \Phi^+$ and some long root $\beta \in O$ satisfying $\beta + 2\alpha \in O$. 
\end{lemma}
\begin{proof}
From \cref{lemma:abs} it follows that $O$ does not coincide with $\Sigma^-_s$ only in the following two cases.
 \begin{enumerate} 
 \item \textit{Case $\Phi = \rB_\ell$, $s=1$, $r=\ell$.} 
 Only the sole short root of $\Sigma_1^-$ is not contained in $O$, denote it by $\gamma$. Set $\alpha = \alpha_2 + \ldots + \alpha_\ell,\ \beta = -\alpha_1 - 2\alpha_2 - \ldots -2\alpha_\ell.$
 It is clear that $\alpha \in \Phi^+$, $\beta, \beta + 2\alpha \in O$.
 \item \textit{Case $\Phi = \rF_4$, $s=4$, $r=1$.} There are three possibilities for the $s$-shape and length of $\alpha\in \Sigma_4^-$: 
 \begin{itemize}
  \item $\shape(\{\alpha_4\}, \alpha) = 1$, $\alpha$ is short (there are 8 such roots);
  \item $\shape(\{\alpha_4\}, \alpha) = 2$, $\alpha$ is long (there are 6 such roots);
  \item $\shape(\{\alpha_4\}, \alpha) = 2$, $\alpha$ is short (there is only one such root, denote it $\gamma$).
 \end{itemize}
 Since $\Delta_r \cap \Sigma_s^-$ contains roots of the first two types, only $\gamma$ is not contained in $O$.
 Clearly, $\gamma = -\alpha_1 -2\alpha_2-3\alpha_3-2\alpha_4 = \alpha + \beta$ for $\alpha = \alpha_2 + \alpha_3\in \Phi^+,$ $\beta = -\alpha_{{\max}} \in O$. \qedhere
\end{enumerate}
\end{proof}

\begin{proof} [Proof of \cref{thm:DennisVaserstein}]
From Lemmas~\ref{lemma:Stein_reduction} and~\ref{lemma:dv-normal} it follows that $N(A_{rs})$ contains $\EP_s(R)$.
Thus, by~\cref{lem:LN} we obtain that $\EP_s(R, I) \subseteq L(A_{rs})$.

Our next goal is to verify the inclusion $\U(\Sigma_s^-, I) \subseteq L(A_{rs})$.
Since $\widetilde{W}(\Delta_s) \subseteq \E(\Delta_s, R) \subseteq N(A_{rs})$ and 
 $X_{\alpha}(I) \subseteq L(A_{rs})$ for $\alpha \in \Sigma^-_s\cap \Delta_r$, 
we obtain from~\cref{lem:LN} and \cref{item-trans2} that $L(A_{rs})$ contains root subgroups $X_\alpha(I)$ for $\alpha \in O = W(\Delta_s) \cdot (\Sigma^-_s\cap \Delta_r)$.
Thus, by~\cref{lemma:root-lemma} we are left to consider two cases when $O$ does not coincide with $\Sigma^-_s$. 
\begin{enumerate} 
  \item \textit{Case $\Phi = \rB_\ell, \rF_4$.} Let $\alpha, \beta, \gamma$ be as in~\cref{lemma:root-lemma}.
  Since $X_{\beta}(I), X_{\beta+2\alpha}(I) \subseteq L(A_{rs})$ and $X_{\alpha}(R) \subseteq N(A_{rs})$ we obtain the required inclusion
   $X_{\gamma}(I)$ using Chevalley commutator formula:
   \begin{equation} \nonumber x_{\gamma}(\pm ab) = [x_{\beta}(a),\, x_{\alpha}(b)] \cdot x_{\beta + 2\alpha}(ab^2). \end{equation}
 \item \textit{Case $\Phi = \rC_\ell$, $s=1$, $r=\ell$.} 
  In this case we need to invoke stability assumptions one more time.
  From~\cref{lemma:DVST} below we obtain that $N(A_{rs})$ contains $X_{-\alpha_1}(R)$ and therefore coincides with $\E(\Phi, R)$.
  The required inclusion now follows~\cref{item-trans1}.
\end{enumerate}
Thus, we have shown that $\mathcal{X}\subseteq L(A_{rs})$. 
Recall from~\cite[Theorem~3.4]{S} that the group $\E(\Phi, R, I)$ is generated by the subset $\mathcal{Z}(\Sigma_s^-)$.
On the other hand, from $\mathcal{Z}(\Sigma_s^-) \subseteq \mathcal{X}^{\EP_s(R)} \subseteq L(A_{rs})^{N(A_{rs})} \subseteq L(A_{rs})$ we conclude that 
 $\E(\Phi, R, I) \subseteq L(A_{rs})$, which completes the proof.
\end{proof}

\begin{lemma}\label{lemma:DVST}
Assume that $\Phi=\rC_\ell$ and $\sr(I) \leq \ell-1$, then $X_{-\alpha_s}(R) \subseteq N_{rs}$.
\end{lemma}
\begin{proof}
Denote by $C$ the set consisting of elements $g \in \EP_s(R, I)$ for which matrix entries $(g_{i, 2})$, $i=2, \ldots, \ell$ form an $I$-unimodular column of height $\ell-1$

Set $A' = \EP_\ell(R, I) \cdot \U(\Sigma_s^- \cap \Sigma_r^-, I) \cdot C$.
Applying \cref{item:asrUnipC} we find for every $g \in \EP_s(R, I)$ an element $x \in \U(\Sigma_r^+ \cap \Delta_s, I)$ such that $xg \in C$.  
Notice that the equality $A_{rs} = A'$ follows from this, indeed, for $g\in \EP_s(R, I)$ one has
\begin{equation*} \EP_r(R, I) \cdot \U(\Sigma_s^- \cap \Sigma_r^-, I) \cdot g \subseteq 
 \EP_r(R, I)x^{-1}  \cdot \U(\Sigma_s^-, I) \cdot xg \subseteq A'. \end{equation*}

By the definition of $C$, for every $g \in C$ one can choose $y \in \U(\Sigma_s^+ \cap \Delta_r, I)$ such that $(y \cdot g)_{\varpi_s, \varpi_s - \alpha_s} = 0$.
Consequently, for every $g\in C$ one has
\[
 \EP_r(R, I) \cdot \U(\Sigma_s^- \cap \Sigma_r^-, I) \cdot g \subseteq \EP_r(R, I) y^{-1} \cdot \U(\Sigma_r^-\cap \Sigma_s^-, I)^{y^{-1}} \cdot y g
  \subseteq \EP_r(R, I) \cdot \U(\Sigma_r^-, I) \cdot y g.
\]
Notice that the matrix entry $(yg)_{\varpi_s, \varpi_s}$ is invertible.
From the choice of $y$ it follows that for every $\xi\in R$ the element $g_1 \coloneqq (yg)^{x_{-\alpha_s}(\xi)}$
satisfies the assumptions of \cref{lemma:Chevalley-Matsumoto} and therefore can be rewritten as $g_1 = uh$ for some $u \in \U(\Sigma_s^-, I)$, $h \in \EP_s(R, I)$.
Consequently, one has
\begin{multline*} {A_{rs}}^{X_{-\alpha_{s}}(R)} \subseteq \EP_r(R, I)^{X_{-\alpha_{s}}(R)} \cdot \U(\Sigma_r^-, I)^{X_{-\alpha_{s}}(R)} \cdot \U(\Sigma_s^-, I) \cdot \EP_s(R, I) \subseteq \\
\subseteq \EP_r(R, I) \cdot \U(\Phi^-, I) \cdot \EP_s(R, I) \subseteq A_{rs}, \end{multline*}
as claimed.
\end{proof}

\subsection{Relative Bass--Kolster decompositions}\label{sec:bass-kolster}
The next theorem is a relative version of the so-called Bass--Kolster decomposition (cf.~\cite[Theorem~2.1]{St78}).
\begin{thm}\label{thm:BassKolster}
Let $\Phi$ be a classical root system of rank $\ell\geqslant2$, let $R$ be an arbitrary commutative ring and $I$ be its ideal, satisfying one of the following assumptions:
\[\begin{array}{l@{\quad}l@{\quad}l@{\quad}c}
\Phi = \rA_\ell, \ \ell\geqslant 2, & \sr(I) \leqslant \ell; \\
\Phi = \rC_\ell, \ \ell\geqslant 2, & \sr(I) \leqslant 2\ell-1; \\
\Phi = \rB_\ell, \rD_\ell, \ \ell\geqslant 3, & \asr(I) \leqslant \ell-1.
\end{array}\]
Then the principal congruence subgroup $\G(\Phi, R, I)$ admits decomposition:
\[ \G(\Phi, R, I)=  \U(\Phi^+, I) \cdot \U(\Phi^-, I) \cdot Z \cdot \U(\Sigma_1^-\setminus\{-\alpha_{\max}\}, I) \cdot \U(\Sigma_1, I) \cdot \G(\Delta_1, R, I), \]
where $Z = Z_{\alpha_{\max}}(I)=\left\{z_{-\alpha_{\max}}(r, 1)\ \middle|\ r\in I \right\}$.
\end{thm}
\begin{proof}

Let $g$ be an element of $\G(\Phi, R, I)$. Set $v=g \cdot v^+\in\Ums(n, I)$. 
Notice that in each case it suffices to find $g' \in \U(\Phi^-, I) \cdot \U(\Phi^+, I) \cdot g$ such that 
\begin{equation} \label{eq1} (g'\cdot v^+)_{1} = 1 + s \text{ and } (g'\cdot v^+)_{\varpi_{1}-\alpha_{max}} = s\ \text{for some}\ s\in I. \end{equation}
Indeed, set $g'' = z_{-\alpha_{max}}(-s, 1) \cdot g'$.
Obviously, one has $(g''\cdot v^+)_1 = 1$, $(g''\cdot v^+)_{\varpi_{1}-\alpha_{max}}=0$ and the conclusion of the theorem follows from \cref{lemma:Chevalley-Matsumoto}.

\textsc{Case $\Phi=\rA_\ell$, $n=\ell + 1$.}
Since $\sr(I)\leqslant\ell$, one can find $a_1, \ldots, a_\ell\in I$ such that $(v_1+a_1v_{\ell+1}, \ldots, v_\ell+a_\ell v_{\ell+1})^t=(v'_1, \ldots, v'_\ell)^t$ is $I$-unimodular.
Then there are $b_1, \ldots, b_\ell\in I$ such that $b_1v'_1+\ldots b_\ell v'_\ell=v'-1\in I$. 
Thus the vector \[ v'' = \prod_{i=1}^\ell x_{\ell+1, i}(b_i) \cdot \prod_{i=1}^\ell x_{i, \ell+1}(a_i) \cdot v \]
satisfies the equalities~\eqref{eq1}.

\textsc{Case $\Phi=\rC_\ell$, $n=2\ell$.}
Notice that the column $(v_1, \ldots, v_{-2}, v_{-1}^2)^t$ is also $I$-unimodular.
Applying the assumption $\sr(I)\leqslant 2\ell-1$, we find $c_1, c_2, \ldots, c_{-2} \in I \cdot v_{-1}$ such that upper $2\ell -1$ components of $v'=(v_1 + c_1 v_{-1}, \ldots, v_{-2} + c_{-2}v_{-1}, v_{-1})^t$ form an $I$-unimodular column.
By the choice of $c_i$ we can find suitable $d\in I$ such that $h_1 \cdot v = v'$ for
\[ h_1 = x_{1, -1}(c_1 + d) \cdot \prod_{i=2}^{-2} x_{i, -1}(c_i) \in \U(\Sigma_1^+, I). \]
We can find $f_1, f_2, \ldots, f_{-2} \in R$ such that $f_1v'_1+\sum_{i=2}^{-2} f_i v'_i = 1$.
Set $\xi = v''_1-v''_{-1}-1 \in I$, 
\[ h_2 = x_{-1, 1}\biggl(\xi f_1 + \sum_{i=2}^\ell v_1' \xi^2 f_i f_{-i}\biggr) \cdot \prod_{i=2}^{-2} x_{-1, i}(\xi f_i) \in \U(\Sigma_1^-, I). \]
Direct computation shows that the vector $v'' = h_2 \cdot v'$ satisfies equalities~\eqref{eq1}.

\textsc{Case $\Phi=\rD_\ell$, $n= 2\ell$.} 
By \cref{item:asrUnipD} we can find $h_1\in \U(\Sigma^+_\ell, I)$ such that the upper half $v'_+$ of $v'=h_1 \cdot v$ is $I$-unimodular.
Since $\sr(I)\leqslant \ell-1$, we can find $c_1$, $c_3, \ldots c_\ell \in I$ such that $(v''_1, v''_3, \ldots, v''_\ell) \in \Ums(\ell-1, I)$, where
\[ v''=h_2 \cdot x_{1, 2}(c_1) \cdot v', \quad h_2=\prod_{i=3}^\ell x_{i, 2}(c_i). \]
We can find $f_1, f_3, \ldots, f_\ell \in R$ such that $f_1v''_1+\sum_{i=3}^\ell f_i v''_{i} = 1$.
As before, set
\[ \xi = v''_1-v''_{-2}-1 \in I, \quad h_3 = x_{-2, 1}(\xi f_1) \cdot \prod_{i=3}^\ell x_{-2, i}(\xi f_i), \quad v'''=h_3 \cdot v''. \]
Clearly, $t_{1, 2}(c_1) \cdot h_1 \in \U(\Phi^+, I)$, $ h_3 \cdot h_2 \in \U(\Phi^-, I)$ and $v'''$ satisfies~\eqref{eq1}.

\textsc{Case $\Phi=\rB_\ell$, $n=2\ell+1$.} Subdivide $v\in \Ums(2\ell+1, I)$ as $v=(v_+, v_0, v_-)\in R^\ell\times R\times R^\ell$.
Denote by $J\leqslant I$ the ideal spanned by the components of $v_-$.
Since $\sr(I/J)\leqslant \ell$, we can find $c_1, \dots, c_\ell\in I$ such that for $v' = h \cdot v$, $h = \prod_{i=1}^\ell x_{i, 0}(c_i) \in \U(\Phi^+, I)$
one has $\bar{v'}_+=(\bar{v'_1}, \ldots, \bar{v'_\ell}) \in \Ums(\ell, I/J)$ and, therefore, $(v'_+, v'_-) \in \Ums(2\ell, I)$.
Now the proof can be finished by repeating the argument for the case $\Phi=\rD_\ell$ (applied to the subset of long roots of $\rB_\ell$).
\end{proof}

It is easy to see that the proof of the above theorem is effective and gives an estimate of the total number of elementary root unipotents involved in the decomposition.
\begin{cor}\label{cor:bass-kolster-count}
In the assumptions and notation of \cref{thm:BassKolster} every element of $\G(\Phi, R, I)$ 
can be factored into a product of one element of $\G(\Delta_1, R, I)$, one element of $Z$ and at most $4|\Sigma_1|-1$ elements of $\mathcal{X}$.
\end{cor}
\begin{proof}
The assertion can be obtained by a careful analysis of the proof of the previous theorem.
Cases $\Phi=\rA_\ell, \rC_\ell$ are immediate.
In the case $\Phi=\rD_\ell$ from the proof of \cref{thm:BassKolster} one obtains that
\begin{equation*} \G(\Phi, R, I) =  \U(\Sigma_\ell^+, I) \cdot X_{\alpha_1}(I) \cdot \U(\Sigma_2^-\cap\Delta_1, I) \cdot X_{-\alpha_{\max}}(I) \cdot Z \cdot \U(\Sigma_1^-, I) \cdot \U(\Sigma_1^+, I) \cdot \G(\Delta_1, R, I). \end{equation*}
We can present an element $g$ of $\U(\Sigma_\ell^+, I)$ as a product of $g_1 \in \U(\Sigma_{1, 2}^+ \cap \Sigma_\ell^+)$ and $g_2\in \U(\Delta_{1, 2}\cap \Sigma_\ell^+)$.
An examination of the extended Dynkin diagram of $\rD_\ell$ shows that $g_2$ either centralizes or normalizes all factors of the above decomposition (except the last one)
and therefore can be moved to the right until it is consumed by $\G(\Delta_1, R, I)$.
On the other hand, $g_1$ is a product of at most $2\ell-3$ elementary unipotents, while the width of $\U(\Sigma_1^\pm, I)$ and $\U(\Sigma_2^-\cap\Delta_1)$ with respect to the elementary unipotents does not exceed $2\ell-2$ and $2\ell-4$, respectively.
Summing up these upper bounds, we obtain
$$(2\ell-3) + 1 + (2\ell - 4) + 1 + 2\cdot (2\ell - 2) = 8\ell - 9 = 4|\Sigma_1| - 1.$$

The estimate in the case $\Phi=\rB_\ell$ can be obtained in a similar way. \end{proof}
\begin{cor}\label{cor:bass-kolster-iterated}
Assume that $\Phi$ and $I$ satisfy one of the following assumptions
\[\begin{array}{l@{\quad}l@{\quad}l}
\Phi=\rA_\ell, & \sr(I)\leqslant 2, & N'=3\left|\Phi^+\right|+2\ell - 5; \\
\Phi=\rC_\ell, & \sr(I)\leqslant 3, & N'=3\left|\Phi^+\right|+3\ell - 6; \\
\Phi=\rB_\ell, \rD_\ell, & \asr(I)\leqslant 2, & N'=4\left|\Phi^+\right| - 4.
\end{array}\]

Then every element of $\G(\Phi, R, I)$ can be decomposed into a product of one element of $\G(\langle\pm\alpha_\ell\rangle, R, I) \cong \SL(2, R, I)$ and at most $N'$ elements of $\mathcal{Z}(\Sigma_\ell)$:
\end{cor}
\begin{proof}
The assertion can be obtained by iteratively applying (for a total of $\ell-1$ times) the decomposition of \cref{thm:BassKolster}.
The improved estimate for $\Phi=\rA_\ell$ (resp. $\rC_\ell$) follows from the fact that it suffices to make only two (resp. three) additions to shorten the unimodular column in the first step of the proof of \cref{thm:BassKolster}.
\end{proof}

\begin{figure}[hb]\label{fig:bass-kolster}
\[\begin{tikzcd}[row sep=tiny]
\rD_4 \arrow[out=0, in=120]{rd}{\asr\leqslant3} & & & \\
\rA_4 \arrow{r}{\sr\leqslant4} & \rA_3 \arrow{r}{\sr\leqslant3} & \rA_2 \arrow{r}{\sr\leqslant2} & \rA_1 \\
\rC_4 \arrow{r}{\sr\leqslant7} & \rC_3 \arrow{r}{\sr\leqslant5} & \rC_2 \arrow[out=0, in=-120, swap]{ru}{\sr\leqslant3} \\
\rB_4 \arrow{r}{\asr\leqslant3} & \rB_3 \arrow[out=0, in=-120, swap]{ru}{\asr\leqslant2} & &
\end{tikzcd}\] \caption{Reductions used in the proof of \cref{cor:bass-kolster-iterated} and \cref{thm:SL2width}} \end{figure}

\section{Applications} \label{sec:applications}
\subsection{Subsystem factorizations}\label{sec:subsysfact}
Recall from~\cite{LNS11, V13} that a Chevalley group over a field $\G(\Phi, F)$ can be presented as a product of at most $3|\Phi^+|$ of its subgroups $\SL(2, F)$.
As an easy corollary of the relative Bass--Kolster decomposition we get that a similar factorization exists for relative groups over a rather general class of commutative rings.
\begin{thm}\label{thm:SL2width}
Let $I\trianglelefteq R$ be an ideal, $\Phi$ be an irreducible classical root system of rank $\ell$ satisfying one the following 4 conditions:
\[\begin{array}{l@{\quad}l@{\quad}l}
\Phi=\rA_\ell, & \sr(I)\leqslant 2, & N=3\left|\Phi^+\right| - \ell - 1; \\
\Phi=\rC_\ell, & \sr(I)\leqslant 3, & N=3\left|\Phi^+\right| - 2; \\
\Phi=\rB_\ell, \rD_\ell, & \asr(I)\leqslant 2, & N=4\left|\Phi^+\right| - 3\ell.
\end{array}\]
Then the principal congruence subgroup $\G(\Phi, R, I)$ can be presented as a product of at most $N$ of its subgroups, 
 where each subgroup is an isomorphic copy of $\SL(2, R, I)$.
\end{thm}

\begin{proof}
 As in the proof of \cref{cor:bass-kolster-iterated}, one has to iteratively apply \cref{thm:BassKolster}.
 To reduce the number of $\SL_2$-factors involved in the decompositoin one has to group into a single $\SL_2$-factor a pair of opposite root subgroups $X_{\alpha}(I)$, $X_{-\alpha}(I)$ (or $Z_{\pm\alpha}(I)$) appearing on each
 of the $3$ junctions between the positive and negative unipotent subgroups in the Bass--Kolster decomposition.
 Since a total of $\ell-1$ reductions are used, we get the estimate $N \leq N' - 3(\ell - 1) + 1$ and the assertion of \cref{thm:SL2width} follows.
\end{proof}

We now turn our attention to the proof of \cref{thm:spin-sln-prod}, which will occupy the rest of this subsection.

Consider the decreasing chain $\Phi_k$, $k=1, \ldots, \lfloor \ell/2 \rfloor$ of root subsystems of $\Phi=\rD_\ell$ defined as follows.
If $2k \neq \ell$, let $\Phi_k$ be the subsystem of $\Phi$, spanned by the fundamental roots $\alpha_{2k-1}, \ldots, \alpha_\ell$.
Clearly, such $\Phi_k$ has type $\rD_{\ell-2k+2}$. 
In the remaining case $2k = \ell$ set $\Phi_k = \langle \alpha_\ell \rangle \cong \rA_1$.
Now let $\beta_k$ be the maximal root of $\Phi_k$, i.\,e. $\beta_k = \alpha_{\max}(\Phi_k)$, $k=1, \ldots, \lfloor \ell/2 \rfloor$.
Denote by $B$ the set of all $\beta_k$. From the definition it is clear that the elements of $B$ are mutually orthogonal to each other.
The roots $\beta_k$ can also be defined by explicit formulae:
\begin{align*}
 \beta_k =  \alpha_{2k-1} + 2\alpha_{2k}+ \ldots + 2\alpha_{\ell-2} + \alpha_{\ell-1} + \alpha_\ell, & \text{ for } k=1, \ldots, \lfloor\ell/2\rfloor-1, \\
 \beta_{\lfloor\ell/2\rfloor} = \alpha_{\ell-2}+\alpha_{\ell-1}+\alpha_\ell, & \text{ if $\ell$ is odd, } \\
 \beta_{\lfloor\ell/2\rfloor} = \alpha_\ell, & \text{ if $\ell$ is even.}
\end{align*}

\begin{lemma}\label{lemma:nikolov-weyl} There exists an element $w\in W(\rD_\ell)$ such that $w(B) \subseteq \Delta_\ell^+$. \end{lemma}
\begin{proof}
\textsc{Case $\ell=4$.} Set $w = \sigma_{\alpha_{1} + \alpha_{2}} \circ \sigma_{\alpha_{2} + \alpha_{4}}$.
Straightforward computation shows that 
\begin{align*}
& w(\beta_1) = w(\alpha_{\max}) = \sigma_{\alpha_{1} + \alpha_{2}}(\alpha_1 + \alpha_2 + \alpha_3) = \alpha_3, \\
& w(\beta_2) = w(\alpha_4) = \sigma_{\alpha_{1} + \alpha_{2}}(- \alpha_2) = \alpha_1,
\end{align*}
which implies the assertion of the lemma.

\textsc{Case $\ell \geq 5$.}
Recall from \cite[Table~9]{Dy72} that for odd (resp. even) $\ell$ all maximal subsystems of type $\rA_1+\ldots+\rA_1+\rD_3$
(resp. $\rA_1+\ldots+\rA_1+\rD_4$) are conjugate under the action of $W(\Phi)$. Consequently, we can find $w\in W(\Phi)$
such that $w(\beta_k) = \alpha_{2k-1}$ for $k < \lfloor\ell/2\rfloor$ (resp. $k < \lfloor\ell/2\rfloor-1$). 
Now using transitivity of the action of $W(\rD_3)$ on the roots (resp. by the same argument as in the case $\ell=4$) we can move the remaining root $\beta_{\lfloor\ell/2\rfloor}$ 
(resp. the remaining $2$ roots $\beta_{\lfloor\ell/2\rfloor-1}$, $\beta_{\lfloor\ell/2\rfloor}$) to
$\alpha_{\ell-1}$ (resp. to $\alpha_{\ell-3}$, $\alpha_{\ell-1}$) while fixing all the other $\beta_k$. \end{proof}

The following lemma is an analogue of Proposition~1 of~\cite{Nik07}.
\begin{lemma}\label{lemma:nikolov-type-dl}
Let $\Phi=\rD_\ell$, $\ell\geq 2$ and let $I$ be an ideal of a  commutative ring $R$.
There exist an element $y\in\E(\Phi, R)$ and an element $w\in\widetilde{W}(\Phi)$ such that
\[ \U(\Sigma_\ell^+, I)\subset[\U(\Delta_\ell^-, I), y]\cdot{}^w\!\U(\Delta_\ell^+, I). \]
\end{lemma}
\begin{proof}
Since $\U(\Sigma_\ell^+, I)$ is abelian, we can decompose it as $\U(\Sigma_\ell^+, I)=\U(\Sigma_\ell^+\setminus B, I) \cdot \U(B, I)$. 
Set $y=\prod_{\beta\in B}x_\beta(1)$. 
We will now show by induction on $\ell$ that 
\begin{equation}\label{eq:ind-stat} \U(\Sigma_\ell^+\setminus B, I)\subset[\U(\Delta_\ell^-, I), y]\cdot\U(B, I). \end{equation}
The induction base in the cases $\ell=2, 3$ is trivial.

Notice that $\beta_1$ is the only root of $\Phi$ satisfying $m_2(\beta_1)=2$, therefore Chevalley commutator formula implies
\[ \bigl[\U(\Delta_2^-, I), x_{\beta_1}(1)\bigr]=1. \]
There is no root of the form $\gamma=\alpha+\beta$ with $\alpha\in\Sigma_2^-\cap\Delta_\ell$ and $\beta\in B\setminus\{\beta_1\}$, because such a root $\gamma$ must satisfy $m_2(\gamma)=-1$ and $m_\ell(\gamma)=1$. Thus the commutator formula gives
\[\Bigl[\U(\Sigma_2^-\cap\Delta_\ell, I), \prod_{i\neq1}x_{\beta_i}(1)\Bigr]=1. \]
Since $B\setminus\{\beta_1\}\subset\Sigma_\ell^+\cap\Delta_2$, the above two identities imply
\[
\Bigl[ \U(\Sigma_2^-\cap\Delta_\ell, I)\cdot\U(\Delta_{2, \ell}^-, I), x_{\beta_1}(1)\cdot\prod_{i\neq1}x_{\beta_i}(1) \Bigr]
\equiv \bigl[ \U(\Sigma_2^-\cap\Delta_\ell, I), x_{\beta_1}(1) \bigr] \bmod \U(\Sigma_\ell^+\cap\Delta_2, I).
\]
Take an element $u\in\U(\Sigma_2^-\cap\Delta_\ell, I)$ and decompose it as $u=vw$, where $v\in\U(\Sigma_1^-\cap\Sigma_2^-\cap\Delta_\ell, I)$ and $w\in\U(\Sigma_2^-\cap\Delta_{1, \ell}, I)$.
Using the identity
\begin{equation}\label{eq:comm-ab-c}
[ab, c]={}^a[b, c]\cdot[a, c], 
\end{equation}
we can rewrite
\[ [vw, x_{\beta_1}(1)] = {}^v[w, x_{\beta_1}(1)]\cdot[v, x_{\beta_1}(1)].  \]
Since $\U(\Sigma_1^-\cap\Sigma_2^-\cap\Delta_\ell, I)$ and $\U(\Sigma_2^-\cap\Delta_{1, \ell}, I)$ are abelian, it is easy to see that
\[ [v, x_{\beta_1}(1)]\in\U(\Sigma_2^+\cap\Sigma_\ell^+\cap\Delta_1, I), \quad [w, x_{\beta_1}(1)]\in\U((\Sigma_1^+\cap\Sigma_\ell^+)\setminus\{\beta_1\}, I). \]
Every element of $\U(\Sigma_2^+\cap\Sigma_\ell^+\cap\Delta_1, I)$ (resp. $\U(\Sigma_1^+\cap\Sigma_\ell^+\setminus\{\beta_1\}, I)$) can be expressed as such a commutator for a suitable choice of $v$ (resp. $w$).
Indeed, set $v=x_\gamma(\xi_\gamma)\cdot v'$, $\gamma=-\alpha_1-\alpha_2$, $v' \in \U(\Sigma_1^- \cap \Sigma_2^- \cap \Delta_\ell \setminus \{\gamma\}, I)$.
Using relation \eqref{eq:comm-ab-c} and the fact that $X_\gamma(I)$ commutes with $\U(\Sigma_2^+\cap\Delta_1, I)$ we get that:
\begin{multline*}
[v, x_{\beta_1}(1)] = [x_\gamma(\xi_\gamma)\cdot v', x_{\beta_1}(1)] = {}^{x_\gamma(\xi_\alpha)}[v', x_{\beta_1}(1)] \cdot [x_\gamma(\xi_\gamma), x_{\beta_1}(1)] = \\
= [v', x_{\beta_1}(1)]\cdot x_{\beta_1-\alpha_1-\alpha_2}(\xi_\gamma) = \ldots = \prod_{\mathclap{\alpha\in\Sigma_1^- \cap \Sigma_2^- \cap \Delta_\ell}} x_{\beta_1+\alpha}(\xi_\alpha).
 \end{multline*}
It remains to note that $\Sigma_2^+ \cap \Sigma_\ell^+ \cap \Delta_1 = \beta_1 + (\Sigma_1^- \cap \Sigma_2^- \cap \Delta_\ell)$. The same argument works for $[w, x_{\beta_1}(1)]$.
Direct calculation using the commutator formula shows that
\[ {}^v\!\U(\Sigma_1^+\cap\Sigma_\ell^+\setminus\{\beta_1\}, I) \equiv \U(\Sigma_1^+\cap\Sigma_\ell^+\setminus\{\beta_1\}, I) \bmod \U(\Sigma_\ell^+\cap\Delta_2, I). \]
Summing up the above arguments, we get that
\[ [\U(\Sigma_2^-, I)\cdot\U(\Delta_{2, \ell}^-, I), y] \equiv \U((\Sigma_{1, 2}^+\cap\Sigma_\ell^+)\setminus\{\beta_1\}) \bmod \U(\Sigma_\ell^+\cap\Delta_2, I), \]
hence the inclusion~\eqref{eq:ind-stat} follows from the induction hypothesis (applied to $\Delta_{1, 2} \cong \rD_{\ell-2}$). 

Finally, we have found $a\in\U(\Sigma_\ell^+\setminus B, I)$ and $b\in\U(\Delta_\ell^-, I)$
such that $$a\in[b, y]\cdot\prod_{\beta\in B}X_\beta\subset[\U(\Delta_\ell^-, I), y]\cdot\U(B, I).$$
Now the assertion of the lemma follows from~\cref{lemma:nikolov-weyl}.
\end{proof}

\begin{proof} [Proof of \cref{thm:spin-sln-prod}]
Set $L = \E(\Delta_\ell, R, I) \leq \E(\rD_\ell, R, I)$ and denote by $\sigma$ the automorphism of $\G(\rD_\ell, R)$ induced by the 
diagram automorphism of $\rD_\ell$ swapping $\alpha_\ell$ and $\alpha_{\ell-1}$.
By \cref{thm:DennisVaserstein} one has
\begin{align*}
\E(\rD_\ell, R, I) & = \EP_{\ell}(R, I)\cdot\U(\Sigma_{\ell-1}^- \cap \Sigma_{\ell}^-, I)\cdot\EP_{\ell-1}(R, I) = \\
& = L \cdot\U(\Sigma_\ell^+, I)\cdot\U(\Sigma_{\ell-1}^- \cap \Sigma_{\ell}^-, I)\cdot \big(L \cdot \U(\Sigma_\ell^+, I) \big)^\sigma\!.
\end{align*}  
Now using \cref{lemma:nikolov-type-dl}, one can find $y_1, y_2\in\G(\rD_\ell, R)$ and $w_1, w_2\in\widetilde{W}(\rD_\ell)$ such that
\begin{align*} & L \cdot \U(\Sigma_\ell^+, I) \subset L \cdot \U(\Delta_{\ell-1}^-, I) \cdot {}^{y_1}\!\U(\Delta_{\ell-1}^-, I) \cdot {}^{w_1}\!\U(\Delta_{\ell-1}^+, I), \\
& \U(\Sigma_{\ell-1}^-\cap\Sigma_\ell^-, I) \subset \U(\Delta_\ell^+, I) \cdot {}^{y_2}\!\U(\Delta_\ell^+, I) \cdot {}^{w_2}\!\U(\Delta_\ell^-, I). \end{align*} 
Thus $\E(\rD_\ell, R, I)$ is a product of at most $9$ subgroups isomorphic to $\E(\rA_{\ell-1}, R, I)$. \end{proof}

\subsection{Bounded generation}\label{sec:boundgen}
For a group $G$ denote by $W(G, X)$ \emph{the width of $G$ with respect to a generating set $X \subseteq G$}, i.\,e. the smallest natural number $N$ 
such that every element of $G$ is a product of at most $N$ elements of $X$ or their inverses.

\begin{lemma}\label{lemma:srRI1_width}
In the assumptions of \cref{thm:srRI1} the width of $\E(\Phi,R,I)$ with respect to $\mathcal{Z}(\Pi)$ does not exceed $3\left|\Phi^+\right|+2\rk\Phi-1$.
\end{lemma}
\begin{proof}
Take an element $g\in\E(\Phi,R,I)$ and decompose it into $g=u_1 h v_2 u_3$, where $h\in\Hh(\Phi,R,I)$, $u_1,u_3\in\U(\Phi,I)$, $v_2\in\U(\Phi^-,I)$. 
Write $h=\prod_{i=1}^\ell h_{\alpha_i}(\varepsilon_i)$, $\varepsilon\in1+I$. 
Each $h_{\alpha_i}(\varepsilon_i)$ decomposes into $h_{\alpha_i}(\varepsilon_i) = x_{\alpha_i}(*) z_{-\alpha_i}(*,*) x_{-\alpha_i}(*)$ 
(see \eqref{eq:rel-tor-elementary}), and since the torus normalizes each of $X_\alpha(I)$ (see formula~\eqref{rel:h-w}), we can decompose $g$ as follows:
\[ g\in\U(\Phi,I)\cdot\prod_{i=1}^\ell\bigl(x_{\alpha_i}(*)z_{-\alpha_i}(*,*)\bigr)\cdot \U^-(\Phi,I) \U(\Phi,I), \]
and the estimate follows.
\end{proof}

The following lemma is a corollary of Theorems~5.7 and 5.8 of~\cite{LSM}.
\begin{lemma}
Let $p$ be a rational prime, let $c$, $d$ be a pair of coprime integers such that $p \perp d$.
Then under the assumption of the Generalized Riemann Hypothesis there exist infinitely many primes $q\equiv c\pmod{d}$ such that $p$ is a primitive root modulo $q$.
\end{lemma}

The following lemma is a relative version of \cite[Lemma~6]{VavSmSuUnitrEng} (see also~\cite{VseUnitrZ1p}):

\begin{lemma}\label{lemma:Z1p}
Set $R=\mathbb{Z}[\sfrac{1}{p}]$ and let $I$ be an ideal of $R$.
Under the assumption of the GRH the width of $\SL(2, R, I)$ with respect to the generating set
\[ \mathcal{Z}(\{-\alpha_1\})=X_{12}(I)\cup X_{21}(I) \cup \{z_{21}(s, \xi) \mid s\in I,\ \xi\in R\} \]
does not exceed $6$.
\end{lemma}

\begin{proof}
Clearly, $I$ is a principal ideal generated by some integer $m\in\mathbb{Z}$ not divisible by $p$.
Let $g$ be an element of $\SL(2,R,I)$. Write
\[ g=\begin{pmatrix}x & y \\ z & w\end{pmatrix},\ \text{for}\ x=p^\alpha a,\ z=p^\beta bm,\ \text{where}\ a,b,\alpha,\beta\in\mathbb{Z},\ p\nmid a,b. \]

\textsc{Case 1:} $\alpha\geqslant\beta$. 
Since $p^{\alpha-\beta}a\perp bm^2$ and $p\perp bm^2$, there exist infinitely many rational primes $q$ of the form $p^{\alpha-\beta}a+bm^2k$,
such that $p$ is a primitive root modulo $q$. 
We may assume that $q$ is prime to $b$. 
Write
\begin{align*}
g_1 & = x_{12}(mk)\cdot g =
\begin{pmatrix} p^\beta q & * \\ p^\beta bm & * \end{pmatrix}. \\
\intertext{
There exists $u\geqslant 1$ such that $p^u\equiv b\pmod q$, say $p^u=b+lq$. Then
}
g_2 & = x_{21}(ml)\cdot g_1 =
\begin{pmatrix} p^\beta q & * \\ mp^{\beta+u} & * \end{pmatrix}. \\
\intertext{
Since $g_2\equiv 1\pmod m$, we can write $p^\beta q=1+cm$ for some $c$. Now set
}
g_3 & = x_{12}\left(\dfrac{-c}{p^{\beta+u}}\right)\cdot g_2 =
\begin{pmatrix} 1 & * \\ mp^{\beta+u} & * \end{pmatrix}, \\
g_4 & = x_{21}\left(-mp^{\beta+u}\right)\cdot g_3 =
\begin{pmatrix} 1 & * \\ 0 & * \end{pmatrix}, \\
g_5 & = x_{12}\left(\dfrac{c}{p^{\beta+u}}\right)\cdot g_4 =
\begin{pmatrix} 1 & * \\ 0 & * \end{pmatrix}.
\end{align*}
Notice that $g_5=z_{21}\left(-mp^{\beta+u}, c/p^{\beta+u}\right)\cdot g_2$ hence $g=x_{12} \cdot x_{21} \cdot z_{21} \cdot x_{12}$
and the length of $g$ does not exceed $4$.

\textsc{Case 2:} $\alpha<\beta$. 
Since $\mathbb{Z}[\sfrac{1}{p}]/I$ is finite, there exists $k>0$ such that $p^k\equiv 1\pmod I$.
One can choose $k>\beta-\alpha$.
Then $k+\alpha>-k+\beta$ and
\[ h_{12}\left(p^k\right)\cdot g =
\begin{pmatrix} p^k & 0 \\ 0 & p^{-k} \end{pmatrix}
\begin{pmatrix} p^\alpha a & * \\ p^\beta bm & * \end{pmatrix}=
\begin{pmatrix} p^{k+\alpha} a & * \\ p^{-k+\beta} bm & * \end{pmatrix}. \]
We find ourselves in the situation of the previous case, therefore, we can write $g=h_{12}\cdot x_{12} \cdot x_{21} \cdot z_{21} \cdot x_{12}$.
Finally, expressing $h=x_{21}\cdot z_{21}\cdot x_{12}$ as in \eqref{eq:rel-tor-elementary}, we get that $g=x_{21} \cdot z_{21} \cdot x_{12} \cdot x_{21} \cdot z_{21} \cdot x_{12}$.
\end{proof}

For the rest of this subsection $k$ denotes a global field and $S$ some finite set of places on $k$. 
Let $\mathcal{O}_S$ be the Dedekind ring of arithmetic type defined by $S$ and let $I$ be an ideal of $\mathcal{O}_S$.

\begin{lemma}\label{lemma:width-dedekind}
Let $\Phi$ be an irreducible classical root system of rank $\ell \geqslant 2$.
If $k$ has a real embedding, then $\G(\Phi, \mathcal{O}_S, I)$ has finite width with respect to the generating set $\mathcal{Z}(\Sigma_\ell)$.
\end{lemma}
\begin{proof}
First of all, notice that $\asr(I) \leqslant \asr(\mathcal{O}_S) \leqslant 2$. 
By \cref{cor:bass-kolster-iterated} we can write any element 
of $G=\G(\Phi, \mathcal{O}_S, I)$ as a product of a finite number of generators from $\mathcal{Z}(\Sigma_\ell)$ and one element of 
$G_0 = \G(\{\alpha_\ell, -\alpha_\ell\}, \mathcal{O}_S, I)\cong\SL(2, \mathcal{O}_S, I)$.
Consequently, to prove the statement of the lemma it suffices to express every element 
$g = \begin{psmallmatrix}1+a & b \\ c & 1+d \end{psmallmatrix} \in G_0$
as a product of a finite number of generators contained in some rank $2$ subgroup of $G$ containing $G_0$.

From $\det(g)=1$ we conclude that $a+d=bc-ad\in I^2$. 
Recall that the Vaserstein's congruence subgroup is defined as
\[ G(I, I)=\left\{ \begin{pmatrix}1+a & b \\ c & 1+d\end{pmatrix}\in\SL(2, \mathcal{O}_S)\;\middle|\; a, d\in I^2, \ b, c\in I \right\}. \]
Notice that $g_1=g\cdot z_{21}(a, 1)$ is contained in $G(I, I)$, indeed,
\[ \begin{pmatrix} 1+a & b \\ c & 1+d \end{pmatrix} \cdot \begin{pmatrix} 1-a & -a \\ a & 1+a \end{pmatrix} = \begin{pmatrix} 1+ba-a^2 & b-a-ba-a^2 \\ c+a+ad-ac & 1+bc-ac \end{pmatrix} \in G(I, I). \]
For any matrix $g'=\begin{psmallmatrix}1+a & b \\ c & 1+d\end{psmallmatrix}\in G(I, I)$ the matrix $x_{21}(-c)\cdot g'\cdot x_{12}(-b)$ lies in $\SL\left(2, \mathcal{O}_S, I^2\right)$.

By \cref{lemma:Stepanov-ideal} the group $\E\left(\Phi, \mathcal{O}_S, I^2\right)$ is contained in $\E(\Phi, I)$ for any root system $\Phi\neq\rC_\ell$ of rank $\geqslant2$.
Notice that under the assumptions of the lemma it is known that $\E(\Phi, I)$ has finite width with respect to $\mathcal{X}$, see~\cite[Theorem~3.3]{TavgenThesis}.

In remains to consider the case $\Phi=\rC_\ell$. First of all, notice that $2abc-abd\in II^{\indexbox{2}}$, indeed,
\[ \det(g_1)=a^3d-3a^2bc+a^2bd+ab^2c+a^3+a^2b+a^2d+2abc-abd+1. \]
This means that
\[ g_2=x_{21}(-a-c)\cdot g_1\cdot x_{12}(a-b)\equiv
\begin{pmatrix}
1+ab-a^2 & -ab-a^2 \\ ad-ac-abc & 1-ab+a^2
\end{pmatrix}\bmod II^{\indexbox{2}}. \]
Now for $g_3=g_2\cdot z_{12}\left(a^2-ab, 1\right)$ we have that
\begin{align*}
& g_3\equiv\begin{pmatrix} 1 & -2ab \\ -abc-a^2+ab-ac+ad & 1 \end{pmatrix}\bmod II^{\indexbox{2}}, \\
& g_4=x_{12}(2ab)\cdot g_3\equiv x_{21}\left(-abc-a^2+ab-ac+ad\right)\bmod II^{\indexbox{2}}.
\end{align*}
Thus $g_4\cdot x_{21}(*)\in\SL\left(2, \mathcal{O}_S, II^{\indexbox{2}}\right)$ is contained in $\E(\rC_\ell, I)$ by \cref{lemma:Stepanov-ideal} and therefore can be expressed as a bounded product of $x_\alpha$.
\end{proof}
The above lemmas together with \cref{cor:bass-kolster-count,thm:Gauss} imply the following result
 which is a quantitative analogue of \cite[Theorem~3.4]{S} and an analogue of the results of~\cite{VseUnitrZ1p, VavSmSuUnitrEng, Tavgen91} for relative groups.
\begin{thm}\label{thm:width} Let $I$ be an ideal of a commutative ring $R$.
\begin{thmlist}
\item If $R=\mathcal{O}_S$ is a Dedekind ring of arithmetic type posessing a real embedding and $\Phi$ is classical of rank $\ell\geqslant2$, then 
$W(\G(\Phi, R, I), \mathcal{Z}(\Sigma_\ell^-))$ is finite;
\item If $\sr(I) = 1$ and $\Phi$ is an arbitrary irreducible root system, then 
\[W(\E(\Phi, R, I), \mathcal{Z}(\Pi))\leqslant 3|\Phi^+|+2\rk(\Phi)-1;\]
\item If $R = \mathbb{Z}[\sfrac{1}{p}]$ for some prime number $p$, then under the assumption of the Generalized Riemann Hypothesis one has
\begin{alignat*}{2}
& W(\E(\Phi, R, I), \mathcal{Z}(\Sigma_\ell^-))\leqslant 3|\Phi^+| + 2\ell + 1 & \text{ for } \Phi=\rA_\ell, \rC_\ell, \\
& W(\E(\Phi, R, I), \mathcal{Z}(\Sigma_\ell^-))\leqslant 4|\Phi^+| + \ell + 1 & \text{ for } \Phi=\rB_\ell, \rD_\ell.
\end{alignat*}
\end{thmlist}
\end{thm}

\printbibliography

\end{document}
