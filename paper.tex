\documentclass[12pt]{amsart}
\usepackage{amscd, amsmath, amssymb, amsthm, amsfonts, amstext, verbatim, enumitem, graphicx, mathtools, xfrac, tikz-cd, nameref, thmtools, hyperref}
\usepackage[utf8]{inputenc}
\usepackage[backend=biber, citestyle=numeric-comp, natbib=true, sortlocale=en_US, url=false, doi=false, eprint=true, maxbibnames=4]{biblatex}            
\usepackage[activate={true,nocompatibility}]{microtype}
\usepackage[capitalize]{cleveref}

%Bibliography
\renewbibmacro{in:}{\ifentrytype{article}{}{\printtext{\bibstring{in}\intitlepunct}}}
\newbibmacro{string+doi}[1]{\iffieldundef{doi}{\iffieldundef{url}{#1}{\href{\thefield{url}}{#1}}}{\href{http://dx.doi.org/\thefield{doi}}{#1}}}
\DeclareFieldFormat{title}{\usebibmacro{string+doi}{\mkbibemph{#1}}}
\DeclareFieldFormat[article, inproceedings, inbook, thesis]{title}{\usebibmacro{string+doi}{\mkbibquote{#1}}}
\addbibresource{paper.bib}

\newlist{thmlist}{enumerate}{1} \setlist[thmlist]{label=(\roman{thmlisti}), ref=\thethm.(\roman{thmlisti}),noitemsep} \Crefname{thmlisti}{Theorem}{Theorems}
\newlist{lemlist}{enumerate}{1} \setlist[lemlist]{label=(\roman{lemlisti}), ref=\thelemma.(\roman{lemlisti}),noitemsep} \Crefname{lemlisti}{Lemma}{Lemmas}

\theoremstyle{plain} \declaretheorem[name=Theorem, Refname={Theorem,Theorems}]{thm} \Crefname{thm}{Theorem}{Theorems}
\numberwithin{equation}{section}

\newtheorem{lemma}{Lemma} \numberwithin{lemma}{section} \Crefname{lemma}{Lemma}{Lemmas}
\newtheorem{cor}[lemma]{Corollary} \Crefname{cor}{Corollary}{Corollaries}
\newtheorem{prop}[lemma]{Proposition} \Crefname{prop}{Proposition}{Propositions}
\newtheorem*{thm*}{Theorem}
\newtheorem*{lemma*}{Lemma}
\theoremstyle{definition} \newtheorem{dfn}[lemma]{Definition} \Crefname{dfn}{Definition}{Definitions}
\newtheorem{example}[lemma]{Example} \Crefname{example}{Example}{Examples}
\theoremstyle{remark} \newtheorem{rem}[lemma]{Remark} \Crefname{rem}{Remark}{Remarks}

%Page layout
\textwidth 16cm 
\textheight 22cm 
\headheight 0.5cm 
\evensidemargin 0.3cm 
\oddsidemargin 0.2cm

%Math operators
\DeclareMathOperator{\K}{K}
\DeclareMathOperator{\SK}{SK}
\DeclareMathOperator{\G}{G}
\DeclareMathOperator{\GL}{GL}
\DeclareMathOperator{\SL}{SL}
\DeclareMathOperator{\Sp}{Sp}
\DeclareMathOperator{\SO}{SO}
\DeclareMathOperator{\St}{St}
\DeclareMathOperator{\E}{E}
\DeclareMathOperator{\EP}{EP}
\DeclareMathOperator{\Par}{P}
\DeclareMathOperator{\Hom}{Hom}
\DeclareMathOperator{\B}{B}
\DeclareMathOperator{\Hh}{H}
\DeclareMathOperator{\U}{U}
\DeclareMathOperator{\Z}{Z}
\DeclareMathOperator{\M}{M}
\DeclareMathOperator{\SR}{SR}
\DeclareMathOperator{\sr}{sr}
\DeclareMathOperator{\shape}{shape}
\DeclareMathOperator{\Rad}{Rad}
\DeclareMathOperator{\Max}{Max}
\DeclareMathOperator{\Spec}{Spec}
\DeclareMathOperator{\Spin}{Spin}
\DeclareMathOperator{\Epin}{Epin}
\DeclareMathOperator{\Stab}{Stab}
\DeclareMathOperator{\ASR}{ASR}
\DeclareMathOperator{\asr}{asr}
\DeclareMathOperator{\Ums}{Ums}
\DeclareMathOperator{\Umd}{Umd}
\DeclareMathOperator{\rk}{rk}
\newcommand{\rA}{\mathsf{A}}
\newcommand{\rB}{\mathsf{B}}
\newcommand{\rC}{\mathsf{C}} 
\newcommand{\rD}{\mathsf{D}} 
\newcommand{\rE}{\mathsf{E}}
\newcommand{\rF}{\mathsf{F}}
\newcommand{\rG}{\mathsf{G}}

\makeatletter
\newcommand{\indexbox}[1]{\text{\fboxsep=.1em\fbox{\m@th$\displaystyle#1$}}}
\makeatother

\def\ssub#1{\mathchoice
   {_{\lower2pt\hbox{$\scriptstyle #1$}}}
   {_{\lower2pt\hbox{$\scriptstyle #1$}}}
   {_{\lower1.5pt\hbox{$\scriptscriptstyle #1$}}}
   {_{\!\lower1.5pt\hbox{$\scriptscriptstyle #1$}}}}

\newcommand\restr[2]{\ensuremath{\left.#1\right|_{#2}}}   
   
\title {Decompositions of congruence subgroups of Chevalley groups}
\keywords {Chevalley groups, principal congruence subgroups, stability for $\K_1$, bounded generation, $\SL_2$-factorizations. {\em Mathematical Subject Classification (2010):} 20G40, 20G35, 19B14}
\author {Sergey Sinchuk, Andrei Smolensky}
\email {sinchukss {\it at} yandex.ru,\ andrei.smolensky {\it at} gmail.com}
\date {\today}

\begin{document}

\begin{abstract} 
We formulate and prove relative versions of several classical decompositions known in the theory of Chevalley groups over commutative rings. % Gauss, Bass---Kolster and Dennis---Vaserstein decompositions.
As an application we obtain upper estimates for the width of principal congruence subgroups in terms of several families of generators.
Some of our results are new even in the absolute case and were previously studied only for groups over finite fields.
\end{abstract}

\maketitle

\section{Introduction}\label{sec:intro}
%Let $R$ be a commutative unital ring and $\Phi$ be a reduced irreducible root system of rank $\ell$ with a fixed basis of simple roots.
Let $\G(\Phi, R)$ the split simply-connected Chevalley group of type $\Phi$ over arbitarary commutative ring $R$ and let $\E(\Phi, R)$ be its \emph{elementary subgroup}
i.\,e. the subgroup generated by root unipotents $x_\alpha(\xi)$, $\alpha\in\Phi$, $\xi\in R$ see~\cite{VP, St78, S}.
For an ideal $I \trianglelefteq R$ we denote by $\G(\Phi, R, I)$ the \emph{relative Chevalley group} (aka \emph{principal congruence subgroup}) of level $I$, 
i.\,e. the kernel of the map $\G(\Phi, R)\rightarrow\G(\Phi, R/I)$ induced by the canonical projection $R\rightarrow R/I$.

In our theorems we also use the notion of the \emph{relative elementary Chevalley group} $\E(\Phi, R, I) \leq \G(\Phi, R, I)$.
Recall that $\E(\Phi, R, I)$ is defined as the normal closure in $\E(\Phi, R)$ of the subgroup generated by the set $\mathcal{X} = \{x_\alpha(\xi) \mid \alpha\in \Phi,\ \xi \in I \}$.

Set $z_\alpha(s, \xi) = x_{\alpha}(s)^{x_{-\alpha}(\xi)}$, $s\in I$, $\xi\in R$.
It is a classical result of J.~Tits and L.~Vaserstein (see~\cite{Tits76}, \cite[Theorem~2]{Va86}) that the relative elementary group $\E(\Phi, R, I)$ is generated by
the set of elements $\mathcal{Z}(\Phi) = \{ z_\alpha(s,\xi) \mid s\in I,\ \xi \in R,\ \alpha \in \Phi\}.$

The following recent result of A.~Stepanov served as a starting point for the present paper.
The notation for the subgroups involved in the statements of our results below is introduced properly in section~\ref{sec:prelim}.
\begin{prop}[{\cite[Theorem~3.4]{S}}] \label{prop:Stepanov}
Let $\Phi$ be a reduced irreducible root system of rank $\geq 2$, let $S \subset \Phi$ be any parabolic subset of roots.
Let $\Sigma_S$ be the special part of $S$, then $\E(\Phi, R, I)$ is generated as an abstract group by the set $\mathcal{Z}(\Sigma_S)$, where
$$\mathcal{Z}(\Sigma) = \mathcal{X} \cup \{ z_\alpha(s, \xi) \mid s \in I,\ \xi \in R,\ \alpha\in \Sigma \}.$$ \end{prop}

The purpose of this paper is two-fold. 
First, we obtain relative analogues of several classical parabolic factorizations of Chevalley groups such as Gauss, Bass---Kolster and Dennis---Vaserstein decompositions.
Secondly, we apply these factorizations to the study of various bounded generation properties of Chevalley groups over rings.
%To be more specific, we study relative analogues of the following three kinds of decompositions:
%\begin{itemize}
%\item \emph{Gauss decomposition,} i.\,e. a decomposition of the form $\E(\Phi,R)=\B \cdot \U^- \cdot \U$, where $\B$ denotes the standard Borel subgroup,
%      and $\U$, $\U^-$ are the unipotent radicals of $\B$ and the opposite Borel subgroup $\B^-$;
%\item \emph{Bass---Kolster decomposition} $G = \Par_i \cdot \U_i^- \cdot \U_i \cdot \U_i^-$, where 
%      $\Par_i$, $\Par_i^-$ are the opposite maximal parabolic subgroups of $\G$ corresponding to $i$-th simple root of $\Phi$ and $\U_i$, $\U_i^-$ are their unipotent radicals;
%\item \emph{Dennis---Vaserstein decomposition} $\E(\Phi, R) = \EP_i \cdot (\U_i^- \cap \U_j^-) \cdot \EP_j$, $i\neq j$,
%      where $\EP_i$ denotes the subgroup generated by root unipotents $x_\alpha(\xi)$ contained in $\Par_i$.
%\end{itemize}
%TODO: Insert some links to VIP's papers :)

%TODO: Illustrate each statement of the text below with a reference:
%There are two main reasons to study parabolic factorizations.
%Recall that a number of various factorizations of $\G$ are known over small-dimensional rings.
%For example, Bruhat decomposition is the key instrument in the study of linear groups over fields, whereas Gauss decomposition and unitriangular factorizations are useful tools in the study of groups over local rings.
%Moreover, for each particular analytic or arithmetic setting there is a special decomposition theorem (Iwasawa, Cartan and Iwahori decompositions, to mention a few).
%Unfortunately, neither of these decompositions can be carried over to rings of higher dimension.
%The main advantage of parabolic factorizations lies in the fact that they remain true for a broader class of rings described in terms of milder ring-theoretic assumptions formulated in terms of the \emph{stable rank} of the ground ring.
%Their another advantage is that they prove useful in the study of asymptotical properties of Chevalley groups. We are going to discuss this theme in more detail below.

Our first result is a relative version of Gauss decomposition formulated in terms of the \emph{stable rank} of the ideal $I$ (see section~\ref{sec:stability-conditions} for the definition)
\begin{thm}\label{thm:srRI1}
Let $\Phi$ be a root system, let $I$ be an ideal of arbitrary commutative ring $R$ such that $\sr(I)=1$.
Then the relative elementary Chevalley $\E(\Phi, R, I)$ group admits the following decomposition
\[ \E(\Phi,R,I) = \Hh(\Phi,R,I) \cdot \U(\Phi,I) \cdot \U(\Phi^-,I) \cdot \U(\Phi,I). \]
\end{thm}
In the special case when $I$ is contained in the Jacobson radical of $R$ the above theorem boils down to~\cite[Proposition~2.3]{AbeSuzNormalSubgroups}. 
The special case $I=R$ was treated by the second named author in~\cite{Sm12}.

The next main result is a relative Dennis---Vaserstein decomposition which holds for a broader class of groups of rings as compared to \cref{thm:srRI1}.
\begin{thm}\label{thm:DennisVaserstein}
Let $\Phi \neq \rE_8$ be a reduced irreducible root system and let $\{ r, s \}$ be a pair of distinct terminal vertices of the Dynkin diagram of $\Phi$ the distance between which equals $d$.
Assume that $\Phi$ and $I$ satisfy either of the following assumptions:
\begin{enumerate}
 \item $\asr(I) \leq d$ for $\Phi=\rF_4$ or $\Phi=\rE_\ell$, $\{r, s\} = \{1, \ell \}$,
 \item $\sr(I) \leq d$, otherwise.
\end{enumerate}
Then $\E(\Phi, R, I) = \EP_r(R, I) \cdot \U(\Sigma^-_r \cap \Sigma^-_s, I) \cdot \EP_s(R, I).$ \end{thm}
Even in the special case $I=R$ the above theorem contains several new cases as compared to~\cite[Theorems~2.5, 4.1]{St78}.
The term ``Dennis---Vaserstein decomposition'' is due to A.~Suslin and M.~Tulenbaev (cf.~\cite[Lemma~2.1]{ST76}).

The condition $\asr(I) \leq d$ used in the statement of \cref{thm:DennisVaserstein} is a relative version of the \emph{absolute stable range} condition of D.~Estes and J.~Ohm which depends only on the ideal $I$ (see section~\ref{sec:rel-asr}).

Let $G$ be an abstract group with a generating set $X \subset G$. 
We denote by $W(G, X)$ the \emph{width of $G$ with respect to $X$}, i.\,e. the smallest natural number $N$ such that every element of $G$ is a product of at most $N$ elements of $X$.
%The problem of finding an estimate for the width of $\G(\Phi, R)$ in terms of elementary generators $x_\alpha(\xi)$ has been extensively studied in a number of works (see e.\,g. \cite{CK83, Ka82, Tavgen91, Mo07, VseUnitrZ1p, VavSmSuUnitrEng}).
%Summarizing results of these works, one can hope that such width is finite only for rather narrow classes of rings of number-theoretic nature, e.\,g. localizations of Dedeking rings of arithmetic type.
%For example, W.~van der Kallen has shown that the width of $\SL(3,R)$ is infinite already for a Euclidean ring $R=\mathbb{C}[t]$.

%On the other hand, much less is known about the width of the relative group $\G(\Phi, R, I)$ and the relative elementary group $\E(\Phi, R, I)$.

Our first application of parabolic factorizations is the following theorem which gives an estimate of the width of $\E(\Phi, R)$ in terms of Tits generators and is, thus, a quantitative analogue of \cref{prop:Stepanov}
and a relative analogue of the results of~\cite{CK83, Tavgen91, Mo07, VseUnitrZ1p, VavSmSuUnitrEng}.
\begin{thm}\label{thm:width} Let $I$ be an ideal of a commutative ring $R$.
\begin{enumerate}
\item If $R=\mathcal{O}_S$ is a Dedekind ring of arithmetic type posessing a real embedding and $\Phi$ is classical of rank $\geqslant2$, then 
$W(\G(\Phi,\mathcal{O}_S,I), \mathcal{Z}(\Sigma_\ell^-))$ is finite.
\item If $\sr(I) = 1$ and $\Phi$ is arbitrary irreducible root system then 
\[W(\E(\Phi,R,I),\mathcal{Z}(\Pi))\leqslant 3|\Phi^+|+2\rk(\Phi)-1;\]
\item If $R = \mathbb{Z}[\sfrac{1}{p}]$ for some prime number $p$, then under the assumption of the Generalized Riemann Hypothesis one has
\begin{alignat*}{2}
& W(\E(\Phi,R,I), \mathcal{Z}(\Sigma_\ell^-))\leqslant 3|\Phi^+| + 2\rk(\Phi) + 1 & \text{ if } \Phi=\rA_\ell,\rC_\ell, \\
& W(\E(\Phi,R,I), \mathcal{Z}(\Sigma_\ell^-))\leqslant 4|\Phi^+| + \rk(\Phi) + 1 & \text{ if } \Phi=\rB_\ell,\rD_\ell;
\end{alignat*}
\end{enumerate}
\end{thm}
The first assertion of the above theorem is essentially a reformulation of a previously known result of O.~Tavgen (see \cite{Tavgen93, TavgenThesis}),
while for the second and the third assertion we give a direct proof.
Compare \cref{thm:width} with a recent result of U.~Hadad and D.~W.~Morris \cite[Theorem~1.6]{Ha12} which asserts that 
$W(\SL(n, \mathbb{Z}, I), \mathcal{Z}(\Pi))$ is finite for any ideal $I \leq \mathbb{Z}$ and $n \geq 3$.

Another application of parabolic factorizations developed in this paper are subsystem factorizations in terms of subgroups of type $\rA_1$.
The case of a finite field $R=F$ was studied by M.~Liebeck, N.~Nikolov and A.~Shalev in~\cite{LiebNikShaSL2}.
More precisely, for an untwisted Chevalley group it has been proved that $\G(\Phi, F)$ is a product of at most $N=5|\Phi^+|$ copies of $\SL(2, F)$.

In~\cite{VavKovSL2} N.~Vavilov and E.~Kovach noted that Bruhat decomposition immediately implies a bound of $N=3|\Phi^+|$ factors for arbitrary field $F$.
In fact, in the context of \cref{thm:srRI1} one can immediately obtain the decomposition of $\E(\Phi, R, I)$ with $3|\Phi^+|$ factors of type $\rA_1$.
Our next result further extends this to even larger class of rings (which e.\,g. includes ideals of all Dedekind rings).
\begin{thm}\label{thm:SL2width}
Let $I\trianglelefteq R$ be an ideal, $\Phi$ be an irreducible classical root system satisfying the following assumptions.
\[\begin{array}{l@{\quad}l@{\quad}l}
\Phi=\rA_\ell, & \sr(I)\leqslant 2, & N=3\left|\Phi^+\right| - \rk(\Phi) - 1; \\
\Phi=\rC_\ell, & \sr(I)\leqslant 3, & N=3\left|\Phi^+\right| - 2; \\
\Phi=\rB_\ell,\rD_\ell, & \asr(I)\leqslant 2, & N=4\left|\Phi^+\right| - 3\rk(\Phi).
\end{array}\]
Then the principal congruence subgroup $\G(\Phi,R,I)$ can be presented as a product of at most $N$ copies of its (regularly embedded) subgroups isomorphic to $\SL_2(R, I)$.
\end{thm}

Our yet another application of parabolic factorizations concerns product decompositions of Chevalley groups formulated in terms of subgroups of type $\rA_\ell$ of submaximal rank.
Recall that in \cite{NikProdDecomp} N.~Nikolov proves the following result. 
\begin{thm*} Let $G$ be a classical Chevalley group (of normal or twisted type) of rank $n$ over a finite field.
Then $G$ equals the product of at most $200$ conjugates of $\SL_n$. \end{thm*}

It is classically known that over an arbitrary field $F$ the group $\SL_{n+1}(F)$, $n\geq 3$ is a product of at most $4$ subgroups of type $\SL(n, F)$.
On the other hand, one immediately obtains from \cref{thm:DennisVaserstein} that $\SL_{n+1}(R, I)$ factors into a product of a most $5$ subgroups of type $\SL_n(R, I)$ for any ideal $I$ satisfying $\sr(I)\leqslant n-1$.

Applying \cref{thm:DennisVaserstein} and generalizing Nikolov's \cite[Proposition~1]{NikProdDecomp} in the special case $\Phi=\rD_\ell$ to arbitrary rings we obtain the following result.
\begin{thm}\label{thm:spin-sln-prod}
Assume that $\sr(I) \leqslant 2$. Then the group $\Epin_{2\ell}(R, I)=\E(\rD_\ell, R, I)$ is a product of at most $9$ conjugates of its (regularly embedded) subgroups of type $\rA_{\ell-1}$.
\end{thm}

The rest of the article is organized as follows.
In sections \ref{sec:prelim}--\ref{sec:stability-conditions} we introduce principal notation pertaining to Chevalley groups and stability conditions.
In section \ref{sec:factorizations} we prove our parabolic factorizations and thus obtain Theorems~\ref{thm:srRI1}, \ref{thm:DennisVaserstein} and \ref{thm:SL2width}.
Finally, we treat bounded generation and product decompositions separately in section~\ref{sec:applications} where Theorems~\ref{thm:width} and \ref{thm:spin-sln-prod} are proved.

\subsection{Acknowledgements}
Authors of the present paper acknowledge financial support from Russian Science Foundation grant 14-11-00297.

\section{Preliminaries}\label{sec:prelim}
For any collection of subsets $H_1,\ldots, H_n$ of a group $G$ we denote by $H_1\ldots H_n$ their Minkowski set-product,
i.\,e. the set consisting of arbitrary products $h_1\ldots h_n$ of elements $h_i\in H_i$. In particular, the equality
$G = H_1\cdot\ldots\cdot H_n$ means that every element $g\in G$ can be presented as a product $h_1\ldots h_n$ for $h_i\in H_i$.

\subsection{Root systems}\label{sec:rootsys}
Let $\Phi$ be a reduced irreducible root system of rank $\ell$ with a fixed basis of simple roots $\Pi=\{\alpha_1, \ldots, \alpha_l\}$.
We use conventional numbering of basis vectors of $\Pi$ which follows Bourbaki (cf.~\cite[Table~1]{PSV98}).
Denote by $\Phi^+$ and $\Phi^-$ the subsets of positive and negative roots with respect to $\Pi$.
Let $\alpha_\mathrm{max}$ denote the maximal root of $\Phi$.
For a root $\alpha\in\Phi$ we denote by $m_i(\alpha)$ the $i$-th component of $\alpha$ in $\Pi$, i.\,e. $\alpha = \sum_{i=1}^n m_i(\alpha) \alpha_i$.

A proper closed subset of roots $S\subseteq \Phi$ is called {\it parabolic} (resp. {\it reductive}, resp. {\it special}) if $\Phi=S \cup -S$ (resp. $S = -S$, resp. $S \cap -S=\varnothing$).
Any parabolic subset $S \subseteq \Phi$ can be decomposed into the disjoint union of its \emph{reductive} and \emph{special} parts, i.e. 
$S = \Sigma_S \sqcup \Delta_S$, where $\Sigma_S \cap (-\Sigma_S) = \varnothing$, $\Delta_S = -\Delta_S$.

We denote by $(\alpha, \beta)$ the scalar product of roots and by $\langle \alpha, \beta\rangle$ the integer $2(\alpha, \beta)/(\beta, \beta)$.
Let $\varpi_1, \ldots, \varpi_\ell$ be the \emph{fundamental weights }of $\Phi$, i.\,e. vectors
defined by $\langle\varpi_i,\alpha_j^\vee\rangle=\delta_{i,j},$ where $\alpha^\vee = \frac{\alpha}{2(\alpha,\alpha)}$.

Denote by $W(\Phi)$ the subgroup of isometries of $\Phi$ generated by all reflections $\sigma_\alpha$ where $\sigma_\alpha(\beta)=\beta-\langle\alpha,\beta \rangle\cdot \alpha$.
For a reductive subset $\Delta\subseteq \Phi$ denote by $W(\Delta)$ the subgroup of $W(\Phi)$ generated by $\sigma_\alpha$ for $\alpha\in\Delta$.

Let $J\subseteq \Pi$ be a subset of simple roots. 
For a root $\beta = \sum m_i(\beta)\alpha_i \in \Phi$ define \emph{$J$-shape} of $\beta$ (denoted $\shape(J, \beta)$) as $\sum_{i\in J} m_i(\beta) \alpha_i$.
Set
\begin{align*}
& \Delta_J = \{\alpha \in \Phi \mid \shape(J, \alpha)=0\},\\
& \Sigma^\pm_J = \{\alpha \in \Phi \mid \shape(J, \alpha) \in \mathbb{Z}_{\geqslant0} \Phi^\pm \},\\
& S_J^\pm = \Delta_J \sqcup \Sigma_J^\pm.
\end{align*}
Clearly, $\Delta_J$ is a reductive subset, while $S^\pm_J$ and $\Sigma^\pm_J$ are parabolic and special subsets, respectively.
For two disjoint subsets $I, J\subseteq \Pi$ one has 
\[ \Sigma^\pm_{I \cup J} = \Sigma^\pm_I\cup\Sigma^\pm_J, \quad \Delta_{I\cup J} = \Delta_I \cap \Delta_J. \]
We tend to omit curly braces in the above notations in the case when $J$ is a one or two element set, e.\,g. $\Delta_k=\Delta_{\{k\}}$ and $\Sigma_{i,j}=\Sigma_{\{i,j\}}$ etc.

\begin{lemma}[{\cite[Lemma~1]{ABS}}]\label{lemma:abs}
Let $\alpha, \beta \in \Sigma^\pm_J$ be two roots having the same length such that $\shape(J,\alpha)=\shape(J,\beta)\neq 0$.
Then $\alpha$ and $\beta$ are conjugate under the action of $W(\Delta_J)$.
\end{lemma}

\subsection{Chevalley groups and their elementary subgroups} \label{sec:elementary}
Our exposition of Chevalley groups is standard and follows~\cite{Ta, S, St78, VP, Va86}.
We denote by $\G(\Phi, R)$ the simply-connected split Chevalley group of type $\Phi$ over arbitrary commutative ring $R$. For example, for each of the four classical series $\Phi=\rA_\ell, \rB_\ell, \rC_\ell, \rD_\ell$
the group $\G(\Phi, R)$ equals $\SL(\ell+1, R)$, $\Spin(2\ell+1, R)$, $\Sp(2\ell, R)$, $\Spin(2\ell, R)$, respectively. 

Notice that from the uniqueness theorem of M.~Demazure (see~\cite[Corollaire~5.2]{SGA3}) 
it follows that the exceptional isomorphisms of small-rank groups (which are well-known in the theory of groups over fields) remain valid for Chevalley groups over arbitrary commutative rings.
In particular, there are isomorphisms of groups $\Spin(3,R)\cong\SL(2,R)$, $\Spin(5, R)\cong \Sp(4, R)$, $\Spin(6, R)\cong\SL(4, R)$, $\SL(2, R)\cong \Sp(2, R)$.  

Recall from~\cite{St78, VP} that for $\alpha\in \Phi$, $\xi\in R$ one can define certain elements $x_{\alpha}(\xi)$ of $\G_{sc}(\Phi, R)$ called {\it elementary root unipotents}.
These elements satisfy the well-known Steinberg relations:
\begin{align}
& \phantom{[}
x_\alpha(s) x_\alpha(t) = x_\alpha(s+t), \label{rel:add}\\
& [x_\alpha(s),  x_\beta(t)] = \prod x_{i\alpha + j\beta}\left(N_{\alpha\beta ij}\, s^i t^j\right),\quad \alpha\neq-\beta, \quad N_{\alpha\beta ij}\in\mathbb{Z}. \label{rel:CCF}
\end{align} 
The indices $i$, $j$ in the above formula range over all positive natural numbers such that $i\alpha + j\beta\in\Phi$.
The integers $N_{\alpha\beta ij}$ are called {\it structure constants} of the Chevalley group $\G(\Phi,R)$ and depend only on $\Phi$.
We refer the reader to~\cite[\S14]{VP} for basic properties of structure constants.

For an additive subset $I\subseteq R$ we denote by $X_\alpha(I)$ the corresponding \emph{root subgroup} of level $I$, i.\,e. the subgroup consisting of all elementary root unipotents $x_\alpha(\xi)$, $\xi\in I$.

The subgroup $\E(\Phi, R)$, generated by all $X_\alpha(R)$, $\alpha\in\Phi$ is called the {\it elementary subgroup} of the Chevalley group $\G_{sc}(\Phi, R)$.

For $\varepsilon\in R^*$ set $w_\alpha(\varepsilon) = x_\alpha(\varepsilon) x_{-\alpha}(-\varepsilon^{-1}) x_{\alpha}(\varepsilon).$
If $\rk(\Phi)\geqslant 2$ the following relation is a consequence of \eqref{rel:add}--\eqref{rel:CCF}:
\begin{equation}\label{rel:R3}
w_\alpha(\varepsilon) x_{\beta}(\xi) w_\alpha(\varepsilon)^{-1} =
x_{\sigma\ssub{\alpha}\beta} \left(\eta_{\alpha, \beta}\cdot\varepsilon^{-\langle\beta,\alpha \rangle}\xi\right),\quad \varepsilon\in R^*,\ \xi\in R.
\end{equation}
where $\eta_{\alpha, \beta}=\pm 1$. The coefficients $\eta_{\alpha, \beta}$ can be expressed through structure constants (cf.~\cite[\S13]{VP}).
For a reductive subset $\Delta \subseteq \Phi$ denote by $\widetilde{W}(\Delta)$ the \emph{extended Weyl group}, i.\,e. the subgroup of $\E(\Phi, R)$ generated by all $w_{\alpha}(1)$, $\alpha \in \Delta$.
\begin{lemma} \label{lemma:weylfacts} Let $\Phi$ be an irreducible root system and let $I$ be an ideal of $R$. 
\begin{lemlist}
\item \label{item-trans1} For every roots $\alpha, \beta \in \Phi$ of the same length there exists $w \in \widetilde{W}(\Phi)$ such that $X_{\alpha}(I)^w = X_\beta(I)$.
\item \label{item-egen} The elementary group $\E(\Phi, R)$  is generated as an abstract group by root subgroups $X_{\alpha}(R)$, $\alpha \in \Pi \cup -\Pi$.
\item \label{item-trans2} Assume that $\alpha_s\in \Pi$ is a simple root of $\Phi$ such that $m_s(\alpha_\mathrm{max}) = 1$. 
                          Then for $\alpha, \beta \in \Sigma^\pm_s$ of the same length there exists $w\in \widetilde{W}(\Delta_s)$ such that $X_\alpha(I)^w = X_\beta(I)$. \end{lemlist}
\end{lemma}
\begin{proof}
The first statement of the lemma follows from identity~\eqref{rel:R3} and the well-known fact that the Weyl group $W(\Phi)$ acts transitively on the set of roots of the same length.
 
The same argument combined with \cref{lemma:abs} also proves the third statement. Indeed, by our assumption all roots of the same length lying in $\Sigma_s$ have the same $s$-shape.
 
To prove the second assertion notice that the subgroup $\langle X_\alpha(R) \mid \alpha \in \Pi \cup -\Pi \rangle$ contains $\widetilde{W}(\Phi)$ and therefore contains all other root subgroups $X_\alpha(R)$, $\alpha\in \Phi$. \end{proof}

Define semisimple root elements $h_\alpha(\varepsilon)$, $\alpha\in\Phi$, $\varepsilon\in R^*$ as $h_\alpha(\varepsilon)=w_\alpha(\varepsilon)w_\alpha(-1)$.
The elements $h_\alpha(\varepsilon)$ satisfy the following relations:
\begin{alignat}{2} 
& [ h_\alpha(\varepsilon_1), h_\beta(\varepsilon_2)] = 1, &\quad& \alpha, \beta \in \Phi,\ \varepsilon_1, \varepsilon_2 \in R^*; \label{rel:h-comm} \\
& \phantom{[} h_\alpha(\varepsilon_1 \varepsilon_2) = h_\alpha(\varepsilon_1) h_\alpha(\varepsilon_2), && \alpha \in \Phi,\ \varepsilon_1, \varepsilon_2 \in R^*; \label{rel:h-mult} \\
& \phantom{[} h_\alpha(\varepsilon)x_\beta(\xi)h_\alpha(\varepsilon)^{-1} = x_\beta\left(\varepsilon^{\langle\beta,\alpha\rangle}\xi\right), && \alpha, \beta \in \Phi,\ \varepsilon \in R^*, \xi \in R. \label{rel:h-w}
\end{alignat}

\subsection{Representations of Chevalley groups} \label{sec:repr}
Our notation and terminology pertaining to representations of Chevalley groups follows \cite[\S~1.4]{PSV98} and \cite[\S~I.2]{Ma69})
Let $\pi$ be an irreducible representation of $\G(\Phi, R)$ acting on a free $R$-module $V$.
We denote by $\Lambda=\Lambda(\pi)$ the set of weights of $\pi$ and by $\Lambda(\pi)^*$ the subset of non-zero weights.
Denote by $V_\lambda \leq V$ the \emph{weight subspace} corresponding to $\lambda\in\Lambda(\pi)$ and denote by $m_\lambda$ the dimension of $V_\lambda$.

A representation $\pi$ is called \emph{basic} if for every pair of non-zero weights $\lambda, \mu \in \Lambda^*(\pi)$ such that $\lambda - \mu \in \Pi$ one has $\sigma_\alpha(\lambda) = \mu$.
Obviously, this is equivalent to saying that $W(\Phi)$ acts transitively on $\Lambda(\pi)^*$.

The irreducible representations of Chevalley groups used in these paper form a sublist of~\cite[Table~2]{PSV98}.
More specifically, all of them are \emph{fundamental} (i.\,e. their highest weight is fundamental) and basic.
We widely employ the technique of \emph{weight diagrams} (see~\cite[\S~2]{PSV98}) to help visualize the structure of these representations.
In the simplest case of a \emph{microweight} representation $\pi$ (i.\,e. $\pi$ such that $V_0 = 0$)
the weight diagram of $(V, \pi)$ is just an undirected graph whose vertices are in one-to-one correspondence with the elements of $\Lambda(\pi)$.
Two vertices $\lambda$, $\mu$ of the diagram are joined by a bond marked $i$ iff $\lambda-\mu = \alpha_i$.

An immediate application of weight diagrams is that they allow one to quickly read off the branching of a given basic representation with respect to a subsystem subgroup.
If $J$ is a subset of $\Pi$ and $\Delta$ is the subsystem of $\Phi$ spanned by $J$ then the weight diagram of $\pi$ restricted to $G(\Delta, R)$ can be obtained
from the weight diagram of $\pi$ by simply removing all the bonds whose label $i$ is such that $\alpha_i\not\in J$ (cf.~\cite[\S~2.7]{PSV98}).
 
Recall that all nonzero weights of a basic representation have multiplicty one and the multiplicity $m_0$ of the zero weight subspace equals $|\Delta(\pi)|$, where $\Delta(\pi)= \Lambda(\pi)^* \cap \Pi$ (see~\cite[Lemma~2.1]{Ma69}).
For $\alpha \in \Phi$ one can define certain elements $e_0(\alpha) \in V_0$ and $\alpha_* \in V_0^* = \Hom_R(V_0, R)$
and choose some set of vectors $E^* = \{ e_\lambda \}_{\lambda \in \Lambda^*(\pi)}$ in such a way that 
$E_\pi = E^* \cup \{ e_0(\alpha) \}_{\alpha \in \Delta(\pi)}$ forms a basis of $V$ in which the action of elementary root unipotents $x_\alpha(\xi)$ has particularly simple description.

\begin{lemma}[{\cite[Lemma~2.3]{Ma69}}]\label{lemma:Matsumoto}
Let $(V, \pi)$ be a basic representation of $\G(\Phi, R)$.
Then the action of $x_\alpha(\xi)$ is described by the following formulae.
\newline \begin{tabular}{ll}
If $\lambda\in\Lambda(\pi)^*,\ \lambda+\alpha\notin\Lambda(\pi)$, then & $x_\alpha(\xi)e_\lambda=e_\lambda$; \\
If $\lambda,\lambda+\alpha\in\Lambda(\pi)^*$, then & $x_\alpha(\xi)e_\lambda=e_\lambda\pm\xi e_{\lambda+\alpha}$; \\
If $\alpha\notin\Lambda(\pi)^*$, then & $x_\alpha(\xi)v_0=v_0$ for any $v_0\in V_0$; \\
If $\alpha\in\Lambda(\pi)^*$, then & $x_\alpha(\xi)e_{-\alpha} = e_{-\alpha}\pm\xi e_0(\alpha)\pm\xi^2 e_\alpha$ and \\
                                   & $x_\alpha(\xi)v_0         = v_0\pm\xi\alpha_*(v_0)e_\alpha$ for $v_0 \in V_0$.
\end{tabular} \end{lemma}

The action of $x_\alpha(\xi)$ can be visualized on a weight diagram of $\pi$ as follows. Blah-blah-blah
\textbf{TODO: Write a passage about this} %TODO:

Notice that the highest weight of natural representations of classical groups is $\varpi_1$.
It will be convenient for us to number the weights of these representations as in~\cite[\S~1B]{St78}:
\[\begin{array}{cll}
  1,2,\ldots, \ell+1 & \text{in the case} & \Phi =\rA_\ell, \\
  1,2,\ldots \ell, 0, -\ell,\ldots, -2, -1 & \text{in the case} & \Phi =\rB_\ell, \\
  1,2,\ldots \ell, -\ell,\ldots, -2, -1 & \text{in the cases}   & \Phi =\rC_\ell, \rD_\ell. \\
\end{array}\]
For example, we write $1$ instead of $\varpi_1$, $2$ instead of $\varpi_1-\alpha_1$ etc.

Although it is possible to fix signs of the action constants in the statement of \cref{lemma:Matsumoto} in the general situation (see~\cite{V08}),
for our purposes it suffices to do that only for classical groups in their natural representations.

\begin{rem}
Let $\lambda_1, \lambda_2 \in \Lambda(\pi)$ be a pair of weights of a representation $\pi$ of a classical group such that $\lambda_1-\lambda_2\in \Phi$.
In this situation it will be convenient for us to write $x_{\lambda_1,\lambda_2}(\xi)$ (for some choice of $\lambda_1,\lambda_2$) instead of $x_{\lambda_1-\lambda_2}(\xi)$.
For example, for $\Phi=\rA_\ell$ we have $x_{1,2}(\xi)=x_{1-2}(\xi)=x_{\varpi_1 - \varpi_1 + \alpha_1}(\xi) = x_{\alpha_1}(\xi)$. \end{rem}
For classical groups there is a standard way to fix the signs.
In case $\Phi=\rC_\ell$ the long and short root unipotents are
\begin{align*}
& x_{i,-i}(\xi)=e+\xi e_{i,-i}, \\ 
& x_{i,j}(\xi)=e+\xi e_{ij}-\varepsilon_i\varepsilon_j\xi e_{-j,-i},\quad i\neq\pm j.
\end{align*}
Here $\varepsilon_i$ is the sign of $i$.
In case $\Phi=\rB_\ell,\rD_\ell$ the long and short root unipotents in $\SO(n,R)$ are
\begin{align*}
& x_{i,j}(\xi)=e+\xi e_{ij}-\xi e_{-j,-i},\quad i\neq\pm j,\ i,j\neq0, \\
& x_{i0}(\xi)=e+2\xi e_{i0}-\xi e_{0,i}-\xi^2 e_{i,-i},\quad i\neq0.
\end{align*}

\subsection{Relative Chevalley groups}\label{sec:relative-elementary}
Recall from the introduction that the elements $z_\alpha(s, \xi) = x_\alpha(s)^{x\ssub{-\alpha}(\xi)}$ generate $\E(\Phi, R, I)$ as an abstract group.

\begin{lemma}[{\cite[Corollary~3.3]{S}}]\label{lemma:Stepanov-ideal}
Let $\Phi$ be a root system of rank $\geqslant2$, let $R$ be a commutative ring and $I\trianglelefteq R$ be its ideal.
If $\Phi\neq\rC_\ell$ then $\E\left(\Phi,R,I^2\right)\leqslant\E(\Phi,I)$, otherwise $\E\left(\Phi,R,II^{\indexbox{2}}\right)\leqslant\E(\Phi,I)$.
\end{lemma}
Here $I^{\indexbox{2}}$ denotes the ideal generated by squares $a^2$, where $a\in I$.
Recall also that $I^2$ is generated by products $ab$ for all $a,b\in I$.
Clearly, $II^{\indexbox{2}}$ is generated by elements of the form $a^2b$ for $a,b\in I$.

For a special subset of roots $\Sigma\subseteq \Phi$ we denote by $\U(\Sigma, I)$ the subgroup spanned by all $x_{\alpha}(I)$ for $\alpha\in \Sigma$. Let $J\subset\Pi$.
The subgroup $\U(\Sigma_J, I)$ is normalized by $\E(\Delta_J, R)$, hence the Minkowski product set $\EP_J(R, I) = \E(\Delta_J, R, I) \cdot \U(\Sigma_J, I)$ is a subgroup called a \emph{standard parabolic subgroup}.
The following two equalities will be referred to in the sequel as {\it Levi decomposition}: 
\begin{equation} \label{rel:Levi-decomp} \EP_J(R, I) = \U(\Sigma_J, I) \cdot \E(\Delta_J, R, I) = \E(\Delta_J, R, I) \cdot \U(\Sigma_J, I). \end{equation}
When $J = \{ \alpha_s \}$ for some $1 \leq s\leq \ell$ we use shorthand $\EP_s(R, I)$ for $\EP_{\{s\}}(R, I)$.
When $I=R$ we also write $\EP_J(R)$ instead of $\EP_J(R, R)$.

Denote by $\Hh(\Phi,R)$ the subgroup generated by all $h_\alpha(\varepsilon)$, $\alpha\in\Phi$, $\varepsilon\in R^*$, and set
\[ \Hh(\Phi,R,I) = \Hh(\Phi,R)\cap\G(\Phi,R,I)=\langle h_\alpha(\varepsilon),\ \alpha\in\Phi,\ \varepsilon\in R^*\cap(1+I)\rangle. \]
\begin{lemma}\label{lemma:rel-tor-elementary}
$\Hh(\Phi,R,I)$ is contained in $\E(\Phi,R,I)$.
\end{lemma}
\begin{proof}
Set $\varepsilon=1+s$, $s\in I$ and write
\begin{multline*}
h_\alpha(1+s) = x_\alpha\left(-1\middle)\, x_{-\alpha}\middle(-s\middle)\, x_\alpha\middle((1+s)^{-1}\middle)\, x_{-\alpha}\middle(s(1+s)\right) = \\
= x_\alpha\left((1+s)^{-1}-1\middle)\, z_{-\alpha}\middle(-s,(1+s)^{-1}\middle)\, x_{-\alpha}\middle(s(1+s)\right).
\end{multline*}
It remains to note that $(1+s)^{-1}\in 1+I$ and so all the factors lie in $\E(\Phi,R,I)$.
\end{proof}

The following lemma is a relative version of the classical result sometimes called Chevalley---Matsumoto decomposition.
\begin{lemma}\label{lemma:Chevalley-Matsumoto}
Let $\pi$ be the fundamental representation of $\G_{sc}(\Phi, R)$ with the highest weight $\varpi_s$.
Assume that $g\in \G_{sc}(\Phi, R, I)$ is such that $(g\cdot v^+)_{\varpi\ssub{s}}=1$, then 
\[ g \in \U(\Sigma_s^-, I) \cdot \G_{sc}(\Delta_s, R, I) \cdot \U(\Sigma_s^+, I). \]
\end{lemma}
\begin{proof}
In the absolute case ($I=R$) the statement of the lemma is contained in~\cite[Theorem~1.3]{St78}. 
Thus we can write $g=u_1\cdot g'\cdot u_2$, where $u_1\in\U(\Sigma_s^-,R)$, $u_2\in\U(\Sigma_s^+,R)$ and $g'\in\G(\Delta_s,R)$.
Since $g$ lies in $\G(\Phi,R,I)$, the vector $gv^+$ is $I$-unimodular. 
On the other hand, $g'u_2v^+=v^+$, so $u_1\in\G(\Phi,R,I)\cap\U(\Sigma_s^-,R)=\U(\Sigma_s^-,I)$.
The matrix $g'u_2$ is an element of $\left(\G(\Delta_s,R)\U(\Sigma_s^+,R)\right)\cap\G(\Phi,R,I)$, and by Levi decomposition $u_2\in\U(\Sigma_s^+,I)$, $g'\in\G(\Delta_s,R,I)$.
\end{proof}

\subsection{$\K_1$-functor modeled on Chevalley groups}
Recall that by Taddei's theorem $\E(\Phi, R)$ is a normal subgroup of $\G_{sc}(\Phi, R)$ provided $\Phi$ is an irreducible root system of rank $\geqslant 2$ (see~\cite{Ta}).

Using standard relativization argument it is not hard to deduce from Taddei's result that $\E(\Phi,R,I)$ is normal in $\G(\Phi,R,I)$ under the same assumptions on $\Phi$.
This allows us to define the relative $\K_1$-group as $\K_1(\Phi,R,I)=\G(\Phi,R,I)/\E(\Phi,R,I).$
When $I=R$, we write $\K_1(\Phi, R)$ for $\K_1(\Phi, R, R)$.

In some cases it is known that $\K_1(\Phi, R, I)$ is trivial.
For example, $\SK_1(\ell+1,R)=\K(\rA_\ell,R)=1$ for any ring of stable rank $1$ (see section~\ref{sec:stability-conditions}), 
and for other root systems some stronger assumption is required such as $\asr(R)=1$ or being semilocal. 
For any Euclidean ring $\K_1(\Phi, R, I)$ is trivial for every root system $\Phi$.

Let $k$ be a global field, $S$ be a set of places of $k$, 
$\mathcal{O}_S$ be the Dedekind ring of arithmetic type defined by $S$ and $I$ be an ideal of $\mathcal{O}_S$.
\begin{prop} \label{prop:K1triv}
Let $\Phi$ be a root system of rank $\geqslant2$. Assume that the field $k$ has a real embedding. Then $\K_1(\Phi,\mathcal{O}_S,I)=1$.
\end{prop}
\begin{proof} Follows from \cite[Theorem~3.6]{BassMilnorSerre} and \cite[Corollary~4.5]{Ma69}. \end{proof}

\section{Stability conditions}\label{sec:stability-conditions}
As we will be mainly concerned with applications to stability of $K_1$ of Chevalley groups, our treatment of stability conditions is restricted to the case of commutative $R$. 
We refer the reader to~\cite{Ba64, Va69, Va71} for a more general exposition.

\subsection{Stable rank}
Recall the definition of {\it stable rank} introduced by H.~Bass and L.~Vaserstein (see~\cite{Ba64, Va69}).
\begin{dfn} Recall that a row $a=(a_1,\ldots, a_n)\in {}^n\!R$ is called {\it $I$-unimodular} if elements $a_1-1, a_2, \ldots, a_n$ are contained in $I$ while $a_1, a_2, \ldots, a_n$ generate $R$ as an ideal.\end{dfn}
A column $b \in R^n$ will be called $I$-unimodular if its transpose $b^t$ is an $I$- unimodular row. We denote the set of all $I$-unimodular rows (resp. columns) by $\Umd(n,R,I)$ (resp. $\Ums(n,R,I)$).
When $R=I$ we refer to $R$-unimodular rows and columns as simply unimodular.

\begin{lemma} \label{lemma:relstrlemma} For any $I$-unimodular row $a=(a_1, \ldots, a_n)$ there exists $I$-unimodular column $b=(b_1,\ldots, b_n)^T$ such that $ab = \sum\limits_{i=1}^n a_i b_i = 1$. \end{lemma}

\begin {dfn} An $I$-unimodular row $a=(a_1,\ldots a_{n+1})$ is called {\it stable} if one can choose $b_1,\ldots, b_n\in R$ such that
row $(a_1 + a_{n+1}b_1,\ldots, a_{n}+ a_{n+1}b_n)$ is also $I$-unimodular. if, moreover, elements $b_1,\ldots, b_n$ can be chosen from  $I$ then such $a$ will be called {\it $I$-stable}. \end{dfn} 

We say that a pair $(R, I)$ satisfies stable range condition $\SR_n(R, I)$ if any $I$-unimodular row of length $n+1$ is stable.
\begin{lemma}\label{lemma:relstrlemma2}\strut\begin{enumerate}\item The condition $\SR_n(R, I)$ implies $\SR_m(R,I)$ for any $m\geq n$.
\item The condition $\SR_n(R, I)$ implies that any $I$-unimodular row of length $n+1$ is $I$-stable.\end{enumerate}\end{lemma}

By definition, the {\it relative stable rank} $\sr(R, I)$ of a pair $(R, I)$ is the smallest natural number $n$ such that condition $\SR_n(R, I)$ holds.
If $\SR_n(R, I)$ is not satisfied for any $n>0$ we set $\sr(R, I)=\infty$.

The following proposition summarizes basic properties of stable ranks.
\begin{prop} \label{prop:sr_properties} Let $R$ be arbitary commutative unital ring and let $I\trianglelefteq R$ be its ideal.
 \begin{itemize}
  \item For any ideal $J\trianglelefteq R$, $J\subseteq I$ one has $$\sr(R, I)\leq\sr(R),\quad \sr(R/J, I/J)\leq \sr(R, I).$$
  \item Stable rank of a direct product of rings is equal to the maximum of stable ranks of its factors: $$\sr(\prod\limits_{i=1}^n R_i) = \max\limits_{i=1}^n\left(\sr(R_i)\right).$$
  \item Stable rank does not change after taking quotient modulo nil-radical: $\sr(R)=\sr(R/\Rad(R))$.
 \end{itemize}\end{prop} \begin{proof} See~\cite{Va71}. \end{proof}

\begin{example} Since the stable rank of a field is one, one can conclude from the previous proposition that $\sr(R)=1$ for any semilocal ring $R$. \end{example}

\subsection{Absolute stable rank}

%TODO: $\asr(R,I)$

\section{Relative parabolic factorizations} \label{sec:factorizations}
%TODO: Add references to works of Dennis, Bass, Vaserstein, Suslin etc.
In this section we formulate and prove relative versions of decompositions from~\cite{St78} which will be our main technical tools throughout the next section.

\subsection{Relative Gauss decomposition}\label{sec:gauss}
\begin{thm}\label{thm:Gauss}
Let $\Phi$ be a reduced irreducible root system of rank $\ell$ and let $\Delta_1, \Delta_\ell$ be
two reductive subsystems of $\Phi$ corresponding to the endnodes of the Dynkin diagram of $\Phi$.
Suppose that both relative elementary subgroups $\E(\Delta_i, R, I)$, $i=1,\ell$ admit triangular decomposition with $N$ triangular factors:
\[ \E(\Delta_i, R, I) = \Hh(\Delta_i, R, I) \cdot \U(\Delta^+_i, I) \cdot \U(\Delta^-_i, I) \cdot \ldots \cdot \U(\Delta^\pm_i, I),\quad i=1,\ell. \]
Then the group $\E(\Phi, R, I)$ admits the decomposition with the same number of factors:
\[ \E(\Phi, R, I) = \Hh(\Phi,R,I) \cdot \U(\Phi^+,I) \cdot \U(\Phi^-, I) \cdot \ldots \cdot \U(\Phi^\pm,I). \]
\end{thm}
\begin{proof}
Denote by $Y$ the product of subgroups in the right-hand side of the above equality.
To show that $Y=\E(\Phi,R,I)$ it suffices to check that
\begin{enumerate}
\item $Y$ is normalized by $\E(\Phi,R)$, i.\,e. $Y^{\E(\Phi,R)}\subseteq Y$;
\item there exists a set $X$ generating $\E(\Phi,R,I)$ as a \emph{normal} subgroup of $\E(\Phi,R)$ such that $XY\subseteq Y$.
\end{enumerate}
To prove the first assertion it suffices to show that $Y^{x_\alpha(\xi)} \subseteq Y$ for any $\alpha\in \pm \Pi$, $\xi\in R$.
Fix a root $\alpha\in\pm\Pi$. Clearly, $\alpha \in \Delta_i$ for $i=1$ or $i=\ell$ and we can expand $Y$ as
\[ Y=\Hh(\Phi,R,I) \cdot \U(\Delta_i^+,I) \cdot \ldots \cdot \U^\pm(\Delta_i,I) \cdot \U(\Sigma_i^+,I) \cdot \ldots \cdot \U(\Sigma_i^\pm,I). \]
For every  $h \in \Hh(\Phi, R, I)$ one has for some $s\in I$
\[ x_\alpha(\xi)\cdot h = h\cdot x_\alpha((1+s)\xi)=h\cdot x_\alpha(\xi)\,x_\alpha(s\xi). \]

Therefore by the assumption of the theorem
\begin{multline*}
Y^{x_\alpha(\xi)} \subseteq \Hh(\Phi,R,I) \cdot x_\alpha(-\xi) \X_{\alpha}(I) \cdot \E(\Delta_i,R,I)\cdot x_\alpha(\xi) \cdot \U(\Sigma^+_i,I) \cdot \ldots \cdot \U(\Sigma^\pm_i,I) =\\
= \Hh(\Phi, R, I) \cdot \E(\Delta_i,R,I) \cdot \U(\Sigma^+_i,I) \cdot \ldots \cdot \U(\Sigma^\pm_i,I) = Y.
\end{multline*}

Now set $X=\left\{x_\alpha(\xi)\ \middle|\ \alpha\in\Pi,\ \xi\in I \right\}$. 
Every root is a conjugate of some fundamental root under the action of $W(\Phi)$.
Since $\E(\Phi,R)$ contains the normalizer of the torus, the set $X^{\E(\Phi,R)}$ contains all the generators of the group $\E(\Phi, I) = \langle x_\alpha(\xi),\  \alpha\in\Phi,\ \xi\in I \rangle$.
Finally, the inclusion $XY \subseteq Y$ follows from the fact that $\Hh(\Phi,R,I)$ normalizes every root subgroup $\X_\alpha(I)$.
\end{proof}
\subsection{Relative Bass---Kolster decompositions}\label{sec:bass-kolster}
The next theorem is a relative version of the so called Bass---Kolster decomposition (cf.~\cite[Theorem~2.1]{St78}).
\begin{thm}\label{thm:BassKolster}
Let $\Phi$ be a classical root system of rank $\ell\geqslant2$, let $R$ be an arbitrary commutative ring and $I$ be an ideal, satisfying one of the following assumptions:
\[\begin{array}{l@{\quad}l@{\quad}l@{\quad}c}
\Phi = \rA_\ell,\ \ell\geqslant 2, & \sr(I) \leqslant \ell; \\
\Phi = \rC_\ell,\ \ell\geqslant 2, & \sr(I) \leqslant 2\ell-1; \\
\Phi = \rB_\ell, \rD_\ell,\ \ell\geqslant 3, & \asr(I) \leqslant \ell-1.
\end{array}\]
Then the principal congruence subgroup $\G(\Phi,R,I)$ admits the following relative version of Bass---Kolster decomposition:
\[ \G(\Phi,R,I)=  \U(\Phi^+,I) \cdot \U(\Phi^-,I) \cdot Z \cdot \U(\Sigma_1^-\setminus\{-\alpha_\mathrm{max}\},I) \cdot \U(\Sigma_1,I) \cdot \G(\Delta_1,R,I), \]
where $Z=\left\{ z_{-\alpha_\mathrm{max}}(r,1)\ \middle|\ r\in I \right\}$.
\end{thm}
\begin{proof}

Let $g$ be an element of $\G(\Phi, R, I)$. Set $v=g \cdot v^+\in\Ums(n, I)$. 
Notice that in each case it suffices to find $g' \in \U(\Phi^-, I) \cdot \U(\Phi^+, I) \cdot g$ such that 
\begin{equation} \label{eq1} (g'\cdot v^+)_{1} = 1 + s \text{ and } (g'\cdot v^+)_{\varpi\ssub{1}-\alpha\ssub{max}} = s\ \text{for some}\ s\in I. \end{equation}
Indeed, set $g'' = z_{-\alpha\ssub{max}}(-s, 1) \cdot g'$.
Obviously, one has $(g''\cdot v^+)_1 = 1$, $(g''\cdot v^+)_{\varpi\ssub{1}-\alpha\ssub{max}}=0$ and the conclusion of the theorem follows from Lemma~\ref{lemma:Chevalley-Matsumoto}.

\textsc{Case $\Phi=\rA_\ell$, $n=\ell + 1$.}
%Set $v=(1+v_1,v_2,\ldots,v_\ell,v_{\ell+1})^t\in\Ums(\ell+1,R,I)$.
Thanks to the relative stable rank condition one can add suitable multiples of the last component $v_{\ell+1}$ to the first $\ell$ components of $v$ so that the upper
$\ell$ coefficients of the resulting vector $v'$ form an $I$-unimodular column of length $\ell$.
Now multiplying $v'$ by a suitable $y\in \U(\Sigma_\ell^-, I)$ we obtain equalities~\ref{eq1}.

\textsc{Case $\Phi=\rC_\ell$, $n=2\ell$.}
Notice that column $(v_1,\ldots, v_{-2}, v_{-1}^2)^t$ is also $I$-unimodular.
Applying condition $\sr(I)\leq 2\ell-1$ we find $c_1, c_2, \ldots, c_{-2} \in I \cdot v_{-1}$ such that upper $2\ell -1$ components of $v'=(v_1 + c_1 v_{-1}, \ldots, v_{-2} + c_{-2}v_{-1}, v_{-1})^t$ form an $I$-unimodular column.
By the choice of $c_i$ we can find suitable $d\in I$ such that $h_1 \cdot v = v'$ for
\[ h_1 = x_{1,-1}(c_1 + d) \cdot \prod_{i=2}^{-2} x_{i,-1}(c_i) \in \U(\Sigma_1^-, I). \]

We can find $f_1, f_2,\ldots, f_{-2} \in R$ such that $f_1v'_1+\sum_{i=2}^{-2} f_i v'_i = 1$.
%TODO: Determine exact sign
Set $\xi = v''_1-v''_{-1}-1 \in I$,
\[ h_2 = x_{-1,1}\biggl(\xi f_1 \pm \sum_{i=2}^\ell v_1' \xi^2 f_i f_{-i}\biggr) \cdot \prod_{i=2}^{-2} x_{-1,i}(\xi f_i) \in \U(\Sigma_1, I). \]
Direct computation shows that $v'' = h_2 \cdot v'$ satisfies equalities~\ref{eq1}.

%Now we can assume that the first $2\ell-1$ entries of $v$ are unimodular and find $c_1,\ldots,c_{-2}\in I$ such that $c_1v_1+\ldots+c_{-2}v_{-2}=(v_1-1)-v_{-1}$. Add $c_{-i}v_i$ to $v_{-1}$, $i=2,\ldots,\ell$:
%\[ (v_1,v_2\ldots,v_\ell,v_{-\ell},\ldots,v_{-2},v_{-1})^t\longmapsto (v_1,v'_2\ldots,v'_\ell,v_{-\ell},\ldots,v_{-2},v'_{-1})^t. \]
%Then add $c_iv_i'=c_i(v_1+c_{-i}v_{-i})$, $i=2,\ldots,\ell$ to the last entry:
%\[ v_1\longmapsto v_1,\quad v'_{-1}\longmapsto v_{-1}+\sum_{i=2}^\ell c_{-i}v_{-i}+\sum_{i=2}^\ell c_i(v_i+c_{-i}v_1)=v''_{-1}. \]
%Next add $\left(c_1-\sum_{i=2}^\ell c_ic_{-i}\right)v_1$ to $v''_{-1}$ to get $v_1-1$ in position $-1$.
%Again, as in case of $\rA_n$, apply $z_\gamma(1-v_1,1)$ with $\gamma=-\alpha_\mathrm{max}$, to get $1$ as the first entry and $0$ as the last.

\textsc{Case $\Phi=\rD_\ell$, $n= 2\ell$.} 
By Lemma~\ref{lemma:asrUnip} we can find $h_1\in \U(\Sigma^+_\ell, I)$ such that the upper half $v'_+$ of $v'=h_1 \cdot v$ is $I$-unimodular.
Since $\sr(I)\leq \ell-1$ we can find $c_1$, $c_3, \ldots c_\ell \in I$ such that $(v''_1, v''_3, \ldots, v''_\ell) \in \Ums(\ell-1, I)$, where
\[ v''=h_2 \cdot x_{1,2}(c_1) \cdot v', \quad h_2=\prod_{i=3}^\ell x_{i,2}(c_i). \]

We can find $f_1, f_3,\ldots, f_\ell \in R$ such that $f_1v''_1+\sum_{i=3}^\ell f_i v''_{i} = 1$.
As before, set
\[ \xi = v''_1-v''_{-2}-1 \in I, \quad h_3 = x_{-2,1}(\xi f_1) \cdot \prod_{i=3}^\ell x_{-2,i}(\xi f_i), \quad v'''=h_3 \cdot v''. \]
%Clearly, $v'''_{-2}=v'''_1-1$, therefore for $v_4 = z_{-\alpha_{max}}(-v'''_{-2}, 1) \cdot v'''$ one has $v^4_1 = 1$, as required.
Clearly, $t_{1,2}(c_1) \cdot h_1 \in \U(\Phi^+, I)$, $ h_3 \cdot h_2 \in \U(\Phi^-, I)$ and $v'''$ satisfies \ref{eq1}.

\textsc{Case $\Phi=\rB_\ell$, $n=2\ell+1$.} Subdivide $v\in \Ums(2\ell+1, I)$ as $v=(v_+, v_0, v_-)\in R^\ell\times R\times R^\ell$.
Denote by $J\leq I$ the ideal spanned by components of $v_-$.
Since $\sr(I/J)\leq \ell$ we can find $c_1,\dots,c_\ell\in I$ such that for $v' = h \cdot v$, $h = \prod_{i=1}^\ell x_{i,0}(c_i) \in \U(\Phi^+, I)$
one has $\bar{v'}_+=(\bar{v'_1},\ldots, \bar{v'_\ell}) \in \Ums(\ell, I/J)$ and, therefore, $(v'_+, v'_-) \in \Ums(2\ell, I)$.
Now the proof can be finished by repeating the argument for the case $\Phi=\rD_\ell$ (applied to the subset of long roots of $\rB_\ell$).
%(clearly, the maximal root of $\rD_\ell$ maps to the maximal root of $\rB_\ell$ under the natural embedding $\rD_\ell\subseteq\rB_\ell$). 
\end{proof}

It is easy to see that the proof of the above theorem is effective and gives an estimate of the total number of elementary root unipotents involved in the decomposition.
\begin{cor}
In the assumptions and notation of Theorem~\ref{thm:BassKolster} every element of $\G(\Phi,R,I)$ 
can be factored into a product of one element of $\G(\Delta_1,R,I)$ one element of $Z$ and at most $4(|\Phi^+| - |\Delta_1^+|)-1$ elementary root unipotents $x_\alpha(s)$ of level $I$. \end{cor}
\begin{proof}
The above estimates can be obtained by a careful analysis of the proof of the previous theorem.
Cases $\Phi=\rA_\ell, \rC_\ell$ are immediate.
In the case $\Phi=\rD_\ell$ the proof of Theorem~\ref{thm:BassKolster} implies that
\begin{multline}\nonumber
\G(\Phi,R,I) =  \U(\Sigma_\ell,I) \cdot X_{\alpha_1}(I) \cdot \U(\Sigma_2^-\cap\Delta_1,I) \cdot X_{-\alpha\ssub{\mathrm{max}}}(I) \cdot Z  \cdot \\ \cdot \U(\Sigma_1^-,I) \cdot \U(\Sigma_1,I) \cdot \G(\Delta_1,R,I).
\end{multline}
We can present an element $g$ of $\U(\Sigma_\ell, I)$ as a product of $g_1 \in \U(\Sigma_{\{1,2\}} \cap \Sigma_\ell)$ and $g_2\in \U(\Delta_{\{1,2\}}\cap \Sigma_\ell)$.
An examination of the extended Dynkin diagram of $\rD_\ell$ implies that $g_2$ either centralizes or normalizes all factors of the above decomposition (except the last one) and therefore can be moved to the right until it is consumed by $\G(\Delta_1)$.
On the other hand, $g_1$ is a product of at most $2\ell-3$ elementary unipotents, while the width of $\U(\Sigma_1^\pm, I)$ and $\U(\Sigma_2^-\cap\Delta_1)$ in elementary unipotents does not exceed $2\ell-2$ and $2\ell-4$, respectively.
Summing up these upper bounds we obtain
$$(2\ell-3) + 1 + (2\ell - 4) + 1 + 2\cdot (2\ell - 2) = 8\ell - 9 = 4(|\rD_\ell| - |\rD_{\ell-1}|) - 1.$$

The estimate in the case $\Phi=\rB_\ell$ can be obtained in a similar way. \end{proof}

\begin{proof}[Proof of Theorem~\ref{thm:SL2width}]
Consider the case $\Phi=\rA_\ell$, $\sr(I)\leq 2$. First of all, notice that we can improve the estimate of the number of factors involved in Bass---Kolster decomposition.
Indeed, when performing the first step of the proof of Theorem~\ref{thm:BassKolster} it suffices to make only $2$ additions of $v_{\ell+1}$ (e.\,g. to $v_{1}$ and $v_2$) to make the first $\ell$ entries of $v'$ form a unimodular column.
In particular, $\G(\Phi, R, I)$ can be presented as a product of one element of $\G(\Delta_1, R, I)$, one element of $Z$ and $3\ell+1$ root unipotents $x_\alpha(s)$, $s\in I$.
The latter elements are contained in a product of $3\ell + 1$ copies of $\SL(2, R, I)$. Now the statement of the theorem follows by induction on $\ell$.

The proof in the case $\Phi=\rC_\ell$ is similar (notice that we use the exceptional isomorphism $\SL(2, R)\cong \Sp(2, R)$).
\end{proof}

\subsection{Relative Dennis---Vaserstein decompositions}\label{sec:dennis-vaserstein}
Throughout the present section we denote by $\EP_s(R, I)$ the subgroup $\E(\Delta_s, R, I) \cdot \U(\Sigma_s, I)$, $1 \leq s \leq n$.
%TODO: Add Levi decomposition to preliminaries
%TODO: Introduce notation for U(S, R)
Set $\EP_s := \EP_s(R, R) = \E(\Delta_s, R) \cdot \U(\Sigma_s, R)$. 

Let $\Phi$ be an irreducible root system of rank $\ell$.
Let $r$, $s$ be two distinct indices $1\leq r,s \leq \ell$.
From Levi decomposition it follows that
\begin{multline}\nonumber \U(\Phi^+, I)\cdot \U(\Phi^-, I) \cdot \E(\Delta_r, R, I) \cdot \EP_s(R, I) = 
\U(\Sigma_r, I)\cdot \U(\Sigma^-_r, I) \cdot \E(\Delta_r, R, I) \cdot \EP_s(R, I) = \\
= \EP_r(R, I) \cdot \E(\Delta_s, R, I) \cdot \U(\Sigma_s^-, I)\cdot \U(\Sigma_s, I) = 
\EP_r(R, I) \cdot \U(\Sigma^-_r\cap \Sigma^-_s, I) \cdot \EP_s(R, I). \end{multline}
Denote by $A_{rs}$ any of the equal subsets from the previous formula. 

\begin{thm}\label{theorem:relative_dv}
The relative elementary subgroup $\E(\Phi, R, I)$ coincides with $A_{rs}$ under the following assumptions on $(R, I)$.
  \begin{center}
    \begin{tabular}{| l | l | l | l | l |} \hline
    № & $\Phi$ & $(s,r)$ & ring condition \\ \hline
    1. & $\rA_\ell$, $\ell\geq 2$ & $(1, \ell)$ & $\sr(I) \leq \ell-1$ \\ \hline
    2. & $\rB_\ell$, $\ell\geq 3$ & $(1, \ell)$ & $\sr(I) \leq \ell-1$ \\ \hline
    3. & $\rD_\ell$, $\ell\geq 4$ & $(1, \ell)$ & $\sr(I) \leq \ell-2$ \\ \hline    
    4. & $\rE_\ell$, $\ell=6,7$ & $(\ell, 2)$ & $\sr(I) \leq \ell-3$ \\ \hline     
    5. & $\rE_\ell$, $\ell=6,7$ & $(\ell, 1)$ & $\asr(R, I)\leq \ell-2$ \\ \hline    
    \end{tabular} \end{center} 
\end{thm}
The proof of Theorem~\ref{theorem:relative_dv} occupies the rest of this section.

Consider the usual conjugation action of $\E(\Phi, R)$ on $\E(\Phi, R, I)$. 
This action induces an action of $\E(\Phi, R)$ on the set $\mathfrak{S}$ of all subsets of $\E(\Phi, R, I)$.
On the other hand, $\E(\Phi, R, I)$ acts on $\mathfrak{S}$ by left multiplication.
Denote by $N_{rs}$ and $L_{rs}$ stabilizers of $A_{rs} \in \mathfrak{S}$ with respect to these actions.
In other words $$N_{rs} = \{ g\in \E(\Phi, R) \mid g \cdot A_{rs} \cdot g^{-1} \subseteq A_{rs} \};\quad L_{rs}= \{ g\in \E(\Phi, R, I) \mid g \cdot A_{rs} \subseteq A_{rs} \}.$$

It is easy to see that $N_{rs}$ normalizes $L_{rs}$. Indeed, for $g\in N_{rs}$, $h\in L_{rs}$ one has
$$h^g \cdot A_{rs} = g^{-1} \cdot h \cdot g \cdot A_{rs} \subseteq g^{-1} \cdot h \cdot A_{rs} \cdot g \subseteq A_{rs}^g \subseteq A_{rs}.$$

\begin{lemma}\label{lemma:dv_unipotent} For any $1\leq i\leq n$ the following statements hold. \begin{enumerate} 
\item $\U(\Phi^\pm, I) = X_{\pm\alpha\ssub{i}}(I)\cdot \U(\Phi^\pm\setminus\{\pm\alpha\ssub{i}\}, I) = \U(\Phi^\pm\setminus\{\pm\alpha\ssub{i}\}, I)\cdot X_{\pm\alpha\ssub{i}}(I).$
\item For any $\xi\in R$ one has $\U(\Phi^\pm\setminus\{\alpha_i\}, I)^{x_{\mp\alpha\ssub{i}}(\xi)^{-1}} \subseteq \U(\Phi^\pm, I).$
\item $\U(\Phi^+, I)\cdot \U(\Phi^-, I) \subseteq \U(\Phi^+\setminus \{\alpha_i\}, I) \cdot \U(\Phi^-, I) \cdot X_{\alpha\ssub{i}}(I) \cdot X_{-\alpha\ssub{i}}(I)$.
\end{enumerate} \end{lemma}
\begin{proof}
 The first two statements easily follow from Chevalley commutator formula while the third one is a formal consequence of the first two.
\end{proof}

The following lemma is a relative version of the main reduction used by M.~Stein in~\cite{St78} for the proof of the absolute Dennis--Vaserstein decomposition.
\begin{lemma}\label{lemma:Stein_reduction}
Assume that there exists a subset $\widetilde{L_r} \subseteq \E(\Delta_r, R, I)$ with the following properties:
\begin{enumerate}[label=(\alph*)] 
 \item\label{stein_cond1} One has $\U(\Sigma^+_r, I)\cdot \U(\Sigma^-_r, I) \cdot \E(\Delta_r, R, I) \subseteq \U(\Phi^+, I)\cdot \U(\Phi^-, I) \cdot \widetilde{L_r}.$
 \item\label{stein_cond2} One has $X_{-\alpha\ssub{r}}(I)^{\widetilde{L_r}} \subseteq \EP_s(R, I).$
\end{enumerate}
Then $X_{-\alpha_r}(I) \subseteq L_{rs}$ and $\EP_s \subseteq N_{rs}.$
\end{lemma}
\begin{proof} Set $A:=\U(\Phi^+, I)\cdot \U(\Phi^-, I) \cdot \widetilde{L_r} \cdot \EP_s(\Phi, R, I).$
From condition~\ref{stein_cond1} of the lemma it follows that $A_{rs}=A$.
Observe that from Lemma~\ref{lemma:dv_unipotent} and condition~\ref{stein_cond2} it follows that
\begin{multline}\nonumber 
A \subseteq \U(\Phi^+\setminus \{\alpha_r\}, I) \cdot \U(\Phi^-, I) \cdot X_{\alpha\ssub{r}}(I) \cdot X_{-\alpha\ssub{r}}(I) \cdot \widetilde{L_r} \cdot \EP_s(R, I) \subseteq \\ 
\subseteq \U(\Phi^+\setminus\{\alpha_r\}, I) \cdot \U(\Phi^-, I) \cdot X_{\alpha\ssub{r}}(I) \cdot X_{-\alpha\ssub{r}}(I) \cdot \widetilde{L_r} \cdot \EP_s(R, I) \subseteq \\
\subseteq \U(\Phi^+\setminus\{\alpha_r\}, I) \cdot \U(\Phi^-, I) \cdot \widetilde{L_r} \cdot \U(\Sigma_r, I) \cdot \EP_s(R, I)  \cdot \EP_s(R, I) \subseteq \\
\subseteq \U(\Phi^+\setminus\{\alpha_r\}, I) \cdot \U(\Phi^-, I) \cdot \widetilde{L_r} \cdot \EP_s(R, I). \end{multline}
Applying Lemma~\ref{lemma:dv_unipotent} we get that:
\begin{equation}\nonumber A^{X_{-\alpha\ssub{r}}} \subseteq \U(\Phi^+, I) \cdot \U(\Phi^-, I) \cdot \widetilde{L_r} ^{X_{-\alpha\ssub{r}}} \cdot \EP_s(R, I) \subseteq A. \end{equation}
\begin{equation}\nonumber X_{-\alpha\ssub{r}}(I) \cdot A \subseteq \U(\Phi^+, I) \cdot X_{-\alpha\ssub{r}}(I) \cdot \U(\Phi^-, I) \cdot \widetilde{L_r} \cdot \EP_s(R, I) = A. \end{equation}
To prove the second part of the statement observe first that $\EP_s$ is generated by $X_{\alpha\ssub{i}}$ for $1\leq i\leq n$ and $X_{-\alpha\ssub{i}}$ for $i\neq s$.
%TODO: Find proper reference for this fact
We have just shown that $X_{-\alpha_r}\subseteq N_{rs}$.
On the other hand, inclusions $X_{\alpha\ssub{k}} \subseteq N_{rs}$ for $\ 1\leq k\leq \ell$ and $X_{-\alpha\ssub{k}} \subseteq N_{rs}$ for $k\neq r,s$ are obvious.
\end{proof}

\begin{proof}[Proof of Theorem~\ref{theorem:relative_dv}]
We first show that under specified assumptions on $(R, I)$ one can meet the conditions of Lemma~\ref{lemma:Stein_reduction}.
Consider the following two subsets of $\Lambda(\pi)$:
$$\Gamma = \varpi_s- (\Sigma_s^+\cap \Delta_r),\quad \Gamma_0 = \{\lambda \in \Gamma \mid \lambda - \alpha_r \in \Lambda(\pi) \}.$$
Clearly, $\Gamma$ is the set of weights of an irreducible representation of $\G(\Delta_r, R)$ corresponding to the same highest weight $\varpi_s$.
The subsystem $\Delta_r$ has type $\rA_{\ell-1}$ in all cases except the last one.
It is also clear that $|\Gamma_0|=1$ for $\Phi=\rA_\ell, \rB_\ell$, $|\Gamma_0|=2$ for $\Phi=\rD_\ell$, $|\Gamma_0|=3$ for $\Phi=\rE_\ell$ and $r=2$.
In the case $\Phi=\rE_\ell$, $r=2$ the subsystem $\Delta_r$ has type $\rD_{\ell-1}$ and $|\Gamma_0|=\ell-1$.

Let $\widetilde{L_r}$ be the set of all elements $g$ of $\E(\Delta_r,R, I)$ such that $(g \cdot v^+)_\lambda = 0$ for $\lambda\in\Gamma_0$.
In any of specified cases the assumption on $(R, I)$ allows us to apply Lemma~\ref{lemma:uraction} to the subsystem $\Delta_r$ and find
$x\in\U(\Delta_r\cap\Phi^+, I)$, $y\in \U(\Delta_r\cap\Phi^-, I)$ such that $yx\cdot g \in \widetilde{L_r}$.
This proves the first condition of Lemma~\ref{lemma:Stein_reduction}, indeed:
$$ \U(\Sigma^+_r, I)\cdot \U(\Sigma^-_r, I) \cdot g = \U(\Sigma^+_r, I) x^{-1} \cdot \U(\Sigma^-_r, I)^{x^{-1}} y^{-1} \cdot (yxg) \subseteq \U(\Phi^+, I)\cdot \U(\Phi^-, I) \cdot \widetilde{L_r}.$$
To prove the second condition notice that by the definition of $\Gamma_0$ for any $s\in I$, $ g\in\widetilde{L_r}$ one has $x_{-\alpha_r}(s) \cdot g \cdot v^+ = g \cdot v^+$ and, therefore,
$$X_{-\alpha\ssub{r}}(I)^{\widetilde{L_r}} \subseteq \U(\Phi^-, I) \cap \Stab(v^+) \subseteq \E(\Delta_s, R, I) \subseteq \EP_s(R, I).$$

From now on we can assume that the statement of Lemma~\ref{lemma:Stein_reduction} holds and $\EP_s$ normalizes $A_{rs}$.
Notice that in view of Theorem~\ref{theorem:Stepanov} it suffices to show that the following two families of elements are contained in $L_{rs}$:
\begin{itemize} \item $z_{\alpha}(s, \xi)$, $s\in I$, $\xi \in R$, $\alpha\in\Sigma^-_s$;
\item $x_{\beta}(s)$, $s \in I$, $\beta \in \Phi$. \end{itemize}
Since $\EP_s \subseteq N_{rs}$ it suffices to check inclusions only for the second family of elements.
We already know that $\U^+(\Phi, I) \subseteq L_{rs}$.

Notice that the Weyl group $W(\Delta_s)$ acts transitively on $\Delta_s$, therefore in view of relation~\ref{rel:R3} the subgroup
$W_s := \langle w_\alpha(1) \mid \alpha\in\Delta_s\rangle \leq \EP_s$ acts transitively on the set of root subgroups $X_\alpha(I)$, $\alpha\in \Delta_s$.
Since $X_{-\alpha\ssub{r}}(I) \subseteq L_{rs}$ we get that that $\U(\Delta_s \cap \Phi^-, I)\subseteq L_{rs}$.

Now denote by $\widetilde{\alpha}$ the maximal root of $\Phi$. Our assumptions on $\Phi$ guarantee that $m_s(\widetilde{\alpha})=1$, 
and, consequently, every two roots $\alpha, \beta \in \Sigma^-_s$ have the same $s$-shape (i.\,e. $\shape(\{s\}, \alpha = \shape(\{s\}, \beta)$).
By Lemma~\ref{lemma:abs} $W(\Delta_s)$ interchanges $\alpha$ and $\beta$ if their length is equal (which is the case if we assume additionally $\Phi\neq \rB_\ell$).
Since $X_{\alpha\ssub{s}} \subseteq L_{rs}$ the argument similar to the one above implies $\U(\Sigma^-_s, I)\subseteq L_{rs}$.
This completes the proof of the theorem for $\Phi\neq \rB_\ell$. 
%TODO:
\textbf{TODO: Finish the proof for $\Phi=\rB_\ell$.}
\end{proof}


\section{Applications} \label{sec:applications}
\subsection{Bounded generation}\label{sec:boundgen}

\subsection{Relative stable rank 1}
\begin{lemma}\label{lemma:srRI1}
If $sr(R,I)=1$, the width of $SL(2,R,I)$ with respect to $z_\alpha$ does not exceed $4$.
\end{lemma}
\begin{proof}

Let $A=\begin{pmatrix}a & b \\ c & d\end{pmatrix}\in SL(2,R,I)$. First column is $I$-unimodular, therefore there exists $z\in I$ such that $a+cz\in R^*$. Multiply $A$ by $x_{12}(z)$ from left to obtain invertible element in the upper left corner. Applying $x_{21}(-c/a)$ from left and $x_{12}(-b/a)$ from right, we obtain a diagonal matrix. Thus
\begin{multline*}
A=x_{12}(-z)\cdot x_{21}(c/a)\cdot
\begin{pmatrix} \varepsilon & 0 \\ 0 & 1/\varepsilon \end{pmatrix}
\cdot x_{12}(b/a)=\\
=x_{12}(-z)\cdot
\begin{pmatrix} \varepsilon & 0 \\ 0 & 1/\varepsilon \end{pmatrix}
\cdot x_{21}(y) \cdot x_{12}(b/a),
\end{multline*}
where all the coefficients in transvections lie in $I$ and $\varepsilon\in 1+I$. The above formula is a relative version of Gauss decomposition.
Note that
\begin{multline*}
\begin{pmatrix} \varepsilon & 0 \\ 0 & 1/\varepsilon \end{pmatrix} =
\begin{pmatrix} 1 & -1 \\ 0 & 1 \end{pmatrix}
\begin{pmatrix} 1 & 0 \\ 1-\varepsilon & 1 \end{pmatrix}
\begin{pmatrix} 1 & 1/\varepsilon \\ 0 & 1 \end{pmatrix}
\begin{pmatrix} 1 & 0 \\ \varepsilon^2-\varepsilon & 1 \end{pmatrix} =\\=
\begin{pmatrix} 1 & -1 \\ 0 & 1 \end{pmatrix}
\begin{pmatrix} 1 & 0 \\ 1-\varepsilon & 1 \end{pmatrix}
\begin{pmatrix} 1 & 1+z \\ 0 & 1 \end{pmatrix}
\begin{pmatrix} 1 & 1/\varepsilon-1-z \\ 0 & 1 \end{pmatrix}
\begin{pmatrix} 1 & 0 \\ \varepsilon^2-\varepsilon & 1 \end{pmatrix},
\end{multline*}
therefore
\[
A=z_{21}(1-\varepsilon,-1-z)\cdot x_{12}(1/\varepsilon-1-z)\cdot x_{21}(\varepsilon^2-\varepsilon+y)\cdot x_{12}(b/a). \qedhere
\]
\end{proof}

\subsection{Dedekind rings of arithmetic type}
\subsubsection{Number field case}
\paragraph{$\mathbb{Z}[\sfrac{1}{p}]$.}
\textbf{TODO:} Insert here Moree's lemma on primitive roots
\begin{lemma}\label{lemma:Z1p}
Assume GRH. If $R=\mathbb{Z}[\sfrac{1}{p}]$ and $I\lhd R$, the width of $SL(2,R,I)$ does not exceed $6$.
\end{lemma}
\begin{proof}
$I\lhd\mathbb{Z}[\sfrac{1}{p}]$, $I=(m)$, $m\in\mathbb{Z}$, $p\nmid m$.

\[ g=\begin{pmatrix}
x & y \\ z & w
\end{pmatrix},\quad x =p^\alpha a,\quad y =p^\beta bm, \]
where $a,b\in\mathbb{Z}$, $p\nmid a,b$ and $\alpha,\beta\in\mathbb{Z}$.

\textsc{Case 1:} $\alpha\geqslant\beta$. Since $p^{\alpha-\beta}a\perp bm^2$ and $p\perp bm^2$, there exist infinitely many rational primes $q$ of the form $p^{\alpha-\beta}a+bm^2k$, such that $p$ is a primitive root modulo $q$. One may choose $q$ prime to $b$. Then
\[ g_1=g\cdot x_{21}(mk) =
\begin{pmatrix} p^\beta q & p^\beta bm \\ * & * \end{pmatrix}.\]
$\exists u\geqslant 1: p^u\equiv b\mod q$, say $p^u=b+lq$. Then
\[ g_2 = g_1\cdot x_{12}(ml) =
\begin{pmatrix} p^\beta q & mp^{\beta+u} \\ * & * \end{pmatrix}. \]
$g_2\equiv 1\mod m$, thus $p^\beta q=1+cm$.
\begin{align*}
g_3 = & g_2\cdot x_{21}\left(\dfrac{-c}{p^{\beta+u}}\right) =
\begin{pmatrix} 1 & mp^{\beta+u} \\ * & * \end{pmatrix}, \\
g_4 = & g_3\cdot x_{12}(-mp^{\beta+u}) =
\begin{pmatrix} 1 & 0 \\ * & * \end{pmatrix}, \\
g_5 = & g_4\cdot x_{21}\left(\dfrac{c}{p^{\beta+u}}\right) =
\begin{pmatrix} 1 & 0 \\ * & * \end{pmatrix}.
\end{align*}
Note that $g_5=g_2\cdot z_{12}\left(-mp^{\beta+u},\dfrac{c}{p^{\beta+u}}\right)$ and that two other coefficients in transvections are multiples of $m$. Thus in this case the length of $g$ does not exceed $4$: $g=x_{21}z_{12}x_{12}x_{21}$.

\textsc{Case 2:} $\alpha<\beta$. Since $\mathbb{Z}[\sfrac{1}{p}]/I$ is finite, $\exists k>0$ such that $p^k\equiv 1\mod I$. One can choose $k>\beta-\alpha$. Then $k+\alpha>-k+\beta$ and
\[ g\cdot h(p^k) =
\begin{pmatrix} p^\alpha a & p^\beta bm \\ * & * \end{pmatrix}
\begin{pmatrix} p^k & 0 \\ 0 & p^{-k} \end{pmatrix}=
\begin{pmatrix} p^{k+\alpha} a & p^{-k+\beta} bm \\ * & * \end{pmatrix}.
\]
One can apply Case 1 fot the latter, so
$g=x_{21}z_{12}x_{12}x_{21}h^{-1}$,
and $x_{12}h^{-1}$ can be processed in the same way, as in Lemma \ref{lemma:srRI1}:
$g=x_{21}z_{12}x_{12}\cdot z_{12}x_{21}x_{12}$.
\end{proof}
\subsection{Subsystem factorizations}\label{sec:subsysfact}
The main result of \cite{NikProdDecomp} is the following
\begin{thm*}
Let $G$ be a classical (possibly twisted) Chevalley group of rank $n$ over a finite field. Then $G$ equals the product of at most $200$ conjugates of an $\SL_n$ subgroup.
\end{thm*}
As indicated in the introduction, the Dennis---Vaserstein decomposition gives the following result.
\begin{lemma}
Assume $\sr(I)\leqslant n-1$. Then $\SL(n+1,R,I)$ is a product of at most $5$ subgroups isomorphic to $\SL(n,R,I)$.
\end{lemma}
\begin{proof}
By the Dennis---Vaserstein decomposition one can present $\SL(n+1,R,I)$ as a product
\[ \SL(n+1,R,I) =  \Par_1\cdot X_{n1}\cdot\Par_n=S_1\U_1\cdot X_{n1}\cdot\U_n S_n, \]
where $S_1$ and $S_n$ are two obvious embeddings of $\SL(n,R,I)$ in $\SL(n+1,R,I)$, avoiding respectively the first and the last row and column. Now $\U_1=(\U_1\cap S_n)\cdot X_{1n}$ and $U_n= X_{1n}\cdot(U_n\cap S_1)$, while $X_{1n}X_{n1}X_{1n}\in S_1^{w_{12}(1)}$.
\end{proof}
We will now elaborate on the case $\Phi=\rD_\ell$ to show that the Dennis---Vaserstein decomposition is suitable for handling other Chevalley groups, albeit with a stronger assumption on the base ring. The following lemma is an analogue of Proposition~1 of~\cite{NikProdDecomp}.
\begin{lemma}
Let $\Phi=\rD_\ell$. There exist an element $y\in\G(\Phi,R)$ and an element $w\in\widetilde{W}(\Phi)$ such that $\U(\Sigma_\ell)\subset[\U(\Delta_\ell^-),y]\cdot{}^w\U(\Delta_\ell^+)$.
\end{lemma}
\begin{proof}
Set $\beta_i = \alpha_{2i-1} + 2\alpha_{2i}+ \ldots + 2\alpha_{\ell-2} + \alpha_{\ell-1} + \alpha_\ell$, for $i=1,\ldots,\lfloor\ell/2\rfloor$, i.\,e. $\beta_i$ form a maximal set of pairwise orthogonal maximal roots in some subsystems of type $\rD_{k}$. Denote $B=\{\beta_i\}$, then decompose $\U(\Sigma_\ell)=\U(\Sigma_\ell\setminus B)\cdot\U(B)$.

Set $y=\prod_{\beta\in B}x_\beta(1)$. We will now show that $\U(\Sigma_\ell\setminus B)\subset[\U(\Delta_\ell^-),y]\cdot\U(B)$.

We first note that
\[ \bigl[\U(\Delta_2^-),x_{\beta_1}(1)\bigr]=1,\quad \Bigl[\U(\Sigma_2^-\cap\Delta_\ell),\prod_{i\neq1}x_{\beta_i}(1)\Bigr]=1. \]
This implies
\[ \Bigl[ \U(\Sigma_2^-\cap\Delta_\ell)\cdot\U(\Delta_{2,\ell}^-), x_{\beta_1}(1)\cdot\prod_{i\neq1}x_{\beta_i}(1) \Bigr] \equiv \bigl[ \U(\Sigma_2^-\cap\Delta_\ell), x_{\beta_1}(1) \bigr] \bmod \U(\Sigma_\ell\cap\Delta_2). \]
Take an element $u\in\U(\Sigma_2^-\cap\Delta_\ell)$ and decompose it as $u=vw$, $v\in\U(\Sigma_{1,2}^-\cap\Delta_\ell)$, $w\in\U(\Sigma_2^-\cap\Delta_{1,\ell})$, then, using the identity $[ab,c]={}^a[b,c]\cdot[a,c]$, rewrite
\[ [vw,x_{\beta_1}(1)] = {}^v[w,x_{\beta_1}(1)]\cdot[v,x_{\beta_1}(1)].  \]
Since $\U(\Sigma_{1,2}^-\cap\Delta_\ell)$ and $\U(\Sigma_2^-\cap\Delta_{1,\ell})$ are abelian, it is easy to see that
\[ [v,x_{\beta_1}(1)]\in\U(\Sigma_{2,\ell}\cap\Delta_1), \quad [w,x_{\beta_1}(1)]\in\U(\Sigma_{1,\ell}\setminus\{\beta_1\}), \]
and varying $v$ and $w$ one can obtain any element of $\U(\Sigma_{2,\ell}\cap\Delta_1)$ and $\U(\Sigma_{1,\ell}\setminus\{\beta_1\})$ respectively. By the Levi decomposition
\[ {}^v\U(\Sigma_{1,\ell}\setminus\{\beta_1\}) \equiv \U(\Sigma_{1,\ell}\setminus\{\beta_1\}) \bmod \U(\Sigma_\ell\cap\Delta_2). \]
Thus we have shown that
\[ [\U(\Sigma_2^-)\cdot\U(\Delta_{2,\ell}),y] \equiv \U(\Sigma_{1,\ell}\cup\Sigma_{2,\ell}\setminus\{\beta_1\}) \bmod \U(\Sigma_\ell\cap\Delta_2), \]
and this reduces the problem to the $\rD_{\ell-2}$-subsystem $\Delta_{1,2}$. So we can carry the induction on the rank of the subsystem, and construct for any given $a\in\U(\Sigma_\ell\setminus B)$ an element $b\in\U(\Delta_\ell^-)$ such that $a\in[b,y]\cdot\prod_{\beta\in B}X_\beta\subset[\U(\Delta_\ell^-),y]\cdot\U(B)$.
\end{proof}

\printbibliography

\end{document}
