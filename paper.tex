\documentclass[12pt]{amsart}
\usepackage{amscd, amsmath, verbatim, enumitem, graphicx, amssymb, mathtools, xfrac}
\usepackage[utf8]{inputenc}
\usepackage{hyperref}
\usepackage[backend=biber, citestyle=numeric-comp, natbib=true, sortlocale=en_US, url=false, doi=false, eprint=true, maxbibnames=4]{biblatex}            
\usepackage[all]{xy}

\renewbibmacro{in:}{\ifentrytype{article}{}{\printtext{\bibstring{in}\intitlepunct}}}
\newbibmacro{string+doi}[1]{\iffieldundef{doi}{\iffieldundef{url}{#1}{\href{\thefield{url}}{#1}}}{\href{http://dx.doi.org/\thefield{doi}}{#1}}}
\DeclareFieldFormat{title}{\usebibmacro{string+doi}{\mkbibemph{#1}}}
\DeclareFieldFormat[article]{title}{\usebibmacro{string+doi}{\mkbibquote{#1}}}
\DeclareFieldFormat[inproceedings]{title}{\usebibmacro{string+doi}{\mkbibquote{#1}}}
\addbibresource{paper.bib}

\textwidth 16cm 
\textheight 22cm 
\headheight 0.5cm 
\evensidemargin 0.3cm 
\oddsidemargin 0.2cm

\numberwithin{equation}{section}
\newcounter{thmcounter} \newcounter{lemmacounter}
\newtheorem{thm}[thmcounter]{Theorem}
\newtheorem{prop}[thmcounter]{Proposition}
\newtheorem{cor}[thmcounter]{Corollary}
\newtheorem{lemma}[lemmacounter]{Lemma}
\theoremstyle{definition}
\newtheorem{rem}[equation]{Remark}
\newtheorem{example}[equation]{Example}
\newtheorem{dfn}[equation]{Definition}
\newtheorem{notation}[equation]{Notation}

\DeclareMathOperator{\K}{K}
\DeclareMathOperator{\G}{G}
\DeclareMathOperator{\St}{St}
\DeclareMathOperator{\E}{E}
\DeclareMathOperator{\EP}{EP}
\DeclareMathOperator{\U}{U}
\DeclareMathOperator{\X}{X}
\DeclareMathOperator{\Z}{Z}
\DeclareMathOperator{\M}{M}
\DeclareMathOperator{\sr}{sr}
\DeclareMathOperator{\asr}{asr}
\DeclareMathOperator{\Um}{Um}
\DeclareMathOperator{\rk}{rk}
\newcommand{\rA}{\mathsf{A}}
\newcommand{\rB}{\mathsf{B}}
\newcommand{\rC}{\mathsf{C}} 
\newcommand{\rD}{\mathsf{D}} 
\newcommand{\rE}{\mathsf{E}}

\def\ssub#1{\mathchoice
   {_{\lower2pt\hbox{$\scriptstyle #1$}}}
   {_{\lower2pt\hbox{$\scriptstyle #1$}}}
   {_{\lower1.5pt\hbox{$\scriptscriptstyle #1$}}}
   {_{\!\lower1.5pt\hbox{$\scriptscriptstyle #1$}}}}

\title{Relative parabolic factorizations of Chevalley groups}
\keywords {Chevalley groups, relative subgroups, stability for $\K_1$. {\em Mathematical Subject Classification (2010):} 20G35, 19B14}
\author {Sergey Sinchuk, Andrei Smolensky}
\email {sinchukss {\it at} yandex.ru,\ andrei.smolensky {\it at} gmail.com}
\date {\today}

\begin{document}

\begin{abstract} \end{abstract}

\maketitle

\section {Introduction}\label{sec:intro}
 
\subsection{Acknowledgements}
Authors of the present paper acknowledge financial support from Russian Science Foundation grant 14-11-00297.

\section {Preliminaries}\label{sec:prelim}
For any collection of subsets $H_1,\ldots, H_n$ of a group $G$ we denote by $H_1\ldots H_n$ their Minkowski set-product,
i.\,e. the set consisting of arbitrary products $h_1\ldots h_n$ of elements $h_i\in H_i$. In particular, the equality
$G = H_1\cdot\ldots\cdot H_n$ means that every element $g\in G$ can be presented as a product $h_1\ldots h_n$ for $h_i\in H_i$.

The following theorem is the main result of \cite{S}.
\begin{thm}\label{theorem:Stepanov}
Let $\Phi$ be a reduced irreducible root system and let $S$ be any parabolic subset of roots of $\Phi$ with special part $\Sigma_S$.
Then $\E(\Phi, R, I)$ is generated by the following two families of elements:
\begin{itemize}
 \item $x_{\alpha}(s)$, where $s\in I$, $\alpha\in\Phi$;
 \item $z_\alpha(s,\xi)$, where $s\in I$, $\xi\in R$, $\alpha\in\Sigma_S$. \end{itemize} \end{thm}
\begin{proof} See \cite[Theorem~3.4]{S}. \end{proof}

\section{Bass---Kolster decomposition}\label{sec:bass-kolster}
The next theorem is a relative version of the so called Bass---Kolster decomposition (cf.~\cite[Theorem~2.1]{St78}).
\begin{thm}\label{thm:BassKolster}
Let $\Phi$ be a classical root system of rank $\ell\geqslant2$, let $R$ be an arbitrary commutative ring and $I$ be an ideal, satisfying one of the following assumptions:
\[\begin{array}{l@{\quad}l@{\quad}l@{\quad}c}
\Phi = \rA_\ell,\ \ell\geqslant 2, & \sr(I) \leqslant \ell; \\
\Phi = \rC_\ell,\ \ell\geqslant 2, & \sr(I) \leqslant 2\ell-1; \\
\Phi = \rB_\ell, \rD_\ell,\ \ell\geqslant 3, & \asr(I) \leqslant \ell-1.
\end{array}\]
Then the principal congruence subgroup $\G(\Phi,R,I)$ admits the following relative version of Bass---Kolster decomposition:
\[ \G(\Phi,R,I)=  \U(\Phi^+,I) \cdot \U(\Phi^-,I) \cdot Z \cdot \U(\Sigma_1^-\setminus\{-\alpha_\mathrm{max}\},I) \cdot \U(\Sigma_1,I) \cdot \G(\Delta_1,R,I), \]
where $Z=\left\{ z_{-\alpha_\mathrm{max}}(r,1)\ \middle|\ r\in I \right\}$.
\end{thm}
\begin{proof}

Let $g$ be an element of $\G(\Phi, R, I)$. Set $v=g \cdot v^+\in\Ums(n, I)$. 
Notice that in each case it suffices to find $g' \in \U(\Phi^-, I) \cdot \U(\Phi^+, I) \cdot g$ such that 
\begin{equation} \label{eq1} (g'\cdot v^+)_{1} = 1 + s \text{ and } (g'\cdot v^+)_{\varpi\ssub{1}-\alpha\ssub{max}} = s\ \text{for some}\ s\in I. \end{equation}
Indeed, set $g'' = z_{-\alpha\ssub{max}}(-s, 1) \cdot g'$.
Obviously, one has $(g''\cdot v^+)_1 = 1$, $(g''\cdot v^+)_{\varpi\ssub{1}-\alpha\ssub{max}}=0$ and the conclusion of the theorem follows from Lemma~\ref{lemma:Chevalley-Matsumoto}.

\textsc{Case $\Phi=\rA_\ell$, $n=\ell + 1$.}
%Set $v=(1+v_1,v_2,\ldots,v_\ell,v_{\ell+1})^t\in\Ums(\ell+1,R,I)$.
Thanks to the relative stable rank condition one can add suitable multiples of the last component $v_{\ell+1}$ to the first $\ell$ components of $v$ so that the upper
$\ell$ coefficients of the resulting vector $v'$ form an $I$-unimodular column of length $\ell$.
Now multiplying $v'$ by a suitable $y\in \U(\Sigma_\ell^-, I)$ we obtain equalities~\ref{eq1}.

\textsc{Case $\Phi=\rC_\ell$, $n=2\ell$.}
Notice that column $(v_1,\ldots, v_{-2}, v_{-1}^2)^t$ is also $I$-unimodular.
Applying condition $\sr(I)\leq 2\ell-1$ we find $c_1, c_2, \ldots, c_{-2} \in I \cdot v_{-1}$ such that upper $2\ell -1$ components of $v'=(v_1 + c_1 v_{-1}, \ldots, v_{-2} + c_{-2}v_{-1}, v_{-1})^t$ form an $I$-unimodular column.
By the choice of $c_i$ we can find suitable $d\in I$ such that $h_1 \cdot v = v'$ for
\[ h_1 = x_{1,-1}(c_1 + d) \cdot \prod_{i=2}^{-2} x_{i,-1}(c_i) \in \U(\Sigma_1^-, I). \]

We can find $f_1, f_2,\ldots, f_{-2} \in R$ such that $f_1v'_1+\sum_{i=2}^{-2} f_i v'_i = 1$.
%TODO: Determine exact sign
Set $\xi = v''_1-v''_{-1}-1 \in I$,
\[ h_2 = x_{-1,1}\biggl(\xi f_1 \pm \sum_{i=2}^\ell v_1' \xi^2 f_i f_{-i}\biggr) \cdot \prod_{i=2}^{-2} x_{-1,i}(\xi f_i) \in \U(\Sigma_1, I). \]
Direct computation shows that $v'' = h_2 \cdot v'$ satisfies equalities~\ref{eq1}.

%Now we can assume that the first $2\ell-1$ entries of $v$ are unimodular and find $c_1,\ldots,c_{-2}\in I$ such that $c_1v_1+\ldots+c_{-2}v_{-2}=(v_1-1)-v_{-1}$. Add $c_{-i}v_i$ to $v_{-1}$, $i=2,\ldots,\ell$:
%\[ (v_1,v_2\ldots,v_\ell,v_{-\ell},\ldots,v_{-2},v_{-1})^t\longmapsto (v_1,v'_2\ldots,v'_\ell,v_{-\ell},\ldots,v_{-2},v'_{-1})^t. \]
%Then add $c_iv_i'=c_i(v_1+c_{-i}v_{-i})$, $i=2,\ldots,\ell$ to the last entry:
%\[ v_1\longmapsto v_1,\quad v'_{-1}\longmapsto v_{-1}+\sum_{i=2}^\ell c_{-i}v_{-i}+\sum_{i=2}^\ell c_i(v_i+c_{-i}v_1)=v''_{-1}. \]
%Next add $\left(c_1-\sum_{i=2}^\ell c_ic_{-i}\right)v_1$ to $v''_{-1}$ to get $v_1-1$ in position $-1$.
%Again, as in case of $\rA_n$, apply $z_\gamma(1-v_1,1)$ with $\gamma=-\alpha_\mathrm{max}$, to get $1$ as the first entry and $0$ as the last.

\textsc{Case $\Phi=\rD_\ell$, $n= 2\ell$.} 
By Lemma~\ref{lemma:asrUnip} we can find $h_1\in \U(\Sigma^+_\ell, I)$ such that the upper half $v'_+$ of $v'=h_1 \cdot v$ is $I$-unimodular.
Since $\sr(I)\leq \ell-1$ we can find $c_1$, $c_3, \ldots c_\ell \in I$ such that $(v''_1, v''_3, \ldots, v''_\ell) \in \Ums(\ell-1, I)$, where
\[ v''=h_2 \cdot x_{1,2}(c_1) \cdot v', \quad h_2=\prod_{i=3}^\ell x_{i,2}(c_i). \]

We can find $f_1, f_3,\ldots, f_\ell \in R$ such that $f_1v''_1+\sum_{i=3}^\ell f_i v''_{i} = 1$.
As before, set
\[ \xi = v''_1-v''_{-2}-1 \in I, \quad h_3 = x_{-2,1}(\xi f_1) \cdot \prod_{i=3}^\ell x_{-2,i}(\xi f_i), \quad v'''=h_3 \cdot v''. \]
%Clearly, $v'''_{-2}=v'''_1-1$, therefore for $v_4 = z_{-\alpha_{max}}(-v'''_{-2}, 1) \cdot v'''$ one has $v^4_1 = 1$, as required.
Clearly, $t_{1,2}(c_1) \cdot h_1 \in \U(\Phi^+, I)$, $ h_3 \cdot h_2 \in \U(\Phi^-, I)$ and $v'''$ satisfies \ref{eq1}.

\textsc{Case $\Phi=\rB_\ell$, $n=2\ell+1$.} Subdivide $v\in \Ums(2\ell+1, I)$ as $v=(v_+, v_0, v_-)\in R^\ell\times R\times R^\ell$.
Denote by $J\leq I$ the ideal spanned by components of $v_-$.
Since $\sr(I/J)\leq \ell$ we can find $c_1,\dots,c_\ell\in I$ such that for $v' = h \cdot v$, $h = \prod_{i=1}^\ell x_{i,0}(c_i) \in \U(\Phi^+, I)$
one has $\bar{v'}_+=(\bar{v'_1},\ldots, \bar{v'_\ell}) \in \Ums(\ell, I/J)$ and, therefore, $(v'_+, v'_-) \in \Ums(2\ell, I)$.
Now the proof can be finished by repeating the argument for the case $\Phi=\rD_\ell$ (applied to the subset of long roots of $\rB_\ell$).
%(clearly, the maximal root of $\rD_\ell$ maps to the maximal root of $\rB_\ell$ under the natural embedding $\rD_\ell\subseteq\rB_\ell$). 
\end{proof}

It is easy to see that the proof of the above theorem is effective and gives an estimate of the total number of elementary root unipotents involved in the decomposition.
\begin{cor}
In the assumptions and notation of Theorem~\ref{thm:BassKolster} every element of $\G(\Phi,R,I)$ 
can be factored into a product of one element of $\G(\Delta_1,R,I)$ one element of $Z$ and at most $4(|\Phi^+| - |\Delta_1^+|)-1$ elementary root unipotents $x_\alpha(s)$ of level $I$. \end{cor}
\begin{proof}
The above estimates can be obtained by a careful analysis of the proof of the previous theorem.
Cases $\Phi=\rA_\ell, \rC_\ell$ are immediate.
In the case $\Phi=\rD_\ell$ the proof of Theorem~\ref{thm:BassKolster} implies that
\begin{multline}\nonumber
\G(\Phi,R,I) =  \U(\Sigma_\ell,I) \cdot X_{\alpha_1}(I) \cdot \U(\Sigma_2^-\cap\Delta_1,I) \cdot X_{-\alpha\ssub{\mathrm{max}}}(I) \cdot Z  \cdot \\ \cdot \U(\Sigma_1^-,I) \cdot \U(\Sigma_1,I) \cdot \G(\Delta_1,R,I).
\end{multline}
We can present an element $g$ of $\U(\Sigma_\ell, I)$ as a product of $g_1 \in \U(\Sigma_{\{1,2\}} \cap \Sigma_\ell)$ and $g_2\in \U(\Delta_{\{1,2\}}\cap \Sigma_\ell)$.
An examination of the extended Dynkin diagram of $\rD_\ell$ implies that $g_2$ either centralizes or normalizes all factors of the above decomposition (except the last one) and therefore can be moved to the right until it is consumed by $\G(\Delta_1)$.
On the other hand, $g_1$ is a product of at most $2\ell-3$ elementary unipotents, while the width of $\U(\Sigma_1^\pm, I)$ and $\U(\Sigma_2^-\cap\Delta_1)$ in elementary unipotents does not exceed $2\ell-2$ and $2\ell-4$, respectively.
Summing up these upper bounds we obtain
$$(2\ell-3) + 1 + (2\ell - 4) + 1 + 2\cdot (2\ell - 2) = 8\ell - 9 = 4(|\rD_\ell| - |\rD_{\ell-1}|) - 1.$$

The estimate in the case $\Phi=\rB_\ell$ can be obtained in a similar way. \end{proof}

\begin{proof}[Proof of Theorem~\ref{thm:SL2width}]
Consider the case $\Phi=\rA_\ell$, $\sr(I)\leq 2$. First of all, notice that we can improve the estimate of the number of factors involved in Bass---Kolster decomposition.
Indeed, when performing the first step of the proof of Theorem~\ref{thm:BassKolster} it suffices to make only $2$ additions of $v_{\ell+1}$ (e.\,g. to $v_{1}$ and $v_2$) to make the first $\ell$ entries of $v'$ form a unimodular column.
In particular, $\G(\Phi, R, I)$ can be presented as a product of one element of $\G(\Delta_1, R, I)$, one element of $Z$ and $3\ell+1$ root unipotents $x_\alpha(s)$, $s\in I$.
The latter elements are contained in a product of $3\ell + 1$ copies of $\SL(2, R, I)$. Now the statement of the theorem follows by induction on $\ell$.

The proof in the case $\Phi=\rC_\ell$ is similar (notice that we use the exceptional isomorphism $\SL(2, R)\cong \Sp(2, R)$).
\end{proof}


\section{Bounded generation}\label{sec:boundgen}

\subsection{Relative stable rank 1}
\begin{lemma}\label{lemma:srRI1}
If $sr(R,I)=1$, the width of $SL(2,R,I)$ with respect to $z_\alpha$ does not exceed $4$.
\end{lemma}
\begin{proof}

Let $A=\begin{pmatrix}a & b \\ c & d\end{pmatrix}\in SL(2,R,I)$. First column is $I$-unimodular, therefore there exists $z\in I$ such that $a+cz\in R^*$. Multiply $A$ by $x_{12}(z)$ from left to obtain invertible element in the upper left corner. Applying $x_{21}(-c/a)$ from left and $x_{12}(-b/a)$ from right, we obtain a diagonal matrix. Thus
\begin{multline*}
A=x_{12}(-z)\cdot x_{21}(c/a)\cdot
\begin{pmatrix} \varepsilon & 0 \\ 0 & 1/\varepsilon \end{pmatrix}
\cdot x_{12}(b/a)=\\
=x_{12}(-z)\cdot
\begin{pmatrix} \varepsilon & 0 \\ 0 & 1/\varepsilon \end{pmatrix}
\cdot x_{21}(y) \cdot x_{12}(b/a),
\end{multline*}
where all the coefficients in transvections lie in $I$ and $\varepsilon\in 1+I$. The above formula is a relative version of Gauss decomposition.
Note that
\begin{multline*}
\begin{pmatrix} \varepsilon & 0 \\ 0 & 1/\varepsilon \end{pmatrix} =
\begin{pmatrix} 1 & -1 \\ 0 & 1 \end{pmatrix}
\begin{pmatrix} 1 & 0 \\ 1-\varepsilon & 1 \end{pmatrix}
\begin{pmatrix} 1 & 1/\varepsilon \\ 0 & 1 \end{pmatrix}
\begin{pmatrix} 1 & 0 \\ \varepsilon^2-\varepsilon & 1 \end{pmatrix} =\\=
\begin{pmatrix} 1 & -1 \\ 0 & 1 \end{pmatrix}
\begin{pmatrix} 1 & 0 \\ 1-\varepsilon & 1 \end{pmatrix}
\begin{pmatrix} 1 & 1+z \\ 0 & 1 \end{pmatrix}
\begin{pmatrix} 1 & 1/\varepsilon-1-z \\ 0 & 1 \end{pmatrix}
\begin{pmatrix} 1 & 0 \\ \varepsilon^2-\varepsilon & 1 \end{pmatrix},
\end{multline*}
therefore
\[
A=z_{21}(1-\varepsilon,-1-z)\cdot x_{12}(1/\varepsilon-1-z)\cdot x_{21}(\varepsilon^2-\varepsilon+y)\cdot x_{12}(b/a). \qedhere
\]
\end{proof}

\subsection{Dedekind rings of arithmetic type}
\subsubsection{Number field case}
\paragraph{$\mathbb{Z}[\sfrac{1}{p}]$.}
\textbf{TODO:} Insert here Moree's lemma on primitive roots
\begin{lemma}\label{lemma:Z1p}
Assume GRH. If $R=\mathbb{Z}[\sfrac{1}{p}]$ and $I\lhd R$, the width of $SL(2,R,I)$ does not exceed $6$.
\end{lemma}
\begin{proof}
$I\lhd\mathbb{Z}[\sfrac{1}{p}]$, $I=(m)$, $m\in\mathbb{Z}$, $p\nmid m$.

\[ g=\begin{pmatrix}
x & y \\ z & w
\end{pmatrix},\quad x =p^\alpha a,\quad y =p^\beta bm, \]
where $a,b\in\mathbb{Z}$, $p\nmid a,b$ and $\alpha,\beta\in\mathbb{Z}$.

\textsc{Case 1:} $\alpha\geqslant\beta$. Since $p^{\alpha-\beta}a\perp bm^2$ and $p\perp bm^2$, there exist infinitely many rational primes $q$ of the form $p^{\alpha-\beta}a+bm^2k$, such that $p$ is a primitive root modulo $q$. One may choose $q$ prime to $b$. Then
\[ g_1=g\cdot x_{21}(mk) =
\begin{pmatrix} p^\beta q & p^\beta bm \\ * & * \end{pmatrix}.\]
$\exists u\geqslant 1: p^u\equiv b\mod q$, say $p^u=b+lq$. Then
\[ g_2 = g_1\cdot x_{12}(ml) =
\begin{pmatrix} p^\beta q & mp^{\beta+u} \\ * & * \end{pmatrix}. \]
$g_2\equiv 1\mod m$, thus $p^\beta q=1+cm$.
\begin{align*}
g_3 = & g_2\cdot x_{21}\left(\dfrac{-c}{p^{\beta+u}}\right) =
\begin{pmatrix} 1 & mp^{\beta+u} \\ * & * \end{pmatrix}, \\
g_4 = & g_3\cdot x_{12}(-mp^{\beta+u}) =
\begin{pmatrix} 1 & 0 \\ * & * \end{pmatrix}, \\
g_5 = & g_4\cdot x_{21}\left(\dfrac{c}{p^{\beta+u}}\right) =
\begin{pmatrix} 1 & 0 \\ * & * \end{pmatrix}.
\end{align*}
Note that $g_5=g_2\cdot z_{12}\left(-mp^{\beta+u},\dfrac{c}{p^{\beta+u}}\right)$ and that two other coefficients in transvections are multiples of $m$. Thus in this case the length of $g$ does not exceed $4$: $g=x_{21}z_{12}x_{12}x_{21}$.

\textsc{Case 2:} $\alpha<\beta$. Since $\mathbb{Z}[\sfrac{1}{p}]/I$ is finite, $\exists k>0$ such that $p^k\equiv 1\mod I$. One can choose $k>\beta-\alpha$. Then $k+\alpha>-k+\beta$ and
\[ g\cdot h(p^k) =
\begin{pmatrix} p^\alpha a & p^\beta bm \\ * & * \end{pmatrix}
\begin{pmatrix} p^k & 0 \\ 0 & p^{-k} \end{pmatrix}=
\begin{pmatrix} p^{k+\alpha} a & p^{-k+\beta} bm \\ * & * \end{pmatrix}.
\]
One can apply Case 1 fot the latter, so
$g=x_{21}z_{12}x_{12}x_{21}h^{-1}$,
and $x_{12}h^{-1}$ can be processed in the same way, as in Lemma \ref{lemma:srRI1}:
$g=x_{21}z_{12}x_{12}\cdot z_{12}x_{21}x_{12}$.
\end{proof}

\section{Relative Dennis-Vaserstein decomposition}\label{sec:dennis-vaserstein}
Throughout the present section we denote by $\EP_s(R, I)$ the subgroup $\E(\Delta_s, R, I) \cdot \U(\Sigma_s, I)$, $1 \leq s \leq n$.
%TODO: Add Levi decomposition to preliminaries
%TODO: Introduce notation for U(S, R)
Set $\EP_s := \EP_s(R, R) = \E(\Delta_s, R) \cdot \U(\Sigma_s, R)$. 

Let $\Phi$ be an irreducible root system of rank $\ell$.
Let $r$, $s$ be two distinct indices $1\leq r,s \leq \ell$.
From Levi decomposition it follows that
\begin{multline}\nonumber \U(\Phi^+, I)\cdot \U(\Phi^-, I) \cdot \E(\Delta_r, R, I) \cdot \EP_s(R, I) = 
\U(\Sigma_r, I)\cdot \U(\Sigma^-_r, I) \cdot \E(\Delta_r, R, I) \cdot \EP_s(R, I) = \\
= \EP_r(R, I) \cdot \E(\Delta_s, R, I) \cdot \U(\Sigma_s^-, I)\cdot \U(\Sigma_s, I) = 
\EP_r(R, I) \cdot \U(\Sigma^-_r\cap \Sigma^-_s, I) \cdot \EP_s(R, I). \end{multline}
Denote by $A_{rs}$ any of the equal subsets from the previous formula. 

\begin{thm}\label{theorem:relative_dv}
The relative elementary subgroup $\E(\Phi, R, I)$ coincides with $A_{rs}$ under the following assumptions on $(R, I)$.
  \begin{center}
    \begin{tabular}{| l | l | l | l | l |} \hline
    № & $\Phi$ & $(s,r)$ & ring condition \\ \hline
    1. & $\rA_\ell$, $\ell\geq 2$ & $(1, \ell)$ & $\sr(I) \leq \ell-1$ \\ \hline
    2. & $\rB_\ell$, $\ell\geq 3$ & $(1, \ell)$ & $\sr(I) \leq \ell-1$ \\ \hline
    3. & $\rD_\ell$, $\ell\geq 4$ & $(1, \ell)$ & $\sr(I) \leq \ell-2$ \\ \hline    
    4. & $\rE_\ell$, $\ell=6,7$ & $(\ell, 2)$ & $\sr(I) \leq \ell-3$ \\ \hline     
    5. & $\rE_\ell$, $\ell=6,7$ & $(\ell, 1)$ & $\asr(R, I)\leq \ell-2$ \\ \hline    
    \end{tabular} \end{center} 
\end{thm}
The proof of Theorem~\ref{theorem:relative_dv} occupies the rest of this section.

Consider the usual conjugation action of $\E(\Phi, R)$ on $\E(\Phi, R, I)$. 
This action induces an action of $\E(\Phi, R)$ on the set $\mathfrak{S}$ of all subsets of $\E(\Phi, R, I)$.
On the other hand, $\E(\Phi, R, I)$ acts on $\mathfrak{S}$ by left multiplication.
Denote by $N_{rs}$ and $L_{rs}$ stabilizers of $A_{rs} \in \mathfrak{S}$ with respect to these actions.
In other words $$N_{rs} = \{ g\in \E(\Phi, R) \mid g \cdot A_{rs} \cdot g^{-1} \subseteq A_{rs} \};\quad L_{rs}= \{ g\in \E(\Phi, R, I) \mid g \cdot A_{rs} \subseteq A_{rs} \}.$$

It is easy to see that $N_{rs}$ normalizes $L_{rs}$. Indeed, for $g\in N_{rs}$, $h\in L_{rs}$ one has
$$h^g \cdot A_{rs} = g^{-1} \cdot h \cdot g \cdot A_{rs} \subseteq g^{-1} \cdot h \cdot A_{rs} \cdot g \subseteq A_{rs}^g \subseteq A_{rs}.$$

\begin{lemma}\label{lemma:dv_unipotent} For any $1\leq i\leq n$ the following statements hold. \begin{enumerate} 
\item $\U(\Phi^\pm, I) = X_{\pm\alpha\ssub{i}}(I)\cdot \U(\Phi^\pm\setminus\{\pm\alpha\ssub{i}\}, I) = \U(\Phi^\pm\setminus\{\pm\alpha\ssub{i}\}, I)\cdot X_{\pm\alpha\ssub{i}}(I).$
\item For any $\xi\in R$ one has $\U(\Phi^\pm\setminus\{\alpha_i\}, I)^{x_{\mp\alpha\ssub{i}}(\xi)^{-1}} \subseteq \U(\Phi^\pm, I).$
\item $\U(\Phi^+, I)\cdot \U(\Phi^-, I) \subseteq \U(\Phi^+\setminus \{\alpha_i\}, I) \cdot \U(\Phi^-, I) \cdot X_{\alpha\ssub{i}}(I) \cdot X_{-\alpha\ssub{i}}(I)$.
\end{enumerate} \end{lemma}
\begin{proof}
 The first two statements easily follow from Chevalley commutator formula while the third one is a formal consequence of the first two.
\end{proof}

The following lemma is a relative version of the main reduction used by M.~Stein in~\cite{St78} for the proof of the absolute Dennis--Vaserstein decomposition.
\begin{lemma}\label{lemma:Stein_reduction}
Assume that there exists a subset $\widetilde{L_r} \subseteq \E(\Delta_r, R, I)$ with the following properties:
\begin{enumerate}[label=(\alph*)] 
 \item\label{stein_cond1} One has $\U(\Sigma^+_r, I)\cdot \U(\Sigma^-_r, I) \cdot \E(\Delta_r, R, I) \subseteq \U(\Phi^+, I)\cdot \U(\Phi^-, I) \cdot \widetilde{L_r}.$
 \item\label{stein_cond2} One has $X_{-\alpha\ssub{r}}(I)^{\widetilde{L_r}} \subseteq \EP_s(R, I).$
\end{enumerate}
Then $X_{-\alpha_r}(I) \subseteq L_{rs}$ and $\EP_s \subseteq N_{rs}.$
\end{lemma}
\begin{proof} Set $A:=\U(\Phi^+, I)\cdot \U(\Phi^-, I) \cdot \widetilde{L_r} \cdot \EP_s(\Phi, R, I).$
From condition~\ref{stein_cond1} of the lemma it follows that $A_{rs}=A$.
Observe that from Lemma~\ref{lemma:dv_unipotent} and condition~\ref{stein_cond2} it follows that
\begin{multline}\nonumber 
A \subseteq \U(\Phi^+\setminus \{\alpha_r\}, I) \cdot \U(\Phi^-, I) \cdot X_{\alpha\ssub{r}}(I) \cdot X_{-\alpha\ssub{r}}(I) \cdot \widetilde{L_r} \cdot \EP_s(R, I) \subseteq \\ 
\subseteq \U(\Phi^+\setminus\{\alpha_r\}, I) \cdot \U(\Phi^-, I) \cdot X_{\alpha\ssub{r}}(I) \cdot X_{-\alpha\ssub{r}}(I) \cdot \widetilde{L_r} \cdot \EP_s(R, I) \subseteq \\
\subseteq \U(\Phi^+\setminus\{\alpha_r\}, I) \cdot \U(\Phi^-, I) \cdot \widetilde{L_r} \cdot \U(\Sigma_r, I) \cdot \EP_s(R, I)  \cdot \EP_s(R, I) \subseteq \\
\subseteq \U(\Phi^+\setminus\{\alpha_r\}, I) \cdot \U(\Phi^-, I) \cdot \widetilde{L_r} \cdot \EP_s(R, I). \end{multline}
Applying Lemma~\ref{lemma:dv_unipotent} we get that:
\begin{equation}\nonumber A^{X_{-\alpha\ssub{r}}} \subseteq \U(\Phi^+, I) \cdot \U(\Phi^-, I) \cdot \widetilde{L_r} ^{X_{-\alpha\ssub{r}}} \cdot \EP_s(R, I) \subseteq A. \end{equation}
\begin{equation}\nonumber X_{-\alpha\ssub{r}}(I) \cdot A \subseteq \U(\Phi^+, I) \cdot X_{-\alpha\ssub{r}}(I) \cdot \U(\Phi^-, I) \cdot \widetilde{L_r} \cdot \EP_s(R, I) = A. \end{equation}
To prove the second part of the statement observe first that $\EP_s$ is generated by $X_{\alpha\ssub{i}}$ for $1\leq i\leq n$ and $X_{-\alpha\ssub{i}}$ for $i\neq s$.
%TODO: Find proper reference for this fact
We have just shown that $X_{-\alpha_r}\subseteq N_{rs}$.
On the other hand, inclusions $X_{\alpha\ssub{k}} \subseteq N_{rs}$ for $\ 1\leq k\leq \ell$ and $X_{-\alpha\ssub{k}} \subseteq N_{rs}$ for $k\neq r,s$ are obvious.
\end{proof}

\begin{proof}[Proof of Theorem~\ref{theorem:relative_dv}]
We first show that under specified assumptions on $(R, I)$ one can meet the conditions of Lemma~\ref{lemma:Stein_reduction}.
Consider the following two subsets of $\Lambda(\pi)$:
$$\Gamma = \varpi_s- (\Sigma_s^+\cap \Delta_r),\quad \Gamma_0 = \{\lambda \in \Gamma \mid \lambda - \alpha_r \in \Lambda(\pi) \}.$$
Clearly, $\Gamma$ is the set of weights of an irreducible representation of $\G(\Delta_r, R)$ corresponding to the same highest weight $\varpi_s$.
The subsystem $\Delta_r$ has type $\rA_{\ell-1}$ in all cases except the last one.
It is also clear that $|\Gamma_0|=1$ for $\Phi=\rA_\ell, \rB_\ell$, $|\Gamma_0|=2$ for $\Phi=\rD_\ell$, $|\Gamma_0|=3$ for $\Phi=\rE_\ell$ and $r=2$.
In the case $\Phi=\rE_\ell$, $r=2$ the subsystem $\Delta_r$ has type $\rD_{\ell-1}$ and $|\Gamma_0|=\ell-1$.

Let $\widetilde{L_r}$ be the set of all elements $g$ of $\E(\Delta_r,R, I)$ such that $(g \cdot v^+)_\lambda = 0$ for $\lambda\in\Gamma_0$.
In any of specified cases the assumption on $(R, I)$ allows us to apply Lemma~\ref{lemma:uraction} to the subsystem $\Delta_r$ and find
$x\in\U(\Delta_r\cap\Phi^+, I)$, $y\in \U(\Delta_r\cap\Phi^-, I)$ such that $yx\cdot g \in \widetilde{L_r}$.
This proves the first condition of Lemma~\ref{lemma:Stein_reduction}, indeed:
$$ \U(\Sigma^+_r, I)\cdot \U(\Sigma^-_r, I) \cdot g = \U(\Sigma^+_r, I) x^{-1} \cdot \U(\Sigma^-_r, I)^{x^{-1}} y^{-1} \cdot (yxg) \subseteq \U(\Phi^+, I)\cdot \U(\Phi^-, I) \cdot \widetilde{L_r}.$$
To prove the second condition notice that by the definition of $\Gamma_0$ for any $s\in I$, $ g\in\widetilde{L_r}$ one has $x_{-\alpha_r}(s) \cdot g \cdot v^+ = g \cdot v^+$ and, therefore,
$$X_{-\alpha\ssub{r}}(I)^{\widetilde{L_r}} \subseteq \U(\Phi^-, I) \cap \Stab(v^+) \subseteq \E(\Delta_s, R, I) \subseteq \EP_s(R, I).$$

From now on we can assume that the statement of Lemma~\ref{lemma:Stein_reduction} holds and $\EP_s$ normalizes $A_{rs}$.
Notice that in view of Theorem~\ref{theorem:Stepanov} it suffices to show that the following two families of elements are contained in $L_{rs}$:
\begin{itemize} \item $z_{\alpha}(s, \xi)$, $s\in I$, $\xi \in R$, $\alpha\in\Sigma^-_s$;
\item $x_{\beta}(s)$, $s \in I$, $\beta \in \Phi$. \end{itemize}
Since $\EP_s \subseteq N_{rs}$ it suffices to check inclusions only for the second family of elements.
We already know that $\U^+(\Phi, I) \subseteq L_{rs}$.

Notice that the Weyl group $W(\Delta_s)$ acts transitively on $\Delta_s$, therefore in view of relation~\ref{rel:R3} the subgroup
$W_s := \langle w_\alpha(1) \mid \alpha\in\Delta_s\rangle \leq \EP_s$ acts transitively on the set of root subgroups $X_\alpha(I)$, $\alpha\in \Delta_s$.
Since $X_{-\alpha\ssub{r}}(I) \subseteq L_{rs}$ we get that that $\U(\Delta_s \cap \Phi^-, I)\subseteq L_{rs}$.

Now denote by $\widetilde{\alpha}$ the maximal root of $\Phi$. Our assumptions on $\Phi$ guarantee that $m_s(\widetilde{\alpha})=1$, 
and, consequently, every two roots $\alpha, \beta \in \Sigma^-_s$ have the same $s$-shape (i.\,e. $\shape(\{s\}, \alpha = \shape(\{s\}, \beta)$).
By Lemma~\ref{lemma:abs} $W(\Delta_s)$ interchanges $\alpha$ and $\beta$ if their length is equal (which is the case if we assume additionally $\Phi\neq \rB_\ell$).
Since $X_{\alpha\ssub{s}} \subseteq L_{rs}$ the argument similar to the one above implies $\U(\Sigma^-_s, I)\subseteq L_{rs}$.
This completes the proof of the theorem for $\Phi\neq \rB_\ell$. 
%TODO:
\textbf{TODO: Finish the proof for $\Phi=\rB_\ell$.}
\end{proof}


\printbibliography

\end{document}
