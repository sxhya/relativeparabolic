\documentclass[12pt]{amsart}
\usepackage{amscd, amsmath, verbatim, enumitem, graphicx, amssymb, mathtools, xfrac, amsthm, amsfonts, amstext}
\usepackage{hyperref}
\usepackage[utf8]{inputenc}
\usepackage[backend=biber, citestyle=numeric-comp, natbib=true, sortlocale=en_US, url=false, doi=false, eprint=true, maxbibnames=4]{biblatex}            
\usepackage[all]{xy}

\renewbibmacro{in:}{\ifentrytype{article}{}{\printtext{\bibstring{in}\intitlepunct}}}
\newbibmacro{string+doi}[1]{\iffieldundef{doi}{\iffieldundef{url}{#1}{\href{\thefield{url}}{#1}}}{\href{http://dx.doi.org/\thefield{doi}}{#1}}}
\DeclareFieldFormat{title}{\usebibmacro{string+doi}{\mkbibemph{#1}}}
\DeclareFieldFormat[article]{title}{\usebibmacro{string+doi}{\mkbibquote{#1}}}
\DeclareFieldFormat[inproceedings]{title}{\usebibmacro{string+doi}{\mkbibquote{#1}}}
\addbibresource{paper.bib}

\textwidth 16cm 
\textheight 22cm 
\headheight 0.5cm 
\evensidemargin 0.3cm 
\oddsidemargin 0.2cm

\numberwithin{equation}{section}
\newcounter{thmcounter} \newcounter{lemmacounter}
\newtheorem{thm}[thmcounter]{Theorem}
\newtheorem{prop}[thmcounter]{Proposition}
\newtheorem{cor}[thmcounter]{Corollary}
\newtheorem{lemma}[lemmacounter]{Lemma}
\theoremstyle{definition}
\newtheorem{rem}[equation]{Remark}
\newtheorem{example}[equation]{Example}
\newtheorem{dfn}[equation]{Definition}
\newtheorem{notation}[equation]{Notation}

\newtheorem*{thm*}{Theorem}
\newtheorem*{lemma*}{Lemma}

\DeclareMathOperator{\K}{K}
\DeclareMathOperator{\G}{G}
\DeclareMathOperator{\GL}{GL}
\DeclareMathOperator{\SL}{SL}
\DeclareMathOperator{\St}{St}
\DeclareMathOperator{\E}{E}
\DeclareMathOperator{\EP}{EP}
\DeclareMathOperator{\U}{U}
\DeclareMathOperator{\X}{X}
\DeclareMathOperator{\Z}{Z}
\DeclareMathOperator{\M}{M}
\DeclareMathOperator{\SR}{SR}
\DeclareMathOperator{\sr}{sr}
\DeclareMathOperator{\shape}{shape}
\DeclareMathOperator{\Rad}{Rad}
\DeclareMathOperator{\Max}{Max}
\DeclareMathOperator{\Spec}{Spec}
\DeclareMathOperator{\Stab}{Stab}
\DeclareMathOperator{\ASR}{ASR}
\DeclareMathOperator{\asr}{asr}
%TODO: Deprecate Um -- Use Ums/Umd whenever possible
\DeclareMathOperator{\Um}{Um}
\DeclareMathOperator{\Ums}{Ums}
\DeclareMathOperator{\Umd}{Umd}
\DeclareMathOperator{\rk}{rk}
\newcommand{\rA}{\mathsf{A}}
\newcommand{\rB}{\mathsf{B}}
\newcommand{\rC}{\mathsf{C}} 
\newcommand{\rD}{\mathsf{D}} 
\newcommand{\rE}{\mathsf{E}}

\makeatletter
\newcommand{\indexbox}[1]{\text{\fboxsep=.1em\fbox{\m@th$\displaystyle#1$}}}
\makeatother

\def\ssub#1{\mathchoice
   {_{\lower2pt\hbox{$\scriptstyle #1$}}}
   {_{\lower2pt\hbox{$\scriptstyle #1$}}}
   {_{\lower1.5pt\hbox{$\scriptscriptstyle #1$}}}
   {_{\!\lower1.5pt\hbox{$\scriptscriptstyle #1$}}}}

\title{Relative parabolic factorizations of Chevalley groups}
\keywords {Chevalley groups, relative subgroups, stability for $\K_1$. {\em Mathematical Subject Classification (2010):} 20G35, 19B14}
\author {Sergey Sinchuk, Andrei Smolensky}
\email {sinchukss {\it at} yandex.ru,\ andrei.smolensky {\it at} gmail.com}
\date {\today}

\begin{document}

%\begin{abstract} \end{abstract}

\maketitle

\section{Introduction}\label{sec:intro}
 
\subsection{Acknowledgements}
Authors of the present paper acknowledge financial support from Russian Science Foundation grant 14-11-00297.

\section{Preliminaries}\label{sec:prelim}
For any collection of subsets $H_1,\ldots, H_n$ of a group $G$ we denote by $H_1\ldots H_n$ their Minkowski set-product,
i.\,e. the set consisting of arbitrary products $h_1\ldots h_n$ of elements $h_i\in H_i$. In particular, the equality
$G = H_1\cdot\ldots\cdot H_n$ means that every element $g\in G$ can be presented as a product $h_1\ldots h_n$ for $h_i\in H_i$.

\subsection{Root systems}
Let $\Pi=\{\alpha_1, \ldots, \alpha_l\}$ be a basis of $\Phi$. Denote by $\Phi^+$ and $\Phi^-$ subsets of positive and negative roots with respect to $\Pi$.
Denote by $m_i(\alpha)$ the coefficient of $\beta$ in the expansion of $\alpha$ in $\Pi$, i.\,e. $\alpha = \sum\limits_{i=1}^n m_i(\alpha) \alpha_i$.

A root subset $S\subseteq \Phi$ is called {\it parabolic} (resp. {\it reductive}, resp. {\it special}) if $\Phi=S \cup -S$ (resp. $S = -S$, resp. $S \cap -S=\varnothing$).
Any parabolic subset $S \subseteq \Phi$ can be decomposed into the disjoint union of its reductive and special part, i.e. 
$S = \Sigma_S \sqcup \Delta_S$, where $\Sigma_S \cap (-\Sigma_S) = \varnothing$, $\Delta_S = -\Delta_S$.

We denote by $(\alpha, \beta)$ the scalar product of roots and by $\langle \alpha, \beta\rangle$ the integer $2(\alpha, \beta)/(\alpha, \alpha)$.
Denote by $W(\Phi)$ the subgroup of isometries of $\Phi$ generated by all reflections $\sigma_\alpha$ where $\sigma_\alpha(\beta)=\beta-\langle\alpha,\beta \rangle\cdot \alpha$.
For $S\subseteq \Phi$ denote by $W(S)$ the subgroup of $W(\Phi)$ generated by $\sigma_\alpha$ for $\alpha\in S$.

Let $J\subseteq \Pi$ be a subset of simple roots. 
For a root $\beta = \sum d_i\alpha_i \in \Phi$ set $\shape(J, \beta):=\sum\limits_{i\in J}d_i \alpha_i$.
Set $$\Delta_J = \{\alpha \in \Phi \mid \shape(J, \alpha)=0\},\quad \Sigma^\pm_J = \{\alpha \in \Phi \mid \shape(J, \alpha)\in\Phi^\pm \}.$$
Clearly, $\Delta_J$ is a reductive subset of roots while $\Sigma^\pm_J$ are special subsets.

\begin{lemma}[{\cite[Lemma~1]{ABS}}]\label{lemma:abs}
Let $\alpha, \beta \in \Sigma^\pm_J$ be two roots having the same length such that $\shape(J,\alpha)=\shape(J,\beta)\neq 0$.
Then $\alpha$ and $\beta$ are conjugate under the action of $W(\Delta_J)$.
\end{lemma}

\subsection{Chevalley groups}
Our exposition of Chevalley groups is standard and follows (...).
\textbf{TODO: Write more about Chevalley groups}

In this paper we work with irreducible representations of Chevalley groups corresponding to the highest weight
$\varpi_1$ (in the case of a classical $\Phi_l$) or $\varpi_l$ (in the case $\Phi_\ell=\rE_\ell$). 
The former are the natural vector representations of classical groups acting on free modules $V=R^n$ of dimension 
$n=\ell+1, 2\ell+1, 2\ell,2\ell$ for $\Phi=\rA_\ell,\rB_\ell,\rC_\ell,\rD_\ell$.
The latter representations have dimension $n=27, 56$ for $\ell=6,7$.

%TODO: Rewrite this using better terminology.
For a representation $\pi$ of $\G(\Phi, R)$ on $V$ we denote by $\Lambda(\pi)$ the set of weights of $\pi$ and by $V_\lambda$ the weight subspace corresponding to $\lambda\in\Lambda(\pi)$.
For any of the representations described above (with the sole exception of $(\rE_8, \varpi_8)$) all weight subspaces $V_\lambda$ are one-dimensional. 
This allows us to choose some basis $E_\pi=\{e_\lambda\}_{\lambda\in\Lambda(\pi)}$.

It will be convenient to use standard numbering for the weights of natural representations of classical groups (cp.~\cite[§1B]{St78}):
$$\begin{array}{cll}
  1,2,\ldots, \ell+1 & \text{in the case} & \Phi =\rA_\ell, \\
  1,2,\ldots \ell, 0, -\ell,\ldots, -2, -1 & \text{in the case} & \Phi =\rB_\ell, \\
  1,2,\ldots \ell, -\ell,\ldots, -2, -1 & \text{in the cases}   & \Phi =\rC_\ell, \rD_\ell. \\
\end{array}$$
For example, we write $1$ instead of $\varpi_1$, $2$ instead of $\varpi_1-\alpha_1$ etc.

For a weight $\lambda\in \Lambda(\pi)$ we denote by $(-)_\lambda\colon V\rightarrow R$ the coordinate function corresponding to $\lambda$.
In other words,  $v_\lambda$ is the coefficient of the basis vector $e_\lambda$ in the expansion of $v$ in $E_\pi$, i.e. $v=\sum_{\lambda\in \Lambda(\pi)} v_\lambda e_\lambda$.
We denote by $v^+$ the basis vector corresponding to highest weight.
For classical $\Phi$ we have $v^+=e_1$.

The lemma is a relative version of the classical result sometimes called Chevalley---Matsumoto decomposition.
%TODO: Recheck the statement of the lemma; finish the proof.
\begin{lemma}\label{lemma:Chevalley-Matsumoto}
Let $\pi$ be the fundamental representation of $\G_{sc}(\Phi, R)$ with the highest weight $\varpi_s$.
Assume that $g\in \G_{sc}(\Phi, R, I)$ is such that $(g\cdot v^+)_{\varpi\ssub{s}}=1$, then 
$g \in \U(\Sigma_s^-, I) \cdot \G_{sc}(\Delta_s, R, I) \cdot \U(\Sigma_s^+, I)$.
\end{lemma} \begin{proof}  For $I=R$ the statement of the lemma is contained in~\cite[Theorem~1.3]{St78}). \end{proof}

\subsection{Elementary subgroup}
Recall from~\cite{St78, VP} that for $\alpha\in \Phi$, $\xi\in R$ one can define certain elements $t_{\alpha}(\xi)$ of $\G_{sc}(\Phi, R)$ called {\it elementary root unipotents}.
These elements satisfy well-known Steinberg relations.
\begin{equation}\label{rel:add} t_\alpha(s) t_\alpha(t) = t_\alpha(s+t), \end{equation} 
\begin{equation}\label{rel:CCF} [t_\alpha(s),  t_\beta(t)] = \left(\prod\limits t_{i\alpha + j\beta}(N_{\alpha,\beta, i, j}s^i t^j)\right),\quad \alpha\neq-\beta, N_{\alpha, \beta, i, j}\in\mathbb{Z}.\end{equation}
The indices $i$, $j$ in the above formula range over all positive natural numbers such that $i\alpha + j\beta\in\Phi$.
The integers $N_{\alpha, \beta, i, j}$ are called {\it structure constants} of the Chevalley group $\G(\Phi,R)$ and depend only on $\Phi$.

For $\xi\in R^*$ set $w_\alpha(\xi) := x_\alpha(\xi) x_{-\alpha}(-\xi^{-1}) x_{\alpha}(\xi).$
If $\rk(\Phi)\geq 2$ the following relation is a consequence of \ref{rel:add}--\ref{rel:CCF}:
\begin{equation}\label{rel:R3} w_\alpha(\xi) t_{\beta}(s) w_\alpha(\xi)^{-1} =
t_{\sigma\ssub{\alpha}\beta} (\eta_{\alpha, \beta}\xi^{-\langle\beta,\alpha \rangle} s),\ \xi,s\in R. \end{equation}
where coefficients $\eta_{\alpha, \beta} = \pm 1$ (see~\cite[\S13]{VP} for more details on $\eta_{\alpha, \beta}$).

\begin{rem} Let $\lambda_1, \lambda_2 \in \Lambda(\pi)$ be a pair of weights of a representation $\pi$ of a classical group such that $\lambda_1-\lambda_2\in \Phi$.
In this situation it will be convenient for us to write $t_{\lambda_1,\lambda_2}(\xi)$ instead of $t_{\lambda_1-\lambda_2}(\xi)$.
For example, for $\Phi=\rA_\ell$ we have $t_{1,2}(\xi)=t_{1-2}(\xi)=t_{\varpi_1 - \varpi_1 + \alpha_1}(\xi) = t_{\alpha_1}(\xi)$. \end{rem}

Let $I\leq R$ be an ideal. Denote by $\E(\Phi, I)$ the subgroup of $\G_{sc}(\Phi, R)$ generated by elementary root unipotents $t_\alpha(\xi)$, $\alpha\in\Phi$, $\xi\in I$.
In the special case $I=R$ the subgroup $\E(\Phi, R)$ is called the {\it elementary subgroup} of the Chevalley group $\G_{sc}(\Phi, R)$.
Taddei's theorem asserts that $\E(\Phi, R)$ is a normal subgroup of $\G_{sc}(\Phi, R)$ provided $\Phi$ is an irreducible root system of rank $\geq 2$.

We denote by $\E(\Phi, R, I)$ the normal closure of $\E(\Phi, I)$ in $\E(\Phi, R)$, in other words $\E(\Phi, R, I) = \E(\Phi, I)^{\E(\Phi, R)}$.
Denote by $z_\alpha(s, \xi)$ the element $t_\alpha(s)^{t\ssub{-\alpha}(\xi)}$.
The following result of L.~Vaserstein and J.~Tits describes a set of generators of $\E(\Phi, R, I)$ (see e.\,g.~\cite[Theorem~2]{Va86}).

\begin{thm} Let $\Phi$ be an irreducible root system of rank $\geq 2$, then the relative elementary subgroup $\E(\Phi, R, I)$
is generated by elements $z_\alpha(s, \xi)$, $\alpha\in \Phi$, $s\in I$, $\xi\in R$.
\end{thm}

In fact, one can choose a more economical set of generators for $\E(\Phi, R, I)$ (see~\cite[Theorem~3.4]{S}).
\begin{thm}\label{theorem:Stepanov}
Let $\Phi$ be a reduced irreducible root system of rank $\geq 2$ and let $S$ be any parabolic subset of roots of $\Phi$ with special part $\Sigma_S$.
Then $\E(\Phi, R, I)$ is generated by the following two families of elements:
\begin{itemize}
 \item $t_{\alpha}(s)$, where $s\in I$, $\alpha\in\Phi$;
 \item $z_\alpha(s,\xi)$, where $s\in I$, $\xi\in R$, $\alpha\in\Sigma_S$. \end{itemize} \end{thm}

\section{Stability conditions}\label{sec:stability-conditions}
As we will be mainly concerned with applications to stability of $K_1$ of Chevalley groups, our treatment of stability conditions is restricted to the case of commutative $R$. 
We refer the reader to~\cite{Ba64, Va69, Va71} for a more general exposition.

\subsection{Stable rank}
Recall the definition of {\it stable rank} introduced by H.~Bass and L.~Vaserstein (see~\cite{Ba64, Va69}).
\begin{dfn} Recall that a row $a=(a_1,\ldots, a_n)\in {}^n\!R$ is called {\it $I$-unimodular} if elements $a_1-1, a_2, \ldots, a_n$ are contained in $I$ while $a_1, a_2, \ldots, a_n$ generate $R$ as an ideal.\end{dfn}
A column $b \in R^n$ will be called $I$-unimodular if its transpose $b^t$ is an $I$- unimodular row. We denote the set of all $I$-unimodular rows (resp. columns) by $\Umd(n,R,I)$ (resp. $\Ums(n,R,I)$).
When $R=I$ we refer to $R$-unimodular rows and columns as simply unimodular.

\begin{lemma} \label{lemma:relstrlemma} For any $I$-unimodular row $a=(a_1, \ldots, a_n)$ there exists $I$-unimodular column $b=(b_1,\ldots, b_n)^T$ such that $ab = \sum\limits_{i=1}^n a_i b_i = 1$. \end{lemma}

\begin {dfn} An $I$-unimodular row $a=(a_1,\ldots a_{n+1})$ is called {\it stable} if one can choose $b_1,\ldots, b_n\in R$ such that
row $(a_1 + a_{n+1}b_1,\ldots, a_{n}+ a_{n+1}b_n)$ is also $I$-unimodular. if, moreover, elements $b_1,\ldots, b_n$ can be chosen from  $I$ then such $a$ will be called {\it $I$-stable}. \end{dfn} 

We say that a pair $(R, I)$ satisfies stable range condition $\SR_n(R, I)$ if any $I$-unimodular row of length $n+1$ is stable.
\begin{lemma}\label{lemma:relstrlemma2}\strut\begin{enumerate}\item The condition $\SR_n(R, I)$ implies $\SR_m(R,I)$ for any $m\geq n$.
\item The condition $\SR_n(R, I)$ implies that any $I$-unimodular row of length $n+1$ is $I$-stable.\end{enumerate}\end{lemma}

By definition, the {\it relative stable rank} $\sr(R, I)$ of a pair $(R, I)$ is the smallest natural number $n$ such that condition $\SR_n(R, I)$ holds.
If $\SR_n(R, I)$ is not satisfied for any $n>0$ we set $\sr(R, I)=\infty$.

The following proposition summarizes basic properties of stable ranks.
\begin{prop} \label{prop:sr_properties} Let $R$ be arbitary commutative unital ring and let $I\trianglelefteq R$ be its ideal.
 \begin{itemize}
  \item For any ideal $J\trianglelefteq R$, $J\subseteq I$ one has $$\sr(R, I)\leq\sr(R),\quad \sr(R/J, I/J)\leq \sr(R, I).$$
  \item Stable rank of a direct product of rings is equal to the maximum of stable ranks of its factors: $$\sr(\prod\limits_{i=1}^n R_i) = \max\limits_{i=1}^n\left(\sr(R_i)\right).$$
  \item Stable rank does not change after taking quotient modulo nil-radical: $\sr(R)=\sr(R/\Rad(R))$.
 \end{itemize}\end{prop} \begin{proof} See~\cite{Va71}. \end{proof}

\begin{example} Since the stable rank of a field is one, one can conclude from the previous proposition that $\sr(R)=1$ for any semilocal ring $R$. \end{example}

\subsection{Absolute stable rank}

%TODO: $\asr(R,I)$

\section{Relative parabolic factorizations} \label{sec:factorizations}
\subsection{Relative Bass---Kolster decompositions}\label{sec:bass-kolster}
The next theorem is a relative version of the so called Bass---Kolster decomposition (cf.~\cite[Theorem~2.1]{St78}).
\begin{thm}\label{thm:BassKolster}
Let $\Phi$ be a classical root system of rank $\ell\geqslant2$, let $R$ be an arbitrary commutative ring and $I$ be an ideal, satisfying one of the following assumptions:
\[\begin{array}{l@{\quad}l@{\quad}l@{\quad}c}
\Phi = \rA_\ell,\ \ell\geqslant 2, & \sr(I) \leqslant \ell; \\
\Phi = \rC_\ell,\ \ell\geqslant 2, & \sr(I) \leqslant 2\ell-1; \\
\Phi = \rB_\ell, \rD_\ell,\ \ell\geqslant 3, & \asr(I) \leqslant \ell-1.
\end{array}\]
Then the principal congruence subgroup $\G(\Phi,R,I)$ admits the following relative version of Bass---Kolster decomposition:
\[ \G(\Phi,R,I)=  \U(\Phi^+,I) \cdot \U(\Phi^-,I) \cdot Z \cdot \U(\Sigma_1^-\setminus\{-\alpha_\mathrm{max}\},I) \cdot \U(\Sigma_1,I) \cdot \G(\Delta_1,R,I), \]
where $Z=\left\{ z_{-\alpha_\mathrm{max}}(r,1)\ \middle|\ r\in I \right\}$.
\end{thm}
\begin{proof}

Let $g$ be an element of $\G(\Phi, R, I)$. Set $v=g \cdot v^+\in\Ums(n, I)$. 
Notice that in each case it suffices to find $g' \in \U(\Phi^-, I) \cdot \U(\Phi^+, I) \cdot g$ such that 
\begin{equation} \label{eq1} (g'\cdot v^+)_{1} = 1 + s \text{ and } (g'\cdot v^+)_{\varpi\ssub{1}-\alpha\ssub{max}} = s\ \text{for some}\ s\in I. \end{equation}
Indeed, set $g'' = z_{-\alpha\ssub{max}}(-s, 1) \cdot g'$.
Obviously, one has $(g''\cdot v^+)_1 = 1$, $(g''\cdot v^+)_{\varpi\ssub{1}-\alpha\ssub{max}}=0$ and the conclusion of the theorem follows from Lemma~\ref{lemma:Chevalley-Matsumoto}.

\textsc{Case $\Phi=\rA_\ell$, $n=\ell + 1$.}
%Set $v=(1+v_1,v_2,\ldots,v_\ell,v_{\ell+1})^t\in\Ums(\ell+1,R,I)$.
Thanks to the relative stable rank condition one can add suitable multiples of the last component $v_{\ell+1}$ to the first $\ell$ components of $v$ so that the upper
$\ell$ coefficients of the resulting vector $v'$ form an $I$-unimodular column of length $\ell$.
Now multiplying $v'$ by a suitable $y\in \U(\Sigma_\ell^-, I)$ we obtain equalities~\ref{eq1}.

\textsc{Case $\Phi=\rC_\ell$, $n=2\ell$.}
Notice that column $(v_1,\ldots, v_{-2}, v_{-1}^2)^t$ is also $I$-unimodular.
Applying condition $\sr(I)\leq 2\ell-1$ we find $c_1, c_2, \ldots, c_{-2} \in I \cdot v_{-1}$ such that upper $2\ell -1$ components of $v'=(v_1 + c_1 v_{-1}, \ldots, v_{-2} + c_{-2}v_{-1}, v_{-1})^t$ form an $I$-unimodular column.
By the choice of $c_i$ we can find suitable $d\in I$ such that $h_1 \cdot v = v'$ for
\[ h_1 = x_{1,-1}(c_1 + d) \cdot \prod_{i=2}^{-2} x_{i,-1}(c_i) \in \U(\Sigma_1^-, I). \]

We can find $f_1, f_2,\ldots, f_{-2} \in R$ such that $f_1v'_1+\sum_{i=2}^{-2} f_i v'_i = 1$.
%TODO: Determine exact sign
Set $\xi = v''_1-v''_{-1}-1 \in I$,
\[ h_2 = x_{-1,1}\biggl(\xi f_1 \pm \sum_{i=2}^\ell v_1' \xi^2 f_i f_{-i}\biggr) \cdot \prod_{i=2}^{-2} x_{-1,i}(\xi f_i) \in \U(\Sigma_1, I). \]
Direct computation shows that $v'' = h_2 \cdot v'$ satisfies equalities~\ref{eq1}.

%Now we can assume that the first $2\ell-1$ entries of $v$ are unimodular and find $c_1,\ldots,c_{-2}\in I$ such that $c_1v_1+\ldots+c_{-2}v_{-2}=(v_1-1)-v_{-1}$. Add $c_{-i}v_i$ to $v_{-1}$, $i=2,\ldots,\ell$:
%\[ (v_1,v_2\ldots,v_\ell,v_{-\ell},\ldots,v_{-2},v_{-1})^t\longmapsto (v_1,v'_2\ldots,v'_\ell,v_{-\ell},\ldots,v_{-2},v'_{-1})^t. \]
%Then add $c_iv_i'=c_i(v_1+c_{-i}v_{-i})$, $i=2,\ldots,\ell$ to the last entry:
%\[ v_1\longmapsto v_1,\quad v'_{-1}\longmapsto v_{-1}+\sum_{i=2}^\ell c_{-i}v_{-i}+\sum_{i=2}^\ell c_i(v_i+c_{-i}v_1)=v''_{-1}. \]
%Next add $\left(c_1-\sum_{i=2}^\ell c_ic_{-i}\right)v_1$ to $v''_{-1}$ to get $v_1-1$ in position $-1$.
%Again, as in case of $\rA_n$, apply $z_\gamma(1-v_1,1)$ with $\gamma=-\alpha_\mathrm{max}$, to get $1$ as the first entry and $0$ as the last.

\textsc{Case $\Phi=\rD_\ell$, $n= 2\ell$.} 
By Lemma~\ref{lemma:asrUnip} we can find $h_1\in \U(\Sigma^+_\ell, I)$ such that the upper half $v'_+$ of $v'=h_1 \cdot v$ is $I$-unimodular.
Since $\sr(I)\leq \ell-1$ we can find $c_1$, $c_3, \ldots c_\ell \in I$ such that $(v''_1, v''_3, \ldots, v''_\ell) \in \Ums(\ell-1, I)$, where
\[ v''=h_2 \cdot x_{1,2}(c_1) \cdot v', \quad h_2=\prod_{i=3}^\ell x_{i,2}(c_i). \]

We can find $f_1, f_3,\ldots, f_\ell \in R$ such that $f_1v''_1+\sum_{i=3}^\ell f_i v''_{i} = 1$.
As before, set
\[ \xi = v''_1-v''_{-2}-1 \in I, \quad h_3 = x_{-2,1}(\xi f_1) \cdot \prod_{i=3}^\ell x_{-2,i}(\xi f_i), \quad v'''=h_3 \cdot v''. \]
%Clearly, $v'''_{-2}=v'''_1-1$, therefore for $v_4 = z_{-\alpha_{max}}(-v'''_{-2}, 1) \cdot v'''$ one has $v^4_1 = 1$, as required.
Clearly, $t_{1,2}(c_1) \cdot h_1 \in \U(\Phi^+, I)$, $ h_3 \cdot h_2 \in \U(\Phi^-, I)$ and $v'''$ satisfies \ref{eq1}.

\textsc{Case $\Phi=\rB_\ell$, $n=2\ell+1$.} Subdivide $v\in \Ums(2\ell+1, I)$ as $v=(v_+, v_0, v_-)\in R^\ell\times R\times R^\ell$.
Denote by $J\leq I$ the ideal spanned by components of $v_-$.
Since $\sr(I/J)\leq \ell$ we can find $c_1,\dots,c_\ell\in I$ such that for $v' = h \cdot v$, $h = \prod_{i=1}^\ell x_{i,0}(c_i) \in \U(\Phi^+, I)$
one has $\bar{v'}_+=(\bar{v'_1},\ldots, \bar{v'_\ell}) \in \Ums(\ell, I/J)$ and, therefore, $(v'_+, v'_-) \in \Ums(2\ell, I)$.
Now the proof can be finished by repeating the argument for the case $\Phi=\rD_\ell$ (applied to the subset of long roots of $\rB_\ell$).
%(clearly, the maximal root of $\rD_\ell$ maps to the maximal root of $\rB_\ell$ under the natural embedding $\rD_\ell\subseteq\rB_\ell$). 
\end{proof}

It is easy to see that the proof of the above theorem is effective and gives an estimate of the total number of elementary root unipotents involved in the decomposition.
\begin{cor}
In the assumptions and notation of Theorem~\ref{thm:BassKolster} every element of $\G(\Phi,R,I)$ 
can be factored into a product of one element of $\G(\Delta_1,R,I)$ one element of $Z$ and at most $4(|\Phi^+| - |\Delta_1^+|)-1$ elementary root unipotents $x_\alpha(s)$ of level $I$. \end{cor}
\begin{proof}
The above estimates can be obtained by a careful analysis of the proof of the previous theorem.
Cases $\Phi=\rA_\ell, \rC_\ell$ are immediate.
In the case $\Phi=\rD_\ell$ the proof of Theorem~\ref{thm:BassKolster} implies that
\begin{multline}\nonumber
\G(\Phi,R,I) =  \U(\Sigma_\ell,I) \cdot X_{\alpha_1}(I) \cdot \U(\Sigma_2^-\cap\Delta_1,I) \cdot X_{-\alpha\ssub{\mathrm{max}}}(I) \cdot Z  \cdot \\ \cdot \U(\Sigma_1^-,I) \cdot \U(\Sigma_1,I) \cdot \G(\Delta_1,R,I).
\end{multline}
We can present an element $g$ of $\U(\Sigma_\ell, I)$ as a product of $g_1 \in \U(\Sigma_{\{1,2\}} \cap \Sigma_\ell)$ and $g_2\in \U(\Delta_{\{1,2\}}\cap \Sigma_\ell)$.
An examination of the extended Dynkin diagram of $\rD_\ell$ implies that $g_2$ either centralizes or normalizes all factors of the above decomposition (except the last one) and therefore can be moved to the right until it is consumed by $\G(\Delta_1)$.
On the other hand, $g_1$ is a product of at most $2\ell-3$ elementary unipotents, while the width of $\U(\Sigma_1^\pm, I)$ and $\U(\Sigma_2^-\cap\Delta_1)$ in elementary unipotents does not exceed $2\ell-2$ and $2\ell-4$, respectively.
Summing up these upper bounds we obtain
$$(2\ell-3) + 1 + (2\ell - 4) + 1 + 2\cdot (2\ell - 2) = 8\ell - 9 = 4(|\rD_\ell| - |\rD_{\ell-1}|) - 1.$$

The estimate in the case $\Phi=\rB_\ell$ can be obtained in a similar way. \end{proof}

\begin{proof}[Proof of Theorem~\ref{thm:SL2width}]
Consider the case $\Phi=\rA_\ell$, $\sr(I)\leq 2$. First of all, notice that we can improve the estimate of the number of factors involved in Bass---Kolster decomposition.
Indeed, when performing the first step of the proof of Theorem~\ref{thm:BassKolster} it suffices to make only $2$ additions of $v_{\ell+1}$ (e.\,g. to $v_{1}$ and $v_2$) to make the first $\ell$ entries of $v'$ form a unimodular column.
In particular, $\G(\Phi, R, I)$ can be presented as a product of one element of $\G(\Delta_1, R, I)$, one element of $Z$ and $3\ell+1$ root unipotents $x_\alpha(s)$, $s\in I$.
The latter elements are contained in a product of $3\ell + 1$ copies of $\SL(2, R, I)$. Now the statement of the theorem follows by induction on $\ell$.

The proof in the case $\Phi=\rC_\ell$ is similar (notice that we use the exceptional isomorphism $\SL(2, R)\cong \Sp(2, R)$).
\end{proof}

\subsection{Relative Dennis-Vaserstein decompositions}\label{sec:dennis-vaserstein}
Throughout the present section we denote by $\EP_s(R, I)$ the subgroup $\E(\Delta_s, R, I) \cdot \U(\Sigma_s, I)$, $1 \leq s \leq n$.
%TODO: Add Levi decomposition to preliminaries
%TODO: Introduce notation for U(S, R)
Set $\EP_s := \EP_s(R, R) = \E(\Delta_s, R) \cdot \U(\Sigma_s, R)$. 

Let $\Phi$ be an irreducible root system of rank $\ell$.
Let $r$, $s$ be two distinct indices $1\leq r,s \leq \ell$.
From Levi decomposition it follows that
\begin{multline}\nonumber \U(\Phi^+, I)\cdot \U(\Phi^-, I) \cdot \E(\Delta_r, R, I) \cdot \EP_s(R, I) = 
\U(\Sigma_r, I)\cdot \U(\Sigma^-_r, I) \cdot \E(\Delta_r, R, I) \cdot \EP_s(R, I) = \\
= \EP_r(R, I) \cdot \E(\Delta_s, R, I) \cdot \U(\Sigma_s^-, I)\cdot \U(\Sigma_s, I) = 
\EP_r(R, I) \cdot \U(\Sigma^-_r\cap \Sigma^-_s, I) \cdot \EP_s(R, I). \end{multline}
Denote by $A_{rs}$ any of the equal subsets from the previous formula. 

\begin{thm}\label{theorem:relative_dv}
The relative elementary subgroup $\E(\Phi, R, I)$ coincides with $A_{rs}$ under the following assumptions on $(R, I)$.
  \begin{center}
    \begin{tabular}{| l | l | l | l | l |} \hline
    № & $\Phi$ & $(s,r)$ & ring condition \\ \hline
    1. & $\rA_\ell$, $\ell\geq 2$ & $(1, \ell)$ & $\sr(I) \leq \ell-1$ \\ \hline
    2. & $\rB_\ell$, $\ell\geq 3$ & $(1, \ell)$ & $\sr(I) \leq \ell-1$ \\ \hline
    3. & $\rD_\ell$, $\ell\geq 4$ & $(1, \ell)$ & $\sr(I) \leq \ell-2$ \\ \hline    
    4. & $\rE_\ell$, $\ell=6,7$ & $(\ell, 2)$ & $\sr(I) \leq \ell-3$ \\ \hline     
    5. & $\rE_\ell$, $\ell=6,7$ & $(\ell, 1)$ & $\asr(R, I)\leq \ell-2$ \\ \hline    
    \end{tabular} \end{center} 
\end{thm}
The proof of Theorem~\ref{theorem:relative_dv} occupies the rest of this section.

Consider the usual conjugation action of $\E(\Phi, R)$ on $\E(\Phi, R, I)$. 
This action induces an action of $\E(\Phi, R)$ on the set $\mathfrak{S}$ of all subsets of $\E(\Phi, R, I)$.
On the other hand, $\E(\Phi, R, I)$ acts on $\mathfrak{S}$ by left multiplication.
Denote by $N_{rs}$ and $L_{rs}$ stabilizers of $A_{rs} \in \mathfrak{S}$ with respect to these actions.
In other words $$N_{rs} = \{ g\in \E(\Phi, R) \mid g \cdot A_{rs} \cdot g^{-1} \subseteq A_{rs} \};\quad L_{rs}= \{ g\in \E(\Phi, R, I) \mid g \cdot A_{rs} \subseteq A_{rs} \}.$$

It is easy to see that $N_{rs}$ normalizes $L_{rs}$. Indeed, for $g\in N_{rs}$, $h\in L_{rs}$ one has
$$h^g \cdot A_{rs} = g^{-1} \cdot h \cdot g \cdot A_{rs} \subseteq g^{-1} \cdot h \cdot A_{rs} \cdot g \subseteq A_{rs}^g \subseteq A_{rs}.$$

\begin{lemma}\label{lemma:dv_unipotent} For any $1\leq i\leq n$ the following statements hold. \begin{enumerate} 
\item $\U(\Phi^\pm, I) = X_{\pm\alpha\ssub{i}}(I)\cdot \U(\Phi^\pm\setminus\{\pm\alpha\ssub{i}\}, I) = \U(\Phi^\pm\setminus\{\pm\alpha\ssub{i}\}, I)\cdot X_{\pm\alpha\ssub{i}}(I).$
\item For any $\xi\in R$ one has $\U(\Phi^\pm\setminus\{\alpha_i\}, I)^{x_{\mp\alpha\ssub{i}}(\xi)^{-1}} \subseteq \U(\Phi^\pm, I).$
\item $\U(\Phi^+, I)\cdot \U(\Phi^-, I) \subseteq \U(\Phi^+\setminus \{\alpha_i\}, I) \cdot \U(\Phi^-, I) \cdot X_{\alpha\ssub{i}}(I) \cdot X_{-\alpha\ssub{i}}(I)$.
\end{enumerate} \end{lemma}
\begin{proof}
 The first two statements easily follow from Chevalley commutator formula while the third one is a formal consequence of the first two.
\end{proof}

The following lemma is a relative version of the main reduction used by M.~Stein in~\cite{St78} for the proof of the absolute Dennis--Vaserstein decomposition.
\begin{lemma}\label{lemma:Stein_reduction}
Assume that there exists a subset $\widetilde{L_r} \subseteq \E(\Delta_r, R, I)$ with the following properties:
\begin{enumerate}[label=(\alph*)] 
 \item\label{stein_cond1} One has $\U(\Sigma^+_r, I)\cdot \U(\Sigma^-_r, I) \cdot \E(\Delta_r, R, I) \subseteq \U(\Phi^+, I)\cdot \U(\Phi^-, I) \cdot \widetilde{L_r}.$
 \item\label{stein_cond2} One has $X_{-\alpha\ssub{r}}(I)^{\widetilde{L_r}} \subseteq \EP_s(R, I).$
\end{enumerate}
Then $X_{-\alpha_r}(I) \subseteq L_{rs}$ and $\EP_s \subseteq N_{rs}.$
\end{lemma}
\begin{proof} Set $A:=\U(\Phi^+, I)\cdot \U(\Phi^-, I) \cdot \widetilde{L_r} \cdot \EP_s(\Phi, R, I).$
From condition~\ref{stein_cond1} of the lemma it follows that $A_{rs}=A$.
Observe that from Lemma~\ref{lemma:dv_unipotent} and condition~\ref{stein_cond2} it follows that
\begin{multline}\nonumber 
A \subseteq \U(\Phi^+\setminus \{\alpha_r\}, I) \cdot \U(\Phi^-, I) \cdot X_{\alpha\ssub{r}}(I) \cdot X_{-\alpha\ssub{r}}(I) \cdot \widetilde{L_r} \cdot \EP_s(R, I) \subseteq \\ 
\subseteq \U(\Phi^+\setminus\{\alpha_r\}, I) \cdot \U(\Phi^-, I) \cdot X_{\alpha\ssub{r}}(I) \cdot X_{-\alpha\ssub{r}}(I) \cdot \widetilde{L_r} \cdot \EP_s(R, I) \subseteq \\
\subseteq \U(\Phi^+\setminus\{\alpha_r\}, I) \cdot \U(\Phi^-, I) \cdot \widetilde{L_r} \cdot \U(\Sigma_r, I) \cdot \EP_s(R, I)  \cdot \EP_s(R, I) \subseteq \\
\subseteq \U(\Phi^+\setminus\{\alpha_r\}, I) \cdot \U(\Phi^-, I) \cdot \widetilde{L_r} \cdot \EP_s(R, I). \end{multline}
Applying Lemma~\ref{lemma:dv_unipotent} we get that:
\begin{equation}\nonumber A^{X_{-\alpha\ssub{r}}} \subseteq \U(\Phi^+, I) \cdot \U(\Phi^-, I) \cdot \widetilde{L_r} ^{X_{-\alpha\ssub{r}}} \cdot \EP_s(R, I) \subseteq A. \end{equation}
\begin{equation}\nonumber X_{-\alpha\ssub{r}}(I) \cdot A \subseteq \U(\Phi^+, I) \cdot X_{-\alpha\ssub{r}}(I) \cdot \U(\Phi^-, I) \cdot \widetilde{L_r} \cdot \EP_s(R, I) = A. \end{equation}
To prove the second part of the statement observe first that $\EP_s$ is generated by $X_{\alpha\ssub{i}}$ for $1\leq i\leq n$ and $X_{-\alpha\ssub{i}}$ for $i\neq s$.
%TODO: Find proper reference for this fact
We have just shown that $X_{-\alpha_r}\subseteq N_{rs}$.
On the other hand, inclusions $X_{\alpha\ssub{k}} \subseteq N_{rs}$ for $\ 1\leq k\leq \ell$ and $X_{-\alpha\ssub{k}} \subseteq N_{rs}$ for $k\neq r,s$ are obvious.
\end{proof}

\begin{proof}[Proof of Theorem~\ref{theorem:relative_dv}]
We first show that under specified assumptions on $(R, I)$ one can meet the conditions of Lemma~\ref{lemma:Stein_reduction}.
Consider the following two subsets of $\Lambda(\pi)$:
$$\Gamma = \varpi_s- (\Sigma_s^+\cap \Delta_r),\quad \Gamma_0 = \{\lambda \in \Gamma \mid \lambda - \alpha_r \in \Lambda(\pi) \}.$$
Clearly, $\Gamma$ is the set of weights of an irreducible representation of $\G(\Delta_r, R)$ corresponding to the same highest weight $\varpi_s$.
The subsystem $\Delta_r$ has type $\rA_{\ell-1}$ in all cases except the last one.
It is also clear that $|\Gamma_0|=1$ for $\Phi=\rA_\ell, \rB_\ell$, $|\Gamma_0|=2$ for $\Phi=\rD_\ell$, $|\Gamma_0|=3$ for $\Phi=\rE_\ell$ and $r=2$.
In the case $\Phi=\rE_\ell$, $r=2$ the subsystem $\Delta_r$ has type $\rD_{\ell-1}$ and $|\Gamma_0|=\ell-1$.

Let $\widetilde{L_r}$ be the set of all elements $g$ of $\E(\Delta_r,R, I)$ such that $(g \cdot v^+)_\lambda = 0$ for $\lambda\in\Gamma_0$.
In any of specified cases the assumption on $(R, I)$ allows us to apply Lemma~\ref{lemma:uraction} to the subsystem $\Delta_r$ and find
$x\in\U(\Delta_r\cap\Phi^+, I)$, $y\in \U(\Delta_r\cap\Phi^-, I)$ such that $yx\cdot g \in \widetilde{L_r}$.
This proves the first condition of Lemma~\ref{lemma:Stein_reduction}, indeed:
$$ \U(\Sigma^+_r, I)\cdot \U(\Sigma^-_r, I) \cdot g = \U(\Sigma^+_r, I) x^{-1} \cdot \U(\Sigma^-_r, I)^{x^{-1}} y^{-1} \cdot (yxg) \subseteq \U(\Phi^+, I)\cdot \U(\Phi^-, I) \cdot \widetilde{L_r}.$$
To prove the second condition notice that by the definition of $\Gamma_0$ for any $s\in I$, $ g\in\widetilde{L_r}$ one has $x_{-\alpha_r}(s) \cdot g \cdot v^+ = g \cdot v^+$ and, therefore,
$$X_{-\alpha\ssub{r}}(I)^{\widetilde{L_r}} \subseteq \U(\Phi^-, I) \cap \Stab(v^+) \subseteq \E(\Delta_s, R, I) \subseteq \EP_s(R, I).$$

From now on we can assume that the statement of Lemma~\ref{lemma:Stein_reduction} holds and $\EP_s$ normalizes $A_{rs}$.
Notice that in view of Theorem~\ref{theorem:Stepanov} it suffices to show that the following two families of elements are contained in $L_{rs}$:
\begin{itemize} \item $z_{\alpha}(s, \xi)$, $s\in I$, $\xi \in R$, $\alpha\in\Sigma^-_s$;
\item $x_{\beta}(s)$, $s \in I$, $\beta \in \Phi$. \end{itemize}
Since $\EP_s \subseteq N_{rs}$ it suffices to check inclusions only for the second family of elements.
We already know that $\U^+(\Phi, I) \subseteq L_{rs}$.

Notice that the Weyl group $W(\Delta_s)$ acts transitively on $\Delta_s$, therefore in view of relation~\ref{rel:R3} the subgroup
$W_s := \langle w_\alpha(1) \mid \alpha\in\Delta_s\rangle \leq \EP_s$ acts transitively on the set of root subgroups $X_\alpha(I)$, $\alpha\in \Delta_s$.
Since $X_{-\alpha\ssub{r}}(I) \subseteq L_{rs}$ we get that that $\U(\Delta_s \cap \Phi^-, I)\subseteq L_{rs}$.

Now denote by $\widetilde{\alpha}$ the maximal root of $\Phi$. Our assumptions on $\Phi$ guarantee that $m_s(\widetilde{\alpha})=1$, 
and, consequently, every two roots $\alpha, \beta \in \Sigma^-_s$ have the same $s$-shape (i.\,e. $\shape(\{s\}, \alpha = \shape(\{s\}, \beta)$).
By Lemma~\ref{lemma:abs} $W(\Delta_s)$ interchanges $\alpha$ and $\beta$ if their length is equal (which is the case if we assume additionally $\Phi\neq \rB_\ell$).
Since $X_{\alpha\ssub{s}} \subseteq L_{rs}$ the argument similar to the one above implies $\U(\Sigma^-_s, I)\subseteq L_{rs}$.
This completes the proof of the theorem for $\Phi\neq \rB_\ell$. 
%TODO:
\textbf{TODO: Finish the proof for $\Phi=\rB_\ell$.}
\end{proof}


\section{Applications}\label{sec:applications}
\subsection{Bounded generation}\label{sec:boundgen}

\subsection{Relative stable rank 1}
\begin{lemma}\label{lemma:srRI1}
If $sr(R,I)=1$, the width of $SL(2,R,I)$ with respect to $z_\alpha$ does not exceed $4$.
\end{lemma}
\begin{proof}

Let $A=\begin{pmatrix}a & b \\ c & d\end{pmatrix}\in SL(2,R,I)$. First column is $I$-unimodular, therefore there exists $z\in I$ such that $a+cz\in R^*$. Multiply $A$ by $x_{12}(z)$ from left to obtain invertible element in the upper left corner. Applying $x_{21}(-c/a)$ from left and $x_{12}(-b/a)$ from right, we obtain a diagonal matrix. Thus
\begin{multline*}
A=x_{12}(-z)\cdot x_{21}(c/a)\cdot
\begin{pmatrix} \varepsilon & 0 \\ 0 & 1/\varepsilon \end{pmatrix}
\cdot x_{12}(b/a)=\\
=x_{12}(-z)\cdot
\begin{pmatrix} \varepsilon & 0 \\ 0 & 1/\varepsilon \end{pmatrix}
\cdot x_{21}(y) \cdot x_{12}(b/a),
\end{multline*}
where all the coefficients in transvections lie in $I$ and $\varepsilon\in 1+I$. The above formula is a relative version of Gauss decomposition.
Note that
\begin{multline*}
\begin{pmatrix} \varepsilon & 0 \\ 0 & 1/\varepsilon \end{pmatrix} =
\begin{pmatrix} 1 & -1 \\ 0 & 1 \end{pmatrix}
\begin{pmatrix} 1 & 0 \\ 1-\varepsilon & 1 \end{pmatrix}
\begin{pmatrix} 1 & 1/\varepsilon \\ 0 & 1 \end{pmatrix}
\begin{pmatrix} 1 & 0 \\ \varepsilon^2-\varepsilon & 1 \end{pmatrix} =\\=
\begin{pmatrix} 1 & -1 \\ 0 & 1 \end{pmatrix}
\begin{pmatrix} 1 & 0 \\ 1-\varepsilon & 1 \end{pmatrix}
\begin{pmatrix} 1 & 1+z \\ 0 & 1 \end{pmatrix}
\begin{pmatrix} 1 & 1/\varepsilon-1-z \\ 0 & 1 \end{pmatrix}
\begin{pmatrix} 1 & 0 \\ \varepsilon^2-\varepsilon & 1 \end{pmatrix},
\end{multline*}
therefore
\[
A=z_{21}(1-\varepsilon,-1-z)\cdot x_{12}(1/\varepsilon-1-z)\cdot x_{21}(\varepsilon^2-\varepsilon+y)\cdot x_{12}(b/a). \qedhere
\]
\end{proof}

\subsection{Dedekind rings of arithmetic type}
\subsubsection{Number field case}
\paragraph{$\mathbb{Z}[\sfrac{1}{p}]$.}
\textbf{TODO:} Insert here Moree's lemma on primitive roots
\begin{lemma}\label{lemma:Z1p}
Assume GRH. If $R=\mathbb{Z}[\sfrac{1}{p}]$ and $I\lhd R$, the width of $SL(2,R,I)$ does not exceed $6$.
\end{lemma}
\begin{proof}
$I\lhd\mathbb{Z}[\sfrac{1}{p}]$, $I=(m)$, $m\in\mathbb{Z}$, $p\nmid m$.

\[ g=\begin{pmatrix}
x & y \\ z & w
\end{pmatrix},\quad x =p^\alpha a,\quad y =p^\beta bm, \]
where $a,b\in\mathbb{Z}$, $p\nmid a,b$ and $\alpha,\beta\in\mathbb{Z}$.

\textsc{Case 1:} $\alpha\geqslant\beta$. Since $p^{\alpha-\beta}a\perp bm^2$ and $p\perp bm^2$, there exist infinitely many rational primes $q$ of the form $p^{\alpha-\beta}a+bm^2k$, such that $p$ is a primitive root modulo $q$. One may choose $q$ prime to $b$. Then
\[ g_1=g\cdot x_{21}(mk) =
\begin{pmatrix} p^\beta q & p^\beta bm \\ * & * \end{pmatrix}.\]
$\exists u\geqslant 1: p^u\equiv b\mod q$, say $p^u=b+lq$. Then
\[ g_2 = g_1\cdot x_{12}(ml) =
\begin{pmatrix} p^\beta q & mp^{\beta+u} \\ * & * \end{pmatrix}. \]
$g_2\equiv 1\mod m$, thus $p^\beta q=1+cm$.
\begin{align*}
g_3 = & g_2\cdot x_{21}\left(\dfrac{-c}{p^{\beta+u}}\right) =
\begin{pmatrix} 1 & mp^{\beta+u} \\ * & * \end{pmatrix}, \\
g_4 = & g_3\cdot x_{12}(-mp^{\beta+u}) =
\begin{pmatrix} 1 & 0 \\ * & * \end{pmatrix}, \\
g_5 = & g_4\cdot x_{21}\left(\dfrac{c}{p^{\beta+u}}\right) =
\begin{pmatrix} 1 & 0 \\ * & * \end{pmatrix}.
\end{align*}
Note that $g_5=g_2\cdot z_{12}\left(-mp^{\beta+u},\dfrac{c}{p^{\beta+u}}\right)$ and that two other coefficients in transvections are multiples of $m$. Thus in this case the length of $g$ does not exceed $4$: $g=x_{21}z_{12}x_{12}x_{21}$.

\textsc{Case 2:} $\alpha<\beta$. Since $\mathbb{Z}[\sfrac{1}{p}]/I$ is finite, $\exists k>0$ such that $p^k\equiv 1\mod I$. One can choose $k>\beta-\alpha$. Then $k+\alpha>-k+\beta$ and
\[ g\cdot h(p^k) =
\begin{pmatrix} p^\alpha a & p^\beta bm \\ * & * \end{pmatrix}
\begin{pmatrix} p^k & 0 \\ 0 & p^{-k} \end{pmatrix}=
\begin{pmatrix} p^{k+\alpha} a & p^{-k+\beta} bm \\ * & * \end{pmatrix}.
\]
One can apply Case 1 fot the latter, so
$g=x_{21}z_{12}x_{12}x_{21}h^{-1}$,
and $x_{12}h^{-1}$ can be processed in the same way, as in Lemma \ref{lemma:srRI1}:
$g=x_{21}z_{12}x_{12}\cdot z_{12}x_{21}x_{12}$.
\end{proof}
\subsection{Subsystem factorizations}\label{sec:subsysfact}
The main result of \cite{NikProdDecomp} is the following
\begin{thm*}
Let $G$ be a classical (possibly twisted) Chevalley group of rank $n$ over a finite field. Then $G$ equals the product of at most $200$ conjugates of an $\SL_n$ subgroup.
\end{thm*}
As indicated in the introduction, the Dennis---Vaserstein decomposition gives the following result.
\begin{lemma}
Assume $\sr(I)\leqslant n-1$. Then $\SL(n+1,R,I)$ is a product of at most $5$ subgroups isomorphic to $\SL(n,R,I)$.
\end{lemma}
\begin{proof}
By the Dennis---Vaserstein decomposition one can present $\SL(n+1,R,I)$ as a product
\[ \SL(n+1,R,I) =  \Par_1\cdot X_{n1}\cdot\Par_n=S_1\U_1\cdot X_{n1}\cdot\U_n S_n, \]
where $S_1$ and $S_n$ are two obvious embeddings of $\SL(n,R,I)$ in $\SL(n+1,R,I)$, avoiding respectively the first and the last row and column. Now $\U_1=(\U_1\cap S_n)\cdot X_{1n}$ and $U_n= X_{1n}\cdot(U_n\cap S_1)$, while $X_{1n}X_{n1}X_{1n}\in S_1^{w_{12}(1)}$.
\end{proof}
We will now elaborate on the case $\Phi=\rD_\ell$ to show that the Dennis---Vaserstein decomposition is suitable for handling other Chevalley groups, albeit with a stronger assumption on the base ring. The following lemma is an analogue of Proposition~1 of~\cite{NikProdDecomp}.
\begin{lemma}
Let $\Phi=\rD_\ell$. There exist an element $y\in\G(\Phi,R)$ and an element $w\in\widetilde{W}(\Phi)$ such that $\U(\Sigma_\ell)\subset[\U(\Delta_\ell^-),y]\cdot{}^w\U(\Delta_\ell^+)$.
\end{lemma}
\begin{proof}
Set $\beta_i = \alpha_{2i-1} + 2\alpha_{2i}+ \ldots + 2\alpha_{\ell-2} + \alpha_{\ell-1} + \alpha_\ell$, for $i=1,\ldots,\lfloor\ell/2\rfloor$, i.\,e. $\beta_i$ form a maximal set of pairwise orthogonal maximal roots in some subsystems of type $\rD_{k}$. Denote $B=\{\beta_i\}$, then decompose $\U(\Sigma_\ell)=\U(\Sigma_\ell\setminus B)\cdot\U(B)$.

Set $y=\prod_{\beta\in B}x_\beta(1)$. We will now show that $\U(\Sigma_\ell\setminus B)\subset[\U(\Delta_\ell^-),y]\cdot\U(B)$.

We first note that
\[ \bigl[\U(\Delta_2^-),x_{\beta_1}(1)\bigr]=1,\quad \Bigl[\U(\Sigma_2^-\cap\Delta_\ell),\prod_{i\neq1}x_{\beta_i}(1)\Bigr]=1. \]
This implies
\[ \Bigl[ \U(\Sigma_2^-\cap\Delta_\ell)\cdot\U(\Delta_{2,\ell}^-), x_{\beta_1}(1)\cdot\prod_{i\neq1}x_{\beta_i}(1) \Bigr] \equiv \bigl[ \U(\Sigma_2^-\cap\Delta_\ell), x_{\beta_1}(1) \bigr] \bmod \U(\Sigma_\ell\cap\Delta_2). \]
Take an element $u\in\U(\Sigma_2^-\cap\Delta_\ell)$ and decompose it as $u=vw$, $v\in\U(\Sigma_{1,2}^-\cap\Delta_\ell)$, $w\in\U(\Sigma_2^-\cap\Delta_{1,\ell})$, then, using the identity $[ab,c]={}^a[b,c]\cdot[a,c]$, rewrite
\[ [vw,x_{\beta_1}(1)] = {}^v[w,x_{\beta_1}(1)]\cdot[v,x_{\beta_1}(1)].  \]
Since $\U(\Sigma_{1,2}^-\cap\Delta_\ell)$ and $\U(\Sigma_2^-\cap\Delta_{1,\ell})$ are abelian, it is easy to see that
\[ [v,x_{\beta_1}(1)]\in\U(\Sigma_{2,\ell}\cap\Delta_1), \quad [w,x_{\beta_1}(1)]\in\U(\Sigma_{1,\ell}\setminus\{\beta_1\}), \]
and varying $v$ and $w$ one can obtain any element of $\U(\Sigma_{2,\ell}\cap\Delta_1)$ and $\U(\Sigma_{1,\ell}\setminus\{\beta_1\})$ respectively. By the Levi decomposition
\[ {}^v\U(\Sigma_{1,\ell}\setminus\{\beta_1\}) \equiv \U(\Sigma_{1,\ell}\setminus\{\beta_1\}) \bmod \U(\Sigma_\ell\cap\Delta_2). \]
Thus we have shown that
\[ [\U(\Sigma_2^-)\cdot\U(\Delta_{2,\ell}),y] \equiv \U(\Sigma_{1,\ell}\cup\Sigma_{2,\ell}\setminus\{\beta_1\}) \bmod \U(\Sigma_\ell\cap\Delta_2), \]
and this reduces the problem to the $\rD_{\ell-2}$-subsystem $\Delta_{1,2}$. So we can carry the induction on the rank of the subsystem, and construct for any given $a\in\U(\Sigma_\ell\setminus B)$ an element $b\in\U(\Delta_\ell^-)$ such that $a\in[b,y]\cdot\prod_{\beta\in B}X_\beta\subset[\U(\Delta_\ell^-),y]\cdot\U(B)$.
\end{proof}

\printbibliography

\end{document}
