The main result of \cite{NikProdDecomp} is the following
\begin{thm*}
Let $G$ be a classical (possibly twisted) Chevalley group of rank $n$ over a finite field. Then $G$ equals the product of at most $200$ conjugates of an $\SL_n$ subgroup.
\end{thm*}
As indicated in the introduction, the Dennis---Vaserstein decomposition gives the following result.
\begin{lemma}
Assume $\sr(I)\leqslant n-1$. Then $\SL(n+1,R,I)$ is a product of at most $5$ subgroups isomorphic to $\SL(n,R,I)$.
\end{lemma}
\begin{proof}
By the Dennis---Vaserstein decomposition one can present $\SL(n+1,R,I)$ as a product
\[ \SL(n+1,R,I) =  \Par_1\cdot X_{n1}\cdot\Par_n=S_1\U_1\cdot X_{n1}\cdot\U_n S_n, \]
where $S_1$ and $S_n$ are two obvious embeddings of $\SL(n,R,I)$ in $\SL(n+1,R,I)$, avoiding respectively the first and the last row and column. Now $\U_1=(\U_1\cap S_n)\cdot X_{1n}$ and $U_n= X_{1n}\cdot(U_n\cap S_1)$, while $X_{1n}X_{n1}X_{1n}\in S_1^{w_{12}(1)}$.
\end{proof}
We will now elaborate on the case $\Phi=\rD_\ell$ to show that the Dennis---Vaserstein decomposition is suitable for handling other Chevalley groups, albeit with a stronger assumption on the base ring. The following lemma is an analogue of Proposition~1 of~\cite{NikProdDecomp}.
\begin{lemma}
Let $\Phi=\rD_\ell$. There exist an element $y\in\G(\Phi,R)$ and an element $w\in\widetilde{W}(\Phi)$ such that $\U(\Sigma_\ell)\subset[\U(\Delta_\ell^-),y]\cdot{}^w\U(\Delta_\ell^+)$.
\end{lemma}
\begin{proof}
Set $\beta_i = \alpha_{2i-1} + 2\alpha_{2i}+ \ldots + 2\alpha_{\ell-2} + \alpha_{\ell-1} + \alpha_\ell$, for $i=1,\ldots,\lfloor\ell/2\rfloor$, i.\,e. $\beta_i$ form a maximal set of pairwise orthogonal maximal roots in some subsystems of type $\rD_{k}$. Denote $B=\{\beta_i\}$, then decompose $\U(\Sigma_\ell)=\U(\Sigma_\ell\setminus B)\cdot\U(B)$.

Set $y=\prod_{\beta\in B}x_\beta(1)$. We will now show that $\U(\Sigma_\ell\setminus B)\subset[\U(\Delta_\ell^-),y]\cdot\U(B)$.

We first note that
\[ \bigl[\U(\Delta_2^-),x_{\beta_1}(1)\bigr]=1,\quad \Bigl[\U(\Sigma_2^-\cap\Delta_\ell),\prod_{i\neq1}x_{\beta_i}(1)\Bigr]=1. \]
This implies
\[ \Bigl[ \U(\Sigma_2^-\cap\Delta_\ell)\cdot\U(\Delta_{2,\ell}^-), x_{\beta_1}(1)\cdot\prod_{i\neq1}x_{\beta_i}(1) \Bigr] \equiv \bigl[ \U(\Sigma_2^-\cap\Delta_\ell), x_{\beta_1}(1) \bigr] \bmod \U(\Sigma_\ell\cap\Delta_2). \]
Take an element $u\in\U(\Sigma_2^-\cap\Delta_\ell)$ and decompose it as $u=vw$, $v\in\U(\Sigma_{1,2}^-\cap\Delta_\ell)$, $w\in\U(\Sigma_2^-\cap\Delta_{1,\ell})$, then, using the identity $[ab,c]={}^a[b,c]\cdot[a,c]$, rewrite
\[ [vw,x_{\beta_1}(1)] = {}^v[w,x_{\beta_1}(1)]\cdot[v,x_{\beta_1}(1)].  \]
Since $\U(\Sigma_{1,2}^-\cap\Delta_\ell)$ and $\U(\Sigma_2^-\cap\Delta_{1,\ell})$ are abelian, it is easy to see that
\[ [v,x_{\beta_1}(1)]\in\U(\Sigma_{2,\ell}\cap\Delta_1), \quad [w,x_{\beta_1}(1)]\in\U(\Sigma_{1,\ell}\setminus\{\beta_1\}), \]
and varying $v$ and $w$ one can obtain any element of $\U(\Sigma_{2,\ell}\cap\Delta_1)$ and $\U(\Sigma_{1,\ell}\setminus\{\beta_1\})$ respectively. By the Levi decomposition
\[ {}^v\U(\Sigma_{1,\ell}\setminus\{\beta_1\}) \equiv \U(\Sigma_{1,\ell}\setminus\{\beta_1\}) \bmod \U(\Sigma_\ell\cap\Delta_2). \]
Thus we have shown that
\[ [\U(\Sigma_2^-)\cdot\U(\Delta_{2,\ell}),y] \equiv \U(\Sigma_{1,\ell}\cup\Sigma_{2,\ell}\setminus\{\beta_1\}) \bmod \U(\Sigma_\ell\cap\Delta_2), \]
and this reduces the problem to the $\rD_{\ell-2}$-subsystem $\Delta_{1,2}$. So we can carry the induction on the rank of the subsystem, and construct for any given $a\in\U(\Sigma_\ell\setminus B)$ an element $b\in\U(\Delta_\ell^-)$ such that $a\in[b,y]\cdot\prod_{\beta\in B}X_\beta\subset[\U(\Delta_\ell^-),y]\cdot\U(B)$.
\end{proof}