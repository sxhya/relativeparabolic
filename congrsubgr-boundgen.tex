We now give an application of parabolic factorizations to the problem of bounded generation.
\begin{dfn} 
Let $G$ be an abstract group with a generating set $X \subset G$. %(for simplicity we assume $X=X^{-1}$ and $1\in X$),
One defines {\it the width of $G$ with respect to $X$} as the smallest natural number $N$ such that 
 every element of $G$ is a product of at most $N$ elements of $X$, i.\,e. $G=X^N$. \end{dfn}
 
\begin{lemma}\label{lemma:srRI1}
Suppose that $\sr(I)=1$, then the width of $\SL(2,R,I)$ with respect to generators $z_\alpha(s, \xi)$ of Theorem~\ref{theorem:Tits-Vaserstein} does not exceed $4$.
\end{lemma}
\begin{proof}
Let $A=\begin{psmallmatrix}a & b \\ c & d\end{psmallmatrix}\in\SL(2,R,I)$ be an element of $\SL(2, R, I)$.
The first column of $A$ is $I$-unimodular, therefore there exists $z\in I$ such that $a+cz\in R^*$.
Multiplying $A$ on the left by $x_{12}(z)$ we get matrix $A'=x_{12}(z)\cdot A=\begin{psmallmatrix}a' & b' \\ c & d\end{psmallmatrix}$ with invertible top-left corner element $a'$.
After multiplying $A'$ on the left by $x_{21}(-c/a')$ and on the right by $x_{12}(-b'/a')$ we obtain a diagonal matrix. 
Thus we obtain relative Gauss decomposition of $A$
\begin{equation}\nonumber
A=x_{12}(-z)\cdot x_{21}(c/a')\cdot
\begin{pmatrix} \varepsilon & 0 \\ 0 & 1/\varepsilon \end{pmatrix}
\cdot x_{12}(b'/a')=x_{12}(-z)\cdot
\begin{pmatrix} \varepsilon & 0 \\ 0 & 1/\varepsilon \end{pmatrix}
\cdot x_{21}(y) \cdot x_{12}(b'/a'),
\end{equation}
where $\varepsilon\in 1+I$ and $y\in I$. Direct calculation shows that
\begin{multline*}
\begin{pmatrix} \varepsilon & 0 \\ 0 & 1/\varepsilon \end{pmatrix} =
\begin{pmatrix} 1 & -1 \\ 0 & 1 \end{pmatrix}
\begin{pmatrix} 1 & 0 \\ 1-\varepsilon & 1 \end{pmatrix}
\begin{pmatrix} 1 & 1/\varepsilon \\ 0 & 1 \end{pmatrix}
\begin{pmatrix} 1 & 0 \\ \varepsilon^2-\varepsilon & 1 \end{pmatrix} =\\=
\begin{pmatrix} 1 & -1 \\ 0 & 1 \end{pmatrix}
\begin{pmatrix} 1 & 0 \\ 1-\varepsilon & 1 \end{pmatrix}
\begin{pmatrix} 1 & 1+z \\ 0 & 1 \end{pmatrix}
\begin{pmatrix} 1 & 1/\varepsilon-1-z \\ 0 & 1 \end{pmatrix}
\begin{pmatrix} 1 & 0 \\ \varepsilon^2-\varepsilon & 1 \end{pmatrix},
\end{multline*}
therefore
\[ A=z_{21}(1-\varepsilon,-1-z)\cdot x_{12}(1/\varepsilon-1-z)\cdot x_{21}(\varepsilon^2-\varepsilon+y)\cdot x_{12}(b'/a'). \qedhere \]
\end{proof}

\begin{rem}
Notice tht the proof of the above Lemma also shows that the width of $\SL(2, R, I)$ respect to generators of the form $x_\alpha(s)$ and $z_\alpha(s,1)$, $s\in I$ does not exceed $5$.
This estimate is useful since $X_\alpha(I)^{x_{-\alpha}(1)} = \left\{z_\alpha(s,1)\mid s\in I\right\}$ are $1$-parameter subgroups.
\end{rem}

The following lemma is a corollary of Theorems~5.7 and 5.8 of \cite{LenMorStePrimitiveRoot}.
\begin{lemma}
Let $p$ be a rational prime, $c\perp d$ two coprime integers and $p\perp d$.
Then under the assumption of the Generalized Riemann Hypothesis there exist infinitely many primes $q\equiv c\pmod{d}$ such that $p$ is a primitive root modulo $q$.
\end{lemma}

One can modify Lemma~6 of \cite{VavSmSuUnitrEng} in the following way (cf. also \cite{VseUnitrZ1p}):

\begin{lemma}\label{lemma:Z1p}
Assume that GRH holds. If $R=\mathbb{Z}[\sfrac{1}{p}]$ and $I\lhd R$, then the width of $\SL(2,R,I)$ does not exceed $6$.
\end{lemma}

\begin{proof}
Write $I=(m)$ for some $m\in\mathbb{Z}$, $p\nmid m$. Fix $g\in\SL(2,R,I)$:
\[ g=\begin{pmatrix}x & y \\ z & w\end{pmatrix},\quad
\begin{matrix*}[l]
x=p^\alpha a,\quad & a,b,\alpha,\beta\in\mathbb{Z}, \\ y=p^\beta bm, & p\nmid a,b.
\end{matrix*}\]
\textsc{Case 1:} $\alpha\geqslant\beta$. Since $p^{\alpha-\beta}a\perp bm^2$ and $p\perp bm^2$, there exist infinitely many rational primes $q$ of the form $p^{\alpha-\beta}a+bm^2k$, such that $p$ is a primitive root modulo $q$. One may choose $q$ prime to $b$. Then
\[ g_1=g\cdot x_{21}(mk) =
\begin{pmatrix} p^\beta q & p^\beta bm \\ * & * \end{pmatrix}.\]
There exists $u\geqslant 1$ such that $p^u\equiv b\pmod q$, say $p^u=b+lq$. Then
\[ g_2 = g_1\cdot x_{12}(ml) =
\begin{pmatrix} p^\beta q & mp^{\beta+u} \\ * & * \end{pmatrix}. \]
$g_2\equiv 1\pmod m$, thus $p^\beta q=1+cm$ for some $c$.
\begin{align*}
g_3 = & g_2\cdot x_{21}\left(\dfrac{-c}{p^{\beta+u}}\right) =
\begin{pmatrix} 1 & mp^{\beta+u} \\ * & * \end{pmatrix}, \\
g_4 = & g_3\cdot x_{12}\left(-mp^{\beta+u}\right) =
\begin{pmatrix} 1 & 0 \\ * & * \end{pmatrix}, \\
g_5 = & g_4\cdot x_{21}\left(\dfrac{c}{p^{\beta+u}}\right) =
\begin{pmatrix} 1 & 0 \\ * & * \end{pmatrix}.
\end{align*}
Note that $g_5=g_2\cdot z_{12}\left(-mp^{\beta+u},c/p^{\beta+u}\right)$ and that the two other coefficients in the transvections are multiples of $m$. Thus in this case the length of $g$ does not exceed $4$: $g=x_{21}z_{12}x_{12}x_{21}$.

\textsc{Case 2:} $\alpha<\beta$. Since $\mathbb{Z}[\sfrac{1}{p}]/I$ is finite, there exists $k>0$ such that $p^k\equiv 1\pmod I$. One can choose $k>\beta-\alpha$. Then $k+\alpha>-k+\beta$ and
\[ g\cdot h\left(p^k\right) =
\begin{pmatrix} p^\alpha a & p^\beta bm \\ * & * \end{pmatrix}
\begin{pmatrix} p^k & 0 \\ 0 & p^{-k} \end{pmatrix}=
\begin{pmatrix} p^{k+\alpha} a & p^{-k+\beta} bm \\ * & * \end{pmatrix}.
\]
One can apply Case 1 for the latter, so
$g=x_{21}z_{12}x_{12}x_{21}h^{-1}$,
and $x_{12}h^{-1}$ can be processed in the same way, as in Lemma \ref{lemma:srRI1}:
$g=x_{21}z_{12}x_{12}\cdot z_{12}x_{21}x_{12}$.
\end{proof}
Let $\mathcal{O}_S$ be a Dedekind ring of arithmetic type, defined by a finite set $S$ of places of some global field $k$, and $I$ an ideal in $\mathcal{O}_S$. In case $k$ has a real embedding, then $\K_1(\Phi,\mathcal{O}_S,I)=1$ for any ideal $I\trianglelefteq\mathcal{O}_S$ and any root system $\Phi$ of rank $\geqslant2$ (see~\cite{BassMilnorSerre}). It is shown in \cite{TavgenThesis} that $E(\Phi,I)$ has finite width with respect to $\{x_\alpha(s)\mid\alpha\in\Phi,\ s\in I\}$. From this one immediately gets the bounded generation for $\G(\Phi,R,I)$.

We will use the following
\begin{lemma}[{\cite[Corollary~3.3]{S}}]
Let $\Phi$ be a root system of rank $\geqslant2$, $R$ a commutative ring and $I\trianglelefteq R$ its ideal. If $\Phi\neq\rC_\ell$, then $\E\left(\Phi,R,I^2\right)\leqslant\E(\Phi,I)$, otherwise, $\E\left(\Phi,R,II^{\indexbox{2}}\right)\leqslant\E(\Phi,I)$.
\end{lemma}
Here $I^{\indexbox{2}}$ stand for the ideal generated by all squares. So $I^2$ is generated by the elements of the form $ab$ for all $a,b\in I$, while $II^{\indexbox{2}}$ is generated by all elements of the form $a^2b$ for $a,b\in I$.
\begin{lemma}
If $\rk\Phi\geqslant2$ and $k$ has a real embedding, then $\G(\Phi,\mathcal{O}_S,I)$ has finite width with respect to $z_\alpha$.
\end{lemma}
\begin{proof}
Because of the Bass---Kolster decomposition it suffices to express every element $g$ of $\SL(2,\mathcal{O}_S,I)$ as a product of a bounded number of generators in a group of rank $2$. Write
\[ g=\begin{pmatrix}1+a & b \\ c & 1+d\end{pmatrix}\in\SL(2,\mathcal{O}_S,I),\quad a,b,c,d\in I. \]

The condition $\det(g)=1$ implies $a+d=bc-ad\in I^2$.

Define the Vaserstein's congruence subgroup as
\[ G(I,I)=\left\{ \begin{pmatrix}1+a & b \\ c & 1+d\end{pmatrix}\in\SL(2,\mathcal{O}_S)\;\middle|\; a,d\in I^2,\ b,c\in I \right\}. \]
Note that $g_1=g\cdot z_{21}(a,1)\in G(I,I)$. Indeed,
\[
\begin{pmatrix}
1+a & b \\ c & 1+d
\end{pmatrix} \xmapsto{\cdot z_{21}(a,1)}
\begin{pmatrix}
1+ba-a^2 & b-a-ba-a^2 \\ c+a+ad-ac & 1+bc-ac
\end{pmatrix} \in G(I,I).
\]
Now for a matrix $g'=\begin{psmallmatrix}1+a & b \\ c & 1+d\end{psmallmatrix}\in G(I,I)$ the matrix $x_{21}(-c)\cdot g'\cdot x_{12}(-b)\in\SL\left(2,\mathcal{O}_S,I^2\right)$. For any root system $\Phi\neq\rC_\ell$ of rank $\geqslant2$ one has $\E\left(\Phi,\mathcal{O}_S,I^2\right)\leqslant\E(\Phi,I)$, and this shows that $g$ has a bounded expression in terms of $z_\alpha$.

In case $\Phi=\rC_\ell$, one can compute
\[ \det(g_1)=a^3d-3a^2bc+a^2bd+ab^2c+a^3+a^2b+a^2d+2abc-abd+1 \]
to see that $2abc-abd\in II^{\indexbox{2}}$, so
\[ g_2=x_{21}(-a-c)\cdot g_1\cdot x_{12}(a-b)\equiv
\begin{pmatrix}
1+ab-a^2 & -ab-a^2 \\ ad-ac-abc & 1-ab+a^2
\end{pmatrix}\bmod II^{\indexbox{2}}. \]
Now for $g_3=g_2\cdot z_{12}\left(a^2-ab,1\right)$ one has
\begin{align*}
& g_3\equiv\begin{pmatrix} 1 & -2ab \\ -abc-a^2+ab-ac+ad & 1 \end{pmatrix}\bmod II^{\indexbox{2}},\\
& g_4=x_{12}(2ab)\cdot g_3\equiv x_{21}\left(-abc-a^2+ab-ac+ad\right)\bmod II^{\indexbox{2}}.
\end{align*}
Thus $g_4\cdot x_{21}(*)\in\SL\left(2,\mathcal{O}_S,II^{\indexbox{2}}\right)\leqslant\E(\rC_\ell,I)$ and can be expressed as a bounded product of $x_\alpha$.
\end{proof}